%%% XeLaTeX-article %%%
%# -*- coding: utf-8 -*-
%!TEX encoding = UTF-8 Unicode
%!TEX TS-program = xelatex  
%---------------------虽然加了%还是要保留!

\documentclass[17pt]{beamer}
\mode<presentation>
{
\usetheme[width=40pt]{Hannover}
\usecolortheme[]{dove}
\usefonttheme[]{structurebold}
\setbeameroption{hide notes}
}

\usepackage{fontspec}
\usepackage{polyglossia}
\setmainfont{Arial} %设置主字体
\newfontfamily\sanskritfont[Script=Devanagari,Mapping=romantodevanagari,Scale=1.15]{Sanskrit 2003}             %输出天城体
%\newfontfamily\sanskritfont[Mapping=tex-text]{Times New Roman}              %输出转写
\doublehyphendemerits=-10000
\newcommand{\skt}[1]{{\sanskritfont{#1}}} %输出天城体
\newcommand{\skttrans}[1]{{\skt{#1}~#1}}  %输出天城体和转写
%----------------------------------------------------设置梵文输入方法 danda । ॥

\usepackage[UTF8,fontset=windows]{ctex}
\usepackage{amsmath}
%----------------------------------------------------设置中文环境

\usepackage{graphicx}
\usepackage{flafter} 
\graphicspath{{pic/}}
\usepackage{booktabs} 
\usepackage{nicematrix}
\newenvironment{indentlist}
  {\begin{list}{}{\setlength{\leftmargin}{2em}\setlength{\itemsep}{0.5em}}}
  {\end{list}}
%-----------------------------------------插图表格

\usepackage{hyperref} 
\usepackage[dvipsnames]{xcolor}
\usepackage{colortbl}
\definecolor{light-gray}{gray}{0.9}
%------------------------------颜色

\newcommand{\verbroot}[1]{\textcolor{red}{$\sqrt{}$#1}}
\newcommand{\sktroot}[1]{{\verbroot{\skt{#1}}}}
\newcommand{\skttransroot}[1]{{\sktroot{#1}~\textcolor{red}{#1}}}

\newcommand{\nounstem}[1]{\textcolor{red}{#1\nobreakdash-}}
\newcommand{\sktnounstem}[1]{{\textcolor{red}{\skt{#1\nobreakdash-}}}}
\newcommand{\skttransnounstem}[1]{{\sktnounstem{#1}~\nounstem{#1}}}

\newcommand{\verbstem}[1]{\textcolor{blue}{#1\nobreakdash-}}
\newcommand{\sktverbstem}[1]{{\textcolor{blue}{\skt{#1\nobreakdash-}}}}
\newcommand{\skttransverbstem}[1]{{\sktverbstem{#1}~\verbstem{#1}}}

\newcommand{\wordending}[1]{\textcolor{Orange}{\nobreakdash-#1}}
\newcommand{\sktending}[1]{{\textcolor{Orange}{\skt{-#1}}}}
\newcommand{\skttransending}[1]{{\sktending{#1}~\wordending{#1}}}

\newcommand{\fullpada}[1]{\textcolor{OliveGreen}{#1}}
\newcommand{\sktpada}[1]{{\textcolor{OliveGreen}{\skt{#1}}}}
\newcommand{\skttranspada}[1]{{\sktpada{#1}~\fullpada{#1}}}

\newcommand{\pratyaya}[1]{\textcolor{Plum}{#1}}
\newcommand{\sktpratyaya}[1]{{\textcolor{Plum}{\skt{#1}}}}
\newcommand{\skttranspratyaya}[1]{{\sktpratyaya{#1}~\pratyaya{#1}}}

\newcommand{\reconstruction}[1]{\textcolor{gray}{*#1}}
\newcommand{\fullsentence}[1]{\textcolor{MidnightBlue}{#1}}
\newcommand{\sktsentence}[1]{\textcolor{MidnightBlue}{\skt{#1}}}

\newcommand{\veryimportant}[1]{\textcolor{red}{#1}}
\newcommand{\important}[1]{\textcolor{blue}{#1}}
\newcommand{\notsoimportant}[1]{\textcolor{gray}{#1}}
%-------------------------------------------词根等标颜色

\title{{梵语提高}}
\subtitle{32. 第一、二人称代词}
\author[张雪杉]{文学院~~张雪杉 \\ zhangxueshan@sdnu.edu.cn}
\date{}

\begin{document}	

\begin{frame}
  \titlepage
\end{frame}

\begin{frame}
  \frametitle{本节内容}
  \small
  \tableofcontents
\end{frame}

\section{上节作业}

\begin{frame}{第30章练习5}
  \raggedright
  \small
  \begin{verse}
    \skt{1) narāṇāṃ jīvitamantavaddevānāṃ tvanantavadevetyṛṣibhirabhāṣyata ।}   \\
    \skt{2) pitarau gāyantyā duhitrānandetām ।}   \\
    \skt{3) bhrāturdoṣānkṣamethāḥ ।}   \\
    \skt{4) kathaṃ devaṃ jānāsīti pṛṣṭo gururdevā rūpavanto 'svedā achāyā ajarā animiṣāśceti pratyabhāṣata ।} \\
    \skt{5) \mbox{daṇḍinau pakṣiṇo daṇḍābhyāmudapātayatām ।} annaṃ khādituṃ bhuvyupāviśetām ।}  \\
    \skt{6) gāyantī kanyā svasṛbhyāmaśasyata ।} \\

  \end{verse}
\end{frame}  

\begin{frame}{第30章练习5}
  \raggedright
  \small
  \begin{verse}
    \skt{7) \mbox{yadannaṃ mātaraḥ pecustatkuto nākhādadhvam ।}}\\
    \skt{8) pure vasato janānugrebhyaḥ śatrubhyo rakṣemahītyuktvā tānarakṣāmahi ।} \\
    \skt{9) senayorubhayoḥ kṣatriyau balavattamāvayudhyetām । anyonyaṃ jaghantuḥ ।} \\
    \skt{10) netā karmabhirjñāyate ।} \\
    \skt{11) aśvo bālaṃ bhṛtavānpitrāśasyata ।} \\
  \end{verse}
\end{frame}

\begin{frame}{第31章练习6}
  \raggedright
  \small
  \begin{verse}
    \skt{1) mahākarmāṇaṃ kṣatriyaṃ senayormūrdhni sthitamajānīmahi ।}   \\
    \skt{2) pitarāvakīrtikaraṃ putramanindetām ।}   \\
    \skt{3) yuddhamantakaramasti । alaṃ yuddheneti pitovāca ।}   \\
    \skt{4) prathamaṃ na sadā śreṣṭhamiti dhīmānuktvānyadyatnamakuruta ।}  \\
    \skt{5) apyannamabhuṅgdhvamiti pitā putrānpṛṣṭavān । \mbox{adhunā kaveḥ kathāḥ śrotumiyāmetyavadata ।}}  \\
    \skt{6) na kadā cidyuddhāya kalpiṣyatha iti kṣatriyaḥ kumārāvabalāvabravīt ।} \\
  \end{verse}
\end{frame}

\begin{frame}{第31章练习6}
  \raggedright
  \small
  \begin{verse}
    \skt{7) bālā nadyāmasnāyanta । alaṃ snānena iti mātā bhāṣitvā tāngṛhamanayat ।} \\
    \skt{8) devāḥ sarvakarāḥ sarvavidaśca । devānpūjayā namasā ca juhvīmahi ।} \\
    \skt{9) rājā prajākāmaḥ kathamaprajo jīvitaṃ bhuñjīteti papraccha ।} \\
    \skt{10) mahābuddhergurorvacanāni śṛṇuyāveti bāle 'cintayatām ।} \\
    \skt{11) ciraṃ suptvā bālo 'cirāduttiṣṭheyamityavadat । annakāma udatiṣṭhat ।} \\
  \end{verse}
\end{frame}

\subsection{复习}
\begin{frame}{\insertsubsection }
  \small
  \centering
  \begin{itemize}
    \item \textbf{中间语态派生语尾:} \\用于未完成时与祈愿语气的中间语态\\和被动语态(很少见)
  \end{itemize} 
  \begin{tabular}{@{}cccc@{}} % 4 columns
      & 单数  & 双数  & 复数 \\
    第一人称 & \wordending{i/\nobreakdash-(y)a} & \wordending{vahi}  & \wordending{mahi}  \\
    第二人称 & \wordending{thāḥ} & \wordending{thām} & \wordending{dhvam}   \\
    第三人称 & \wordending{ta} & \wordending{tām} & \wordending{(n)ta/\nobreakdash-ran}  \\
  \end{tabular}   
\end{frame}

\section{代词 III}

\subsection{第一、二人称代词}
\begin{frame}{第一人称代词 \nounstem{asmad}}
  \small
  \centering
  \begin{NiceTabular}{llll}[hvlines]
    \CodeBefore \rowcolor{light-gray}{1} \Body
    格 & 单数 (Sg) & 双数 (Du) & 复数 (Pl) \\
    主 & \fullpada{aham} & \fullpada{āvām} & \fullpada{vayam} \\
    业 & \fullpada{mām, mā} & \fullpada{āvām} & \fullpada{asmān, naḥ} \\
    具 & \fullpada{mayā} & \fullpada{āvābhyām} & \fullpada{asmābhiḥ} \\
    为 & \fullpada{mahyam, me} & \fullpada{āvābhyām} & \fullpada{asmabhyam, naḥ} \\
    从 & \fullpada{mat} & \fullpada{āvābhyām} & \fullpada{asmat} \\
    属 & \fullpada{mama, me} & \fullpada{āvayoḥ} & \fullpada{asmākam, naḥ} \\
    依 & \fullpada{mayi} & \fullpada{āvayoḥ} & \fullpada{asmāsu} \\
  \end{NiceTabular}
\end{frame}

\begin{frame}{第二人称代词 \nounstem{yuṣmad}}
  \small
  \centering
  \begin{NiceTabular}{llll}[hvlines]
    \CodeBefore \rowcolor{light-gray}{1} \Body
    格 & 单数 (Sg) & 双数 (Du) & 复数 (Pl) \\
    主 & \fullpada{tvam} & \fullpada{yuvām} & \fullpada{yūyam} \\
    业 & \fullpada{tvām, tvā} & \fullpada{yuvām} & \fullpada{yuṣmān, vaḥ} \\
    具 & \fullpada{tvayā} & \fullpada{yuvābhyām} & \fullpada{yuṣmābhiḥ} \\
    为 & \fullpada{tubhyam, te} & \fullpada{yuvābhyām} & \fullpada{yuṣmabhyam, vaḥ} \\
    从 & \fullpada{tvat} & \fullpada{yuvābhyām} & \fullpada{yuṣmat} \\
    属 & \fullpada{tava, te} & \fullpada{yuvayoḥ} & \fullpada{yuṣmākam, vaḥ} \\
    依 & \fullpada{tvayi} & \fullpada{yuvayoḥ} & \fullpada{yuṣmāsu} \\
  \end{NiceTabular}
\end{frame}

\subsection{形态要点}
\begin{frame}{语法要点}
  \small
  \begin{itemize}
    \item \veryimportant{无性区别}:\\第一、二人称对所有词性形式相同。
    \item \veryimportant{词干形式}:
      \begin{itemize}
        \item 第一人称:\nounstem{mat}
        \item 第二人称:\nounstem{tvat}
      \end{itemize}
    \item \important{附着形式 (Enclitics)}:表中逗号后的形式(如 \fullpada{me, naḥ, te, vaḥ})为轻读形式。
      \begin{itemize}
        \item 不能放在句首。
        \item 用于非强调语境。
      \end{itemize}
    \item Ruki连声:\nounstem{asmat} vs. \nounstem{yuṣmat}。
  \end{itemize}
\end{frame}

\section{本节作业}

\begin{frame}{\insertsection }
  \begin{itemize}
    \item 第30章练习3
    \item 学习通作业提交pdf
    \item 截止时间12月31日
  \end{itemize}
\end{frame}  

\end{document}	