%%% XeLaTeX-article %%%
%# -*- coding: utf-8 -*-
%!TEX encoding = UTF-8 Unicode
%!TEX TS-program = xelatex  
%---------------------虽然加了%还是要保留!

\documentclass[17pt]{beamer}
\mode<presentation>
{
\usetheme[width=40pt]{Hannover}
\usecolortheme[]{dove}
\usefonttheme[]{structurebold}
\setbeameroption{hide notes}
}

\usepackage{fontspec}
\usepackage{polyglossia}
\setmainfont{Arial} %设置主字体
\newfontfamily\sanskritfont[Script=Devanagari,Mapping=romantodevanagari,Scale=1.15]{Sanskrit 2003}             %输出天城体
%\newfontfamily\sanskritfont[Mapping=tex-text]{Times New Roman}              %输出转写
\doublehyphendemerits=-10000
\newcommand{\skt}[1]{{\sanskritfont{#1}}} %输出天城体
\newcommand{\skttrans}[1]{{\skt{#1}~#1}}  %输出天城体和转写
%----------------------------------------------------设置梵文输入方法 danda । ॥

\usepackage[UTF8,fontset=windows]{ctex}
\usepackage{amsmath}
%----------------------------------------------------设置中文环境

\usepackage{graphicx}
\usepackage{flafter} 
\graphicspath{{pic/}}
\usepackage{booktabs} 
\usepackage{nicematrix}
\newenvironment{indentlist}
  {\begin{list}{}{\setlength{\leftmargin}{2em}\setlength{\itemsep}{0.5em}}}
  {\end{list}}
%-----------------------------------------插图表格

\usepackage{hyperref} 
\usepackage[dvipsnames]{xcolor}
\usepackage{colortbl}
\definecolor{light-gray}{gray}{0.9}
%------------------------------颜色

\newcommand{\verbroot}[1]{\textcolor{red}{$\sqrt{}$#1}}
\newcommand{\sktroot}[1]{{\verbroot{\skt{#1}}}}
\newcommand{\skttransroot}[1]{{\sktroot{#1}~\textcolor{red}{#1}}}

\newcommand{\nounstem}[1]{\textcolor{red}{#1\nobreakdash-}}
\newcommand{\sktnounstem}[1]{{\textcolor{red}{\skt{#1\nobreakdash-}}}}
\newcommand{\skttransnounstem}[1]{{\sktnounstem{#1}~\nounstem{#1}}}

\newcommand{\verbstem}[1]{\textcolor{blue}{#1\nobreakdash-}}
\newcommand{\sktverbstem}[1]{{\textcolor{blue}{\skt{#1\nobreakdash-}}}}
\newcommand{\skttransverbstem}[1]{{\sktverbstem{#1}~\verbstem{#1}}}

\newcommand{\wordending}[1]{\textcolor{Orange}{\nobreakdash-#1}}
\newcommand{\sktending}[1]{{\textcolor{Orange}{\skt{-#1}}}}
\newcommand{\skttransending}[1]{{\sktending{#1}~\wordending{#1}}}

\newcommand{\fullpada}[1]{\textcolor{OliveGreen}{#1}}
\newcommand{\sktpada}[1]{{\textcolor{OliveGreen}{\skt{#1}}}}
\newcommand{\skttranspada}[1]{{\sktpada{#1}~\fullpada{#1}}}

\newcommand{\pratyaya}[1]{\textcolor{Plum}{#1}}
\newcommand{\sktpratyaya}[1]{{\textcolor{Plum}{\skt{#1}}}}
\newcommand{\skttranspratyaya}[1]{{\sktpratyaya{#1}~\pratyaya{#1}}}

\newcommand{\reconstruction}[1]{\textcolor{gray}{*#1}}
\newcommand{\fullsentence}[1]{\textcolor{MidnightBlue}{#1}}
\newcommand{\sktsentence}[1]{\textcolor{MidnightBlue}{\skt{#1}}}

\newcommand{\veryimportant}[1]{\textcolor{red}{#1}}
\newcommand{\important}[1]{\textcolor{blue}{#1}}
\newcommand{\notsoimportant}[1]{\textcolor{gray}{#1}}
%-------------------------------------------词根等标颜色

\title{{梵语入门}}
\subtitle{11. 词间辅音连声}
\author[张雪杉]{文学院~~张雪杉 \\ zhangxueshan@sdnu.edu.cn}
\date{}
%\institute{}

\begin{document}	

\begin{frame}
  \titlepage
\end{frame}

\begin{frame}{默写字母成绩}  
  \centering
  \includegraphics[width=\textwidth]{midtermdistribution.png} %
\end{frame}

\begin{frame}{tha 和 ya}  
  \centering
  \includegraphics[width=\textwidth]{tha_and_ya.png} %
\end{frame}

\begin{frame}
  \frametitle{本节内容}
  \small
  \tableofcontents
\end{frame}

\section{上节作业}

\begin{frame}{第十章练习2}
  \small
  \raggedright
  \begin{verse}
    \skt{1) janāḥ vigacchanti । narāḥ kṣetre bālāḥ ca gṛhe gacchanti ।} \\
    \skt{2) narāḥ bhāryāḥ bālāḥ ca ślokaiḥ devān saṃpūjayanti ।}  \\
    \skt{3) vyāghrāt bālāḥ aśvaiḥ saha gṛhaṃ saṃdravanti khagāḥ ca utpatanti ।}   \\
    \skt{4) īśvaram avanatya kumārāḥ samuttiṣṭhanti ।}  \\
    \skt{5) prājñaḥ nṛpaḥ prajānāṃ cintāḥ prapaśyati ।}  \\
  \end{verse}
\end{frame}  

\begin{frame}{第十章练习2}
  \small
  \raggedright
  \begin{verse}
    \skt{6) kṣatriyaḥ na atijīvati janāḥ ca śocanti ।}   \\
    \skt{7) īśvarasya vacanāni abhiśrutya kanye āgacchataḥ ।}  \\
    \skt{8) priyāḥ cintāḥ vismṛtya kanyā uttiṣṭhati ca nagaraṃ ca pratigacchati ।}   \\
    \skt{9) aśvān kṣetrebhyaḥ saṃnīya gṛhe sīdāmaḥ kathāḥ ca kathayāmaḥ ।}   \\
    \skt{10) janāḥ vyāghreṇa viluptaṃ bālam anuśocanti ।}   \\
    \mbox{\skt{11) bāle gṛhāt vyapagatya kṣetrāṇi upadravataḥ ।}}   \\
  \end{verse}
\end{frame}  


\subsection{复习}

\begin{frame}{动词前缀}
  \small
  \centering
    \begin{tabular}{@{}rlrl@{}} % 6 columns
      \pratyaya{ati\nobreakdash-} & 从上面,从旁边 & \pratyaya{ud\nobreakdash-} & 在上面,来自  \\
      \pratyaya{adhi\nobreakdash-} & 从上面,在上面  & \pratyaya{upa\nobreakdash-} & 向着   \\
      \pratyaya{anu\nobreakdash-} & 在后面,沿着 & \pratyaya{ni\nobreakdash-} & 向下面,向里面 \\
      \pratyaya{antar\nobreakdash-} & 在中间 & \pratyaya{nis\nobreakdash-} & 出来 \\
      \pratyaya{apa\nobreakdash-} & 离开 & \pratyaya{parā\nobreakdash-} & 离开 \\
      \pratyaya{api\nobreakdash-} & 朝……而去 & \pratyaya{pari\nobreakdash-} & 围绕着 \\
      \pratyaya{abhi\nobreakdash-} & 朝着 & \pratyaya{pra\nobreakdash-} & 向前 \\
      \pratyaya{ava\nobreakdash-} & 从……往下 & \pratyaya{prati\nobreakdash-} & 阻挡,回来 \\
      \pratyaya{ā\nobreakdash-} & 向着,来 & \pratyaya{vi\nobreakdash-} & 离开,散开 \\
       &  & \pratyaya{sam\nobreakdash-} & 同,一起 \\
    \end{tabular}
\end{frame}

\begin{frame}{词内元音连声}
 % \small
  \begin{itemize}
    \item 简单元音同类融合为长音\\
    \pratyaya{upa\nobreakdash-ā\nobreakdash-}\verbroot{nī} \fullpada{upānayati}  \\
    \item i/u加不同类时变成对应半元音\\
    \pratyaya{vi\nobreakdash-apa\nobreakdash-}\verbroot{nī} \fullpada{vyapanayati}  \\
    \pratyaya{adhi\nobreakdash-ā\nobreakdash-}\verbroot{gam} \fullpada{adhyāgacchati}  \\
    \item a/ā加不同类时升一级\\ 
    \pratyaya{pra\nobreakdash-ud\nobreakdash-}\verbroot{dhṛ} \fullpada{proddharati}  \\
  \end{itemize}
\end{frame}

\section{词间连声}
\begin{frame}{\insertsection }
  \small
  \tableofcontents[currentsection]
\end{frame}

\subsection{词间连声总则}
\begin{frame}{\insertsubsection }
  \small
  \centering
  \begin{itemize}
    \item 前词尾音和后词首音相遇发生变化。
    \item 一般前音会变,有时后音也变。
    \item 辅音会被后音同化,而且要连写。
    \item 止音 ḥ 变化较特殊,前后原形都要看。
    \item 元音遇元音常融合,遇辅音不变。

  \end{itemize}
\end{frame}

\begin{frame}{词间辅音连声总表}  
  \centering
  \includegraphics[width=0.75\textwidth]{consonantsandhi1.png} %
\end{frame}

\subsection{落尾辅音}
\begin{frame}{\insertsubsection}  
  \centering
  \includegraphics[width=0.5\textwidth]{consonantending.png} %
\end{frame}

\begin{frame}{词尾辅音变化}
  \small
  \begin{itemize}
    \item 多个辅音作尾音,只能留住第一个。 \\
    \verbroot{as} \reconstruction{sants} $\to$ \fullpada{san}  \\
    \item k ṭ t p ṅ n m,以及止声 (ḥ) 可落尾,\\其余都要起变化。 
    \item k ṭ t p 同行变,腭音变喉 j 可 ṭ。 \\
    \reconstruction{tad} $\to$ \fullpada{tat} ~~ \reconstruction{devarāj} $\to$ \fullpada{devarāṭ}  
    \item ś ṣ h 变 ṭ 少变 k, r s 要变 visarga(ḥ)。 \\
    \reconstruction{ṣaṣ} $\to$ \fullpada{ṣaṭ} ~~ \reconstruction{diś} $\to$ \fullpada{dik}  
  \end{itemize}
\end{frame}

\subsection{辅音连声}
\begin{frame}{词间辅音连声总表}  
  \centering
  \includegraphics[width=0.75\textwidth]{consonantsandhi1.png} %
\end{frame}

\subsection{~~~词尾清音}
\begin{frame}{词尾清音}
  \small
  \begin{itemize}
    \item 清遇清来不变化,遇浊浊化,鼻鼻化。 \\
    \fullsentence{tat tat}  \sktsentence{tattat} \\
    \reconstruction{āsit rājā} $\to$ \fullsentence{āsid rājā} \sktsentence{āsidrājā}  \\
    \reconstruction{abhavat atra} $\to$ \fullsentence{abhavad atra} \sktsentence{abhavadatra}  \\
    \reconstruction{tat na} $\to$ \fullsentence{tan na} \sktsentence{tanna} \\
    \reconstruction{tat mṛṣā} $\to$ \fullsentence{tan mṛṣā} \sktsentence{tanmṛṣā}\\
    \reconstruction{vāk me} $\to$ \fullsentence{vāṅ me} \sktsentence{vāṅme} \\
    \item 清音遇 h 影响它,h 变相应送气浊。 \\
    \reconstruction{tat hi} $\to$ \fullsentence{tad dhi} \sktsentence{taddhi} \\
    \reconstruction{vāk hi} $\to$ \fullsentence{vāg ghi} \sktsentence{vāgghi} \\
  \end{itemize}
\end{frame}

\begin{frame}{词尾清音}
  \small
  \begin{itemize}
    \item t 同化于腭、顶、l,遇 ś 变 c,ś 变ch。 \\
    \reconstruction{tat ca} $\to$ \fullsentence{tac ca} \sktsentence{tacca} \\
    \reconstruction{tat jalam} $\to$ \fullsentence{taj jalam} \sktsentence{tajjalam} \\  
    \reconstruction{tat labhate} $\to$ \fullsentence{tal labhate} \sktsentence{tallabhate} \\
    \reconstruction{tat śrutvā} $\to$ \fullsentence{tac chrutvā} \sktsentence{tacchrutvā} \\
  \end{itemize}
\end{frame}

\subsection{~~~词尾鼻音}
\begin{frame}{词尾鼻音}
  \small
  \begin{itemize}
    \item 前短元音后元音,除 m 鼻音要重复。 \\
    \reconstruction{āsan atra} $\to$ \fullsentence{āsann atra} \sktsentence{āsannatra} \\
    \reconstruction{pratyaṅ āsīnaḥ} $\to$ \fullsentence{pratyaṅṅ āsīnaḥ} \\
    \hspace*{4cm} \sktsentence{pratyaṅṅāsīnaḥ} \\
    \item 只要遇辅音,尾 m 变随韵。 \\
    \reconstruction{tam ca} $\to$ \fullsentence{taṃ ca} \sktsentence{taṃ ca} \\
  \end{itemize}
\end{frame}

\begin{frame}{词尾鼻音}
  \small
  \begin{itemize}
    \item n 遇浊的腭、顶、ś,被其同化, ś 可 ch。 \\
    \reconstruction{tān janān} $\to$ \fullsentence{tāñ janān} \sktsentence{tāñjanān} \\
    \reconstruction{tān śaśān} $\to$ \fullsentence{tāñ śaśān} \sktsentence{tāñśaśān} \\
    \hspace*{2.2cm} 或 \fullsentence{tāñ chaśān} \sktsentence{tāñchaśān}\\
    \item n 遇清的腭、顶、齿,变随韵加同位咝。 \\
    \reconstruction{tān ca} $\to$ \fullsentence{tāṃś ca} \sktsentence{tāṃśca} \\
    \reconstruction{tān tān} $\to$ \fullsentence{tāṃs tān} \sktsentence{tāṃstān} \\
    \item n 在 l 前鼻化 l̃ ({\sanskritfont{ल्ँ}}),也可写作 ṃl 或 ṃ 。 \\
    \reconstruction{tān lokān} $\to$ \fullsentence{tā\~{l} lokān / tāṃl lokān} \\
    \hspace*{3cm}\fullsentence{\sanskritfont{ताँल्लोकान्}} ~ \sktsentence{tāṃllokān} \\
  \end{itemize}
\end{frame}

\subsection{课堂练习}
\begin{frame}{\insertsubsection }
  \raggedright
  \begin{verse}
    \skt{1) nagarādgrāmaṃ gacchati ।}   \\
    \skt{2) grāmaṃ nagarānna gacchati ।}  \\
    \skt{3) nagarādiha gacchati ।}   \\
    \skt{4) nagarāddhi gacchati ।}   \\
    \skt{5) nagarādgrāmamāgacchati ।}   \\
    \skt{6) nagarādapagacchati ।}   \\
    \skt{7) buddhvā । bud\-dhvā ।}   \\
    \mbox{\skt{8) etānhatānyudhe nṛpa śocitumarhasi ।}}  \\
  \end{verse}
\end{frame}  

\section{本节作业}

\begin{frame}{\insertsection }
  \begin{itemize}
    \item
      第11章练习5
    \item
      阅读教材第11课相关内容
    \bigskip
    \item
      现在请做学习通\nobreakdash-章节\nobreakdash-课后问卷
  \end{itemize}
\end{frame}  

\end{document}	