%%% XeLaTeX-article %%%
%# -*- coding: utf-8 -*-
%!TEX encoding = UTF-8 Unicode
%!TEX TS-program = xelatex  
%---------------------虽然加了%还是要保留!

\documentclass[17pt]{beamer}
\mode<presentation>
{
\usetheme[width=40pt]{Hannover}
\usecolortheme[]{dove}
\usefonttheme[]{structurebold}
\setbeameroption{hide notes}
}

\usepackage{fontspec}
\setmainfont{Arial} %设置主字体
\newfontfamily\sanskritfont[Script=Devanagari,Mapping=romantodevanagari,Scale=1.15]{Sanskrit 2003}             %输出天城体
%\newfontfamily\sanskritfont[Mapping=tex-text]{Times New Roman}              %输出转写
\doublehyphendemerits=-10000
\newcommand{\skt}[1]{{\sanskritfont{#1}}} %输出天城体
\newcommand{\skttrans}[1]{{\skt{#1}~#1}}  %输出天城体和转写
%----------------------------------------------------设置梵文输入方法 danda । ॥

\usepackage[UTF8,fontset=windows]{ctex}
\usepackage{amsmath}
%----------------------------------------------------设置中文环境

\usepackage{graphicx}
\usepackage{flafter} 
\graphicspath{{pic/}}
\usepackage{booktabs} 
\usepackage{nicematrix}
%-----------------------------------------插图表格

\usepackage{hyperref} 
\usepackage[dvipsnames]{xcolor}
\usepackage{colortbl}
\definecolor{light-gray}{gray}{0.9}
%------------------------------颜色

\newcommand{\verbroot}[1]{\textcolor{red}{$\sqrt{}$#1}}
\newcommand{\sktroot}[1]{{\verbroot{\skt{#1}}}}
\newcommand{\skttransroot}[1]{{\sktroot{#1}~\textcolor{red}{#1}}}

\newcommand{\nounstem}[1]{\textcolor{red}{#1\nobreakdash-}}
\newcommand{\sktnounstem}[1]{{\textcolor{red}{\skt{#1\nobreakdash-}}}}
\newcommand{\skttransnounstem}[1]{{\sktnounstem{#1}~\nounstem{#1}}}

\newcommand{\verbstem}[1]{\textcolor{blue}{#1\nobreakdash-}}
\newcommand{\sktverbstem}[1]{{\textcolor{blue}{\skt{#1\nobreakdash-}}}}
\newcommand{\skttransverbstem}[1]{{\sktverbstem{#1}~\verbstem{#1}}}

\newcommand{\wordending}[1]{\textcolor{Orange}{\nobreakdash-#1}}
\newcommand{\sktending}[1]{{\textcolor{Orange}{\skt{-#1}}}}
\newcommand{\skttransending}[1]{{\sktending{#1}~\wordending{#1}}}

\newcommand{\fullpada}[1]{\textcolor{OliveGreen}{#1}}
\newcommand{\sktpada}[1]{{\textcolor{OliveGreen}{\skt{#1}}}}
\newcommand{\skttranspada}[1]{{\sktpada{#1}~\fullpada{#1}}}

\newcommand{\pratyaya}[1]{\textcolor{Plum}{#1}}
\newcommand{\sktpratyaya}[1]{{\textcolor{Plum}{\skt{#1}}}}
\newcommand{\skttranspratyaya}[1]{{\sktpratyaya{#1}~\pratyaya{#1}}}

\newcommand{\veryimportant}[1]{\textcolor{red}{#1}}
\newcommand{\important}[1]{\textcolor{blue}{#1}}
\newcommand{\notsoimportant}[1]{\textcolor{gray}{#1}}
%-------------------------------------------词根等标颜色

\title{{梵语提高}}
\subtitle{19. 不带插入元音的动词语干(续) }
\author[张雪杉]{文学院~~张雪杉 \\ zhangxueshan@sdnu.edu.cn}
\date{}
%\institute{}


\begin{document}


\begin{frame}
  \titlepage
\end{frame}

\begin{frame}
  \frametitle{本节内容}
  \tableofcontents
\end{frame}

\section{上节作业}

\begin{frame}{第十八章练习3}
  \small
  \raggedright
  \begin{verse}
    \skt{1) pāpāḥ puraṃ dagdhuṃ na śaknuvantīti kṣatriyā vidanti ।} \\
    \skt{2) api janairdviṣṭaṃ kṣatriyaṃ \textcolor{red}{vetsīti} pṛṣṭvā naro mitreṇa sahāpaiti ।} \\
    \skt{3) puramāptuṃ na śaknuva iti bālau cintayataḥ ।} \\
    \skt{4) kanyā gṛhametuṃ na śaknuvanti bibhyati ca ।} \\
  \end{verse}
\end{frame}

\begin{frame}{第十八章练习3}
  \small
  \raggedright
  \begin{verse}
    \skt{5) vane vyāghraṃ viditvā narau bibhītaḥ ।} \\
    \skt{6) paurā īśvarasya dānāni bhuñjantīti kumāro vetti ।} \\
    \skt{7) api vane vastuṃ bhunakṣīti bālā naramapṛcchan ।} \\
    \skt{8) apyannaṃ bhuṅktheti bālānapṛccham । na bhuñjma iti pratyavadan । apyannaṃ bhunakṣīti bālāmapṛccham । annaṃ bhunajmīti bālā pratyavadat ।}
  \end{verse}
\end{frame}

\subsection{复习}
\begin{frame}{Athematic Verbs  语干变化}
  \small
  \raggedright
  \begin{itemize}
    \item 第二类:词根直接加词尾
    \item 第三类:重复语干加词尾
    \item 第五类:强语干加\pratyaya{\nobreakdash-no\nobreakdash-},弱语干加\pratyaya{\nobreakdash-nu\nobreakdash-}
    \item 第七类:强加\pratyaya{\nobreakdash-na\nobreakdash-},弱加\pratyaya{\nobreakdash-n\nobreakdash-},
    
    \hspace*{4em}加在落尾辅音前
    \item 第八类:强语干加\pratyaya{\nobreakdash-o\nobreakdash-},弱语干加\pratyaya{\nobreakdash-u\nobreakdash-}
    \item 第九类:强语干加\pratyaya{\nobreakdash-nā\nobreakdash-},弱语干加\pratyaya{\nobreakdash-nī\nobreakdash-/\nobreakdash-n\nobreakdash-}
  \end{itemize}
\end{frame}

\begin{frame}{现在时语尾/原始语尾}
  \centering
  \begin{tabular}{@{}llll@{}} % 4 columns
    & 单数  & 双数  & 复数 \\
    第一人称 & \wordending{mi} & \wordending{vaḥ}  & \wordending{maḥ}  \\
    第二人称 & \wordending{si} & \wordending{thaḥ} & \wordending{tha}  \\
    第三人称 & \wordending{ti} & \wordending{taḥ} & \cellcolor{green!20}\wordending{anti} \\
  \end{tabular} 
  \bigskip
  \begin{itemize}
    \item 用于现在时\notsoimportant{、简单将来时}
  \end{itemize}  
\end{frame}

\begin{frame}{派生语尾}
  \centering
  \begin{tabular}{@{}llll@{}} % 4 columns
    & 单数  & 双数  & 复数 \\
    第一人称 & \cellcolor{green!20}\wordending{am} & \wordending{va}  & \wordending{ma}  \\
    第二人称 & \cellcolor{yellow!30}\wordending{ḥ} & \wordending{tam} & \wordending{ta}  \\
    第三人称 & \cellcolor{yellow!30}\wordending{t} & \wordending{tām} & \cellcolor{green!20}\wordending{an/uḥ}  \\
  \end{tabular}  
  \bigskip
  \begin{itemize}
    \item 用于未完成时、祈愿语气\notsoimportant{、}
    
    \notsoimportant{不定过去时、条件式}
  \end{itemize}
\end{frame}

\section{不带插入元音的动词语干}
\begin{frame}{\insertsection }
    \tableofcontents[currentsection]
\end{frame}

\subsection{未完成时}
\begin{frame}{\insertsubsection}
  \small
  \begin{itemize}
    \item 主动语态单数是强语干
    \item 词干前加 \pratyaya{a\nobreakdash-}
  \end{itemize}

  \raggedright  
  \begin{tabular}{@{}llll@{}} % 4 columns
    \verbroot{hu (3)} & 单数  & 双数  & 复数 \\
    第一人称 & \cellcolor{light-gray}\fullpada{ajuh\textcolor{red}{av}am} & \fullpada{ajuhuva}  & \fullpada{ajuhuma}  \\
    第二人称 & \cellcolor{light-gray}\fullpada{ajuhoḥ} & \fullpada{ajuhutam} & \fullpada{ajuhuta}  \\
    第三人称 & \cellcolor{light-gray}\fullpada{ajuhot} & \fullpada{ajuhutām} & \fullpada{ajuh\textcolor{red}{v}an}  \\
  \end{tabular} 
  \bigskip
    
  \begin{tabular}{@{}llll@{}} % 4 columns
    \verbroot{dviṣ (2)} & 单数  & 双数  & 复数 \\
    第一人称 & \cellcolor{light-gray}\fullpada{adveṣam} & \fullpada{adviṣva} & \fullpada{adviṣma} \\
    第二人称 & \cellcolor{light-gray}\fullpada{adve\important{ṭ}} & \fullpada{adviṣṭam} & \fullpada{adviṣṭa} \\
    第三人称 & \cellcolor{light-gray}\fullpada{adve\important{ṭ}} & \fullpada{adviṣṭām} & \fullpada{adviṣan/adviṣur} \\
  \end{tabular}   
\end{frame}

\subsection{祈愿语气}
\begin{frame}{\insertsubsection}
  \small
  \begin{itemize}
    \item 全部弱语干,词尾前加 \pratyaya{\nobreakdash-yā-}
    \item 第三人称复数词尾直接是 \wordending{yuḥ}
  \end{itemize}
  \centering
  \begin{tabular}{@{}lllll@{}} % 5 columns
    & 一单  & 二单 & 三单 & 三复 \\
    二 & \fullpada{iyām} & \fullpada{iyāḥ} & \fullpada{iyāt}  & \fullpada{iyuḥ}  \\
    三 & \fullpada{juhuyām} & \fullpada{juhuyāḥ} & \fullpada{juhuyāt} & \fullpada{juhuyuḥ}  \\
    五 & \fullpada{sunuyām} & \fullpada{sunuyāḥ} & \fullpada{sunuyāt} & \fullpada{sunuyuḥ}  \\
    七 & \fullpada{rundhyām} & \fullpada{rundhyāḥ} & \fullpada{rundhyāt} & \fullpada{rundhyuḥ}  \\
    八 & \fullpada{tanuyām} & \fullpada{tanuyāḥ} & \fullpada{tanuyāt} & \fullpada{tanuyuḥ}  \\
    九 & \fullpada{vṛṇīyām} & \fullpada{vṛṇīyāḥ} & \fullpada{vṛṇīyāt} & \fullpada{vṛṇīyuḥ}  \\
  \end{tabular}   
\end{frame}


\section{常用不规则动词}
\begin{frame}{\insertsection }
    \tableofcontents[currentsection]
\end{frame}

\begin{frame}{\verbroot{as (2)} 是}
  %\small
  \begin{itemize}
    \item 现在时:强语干 \verbstem{as},弱语干 \verbstem{s}
  \end{itemize}
  \centering
  \begin{tabular}{@{}llll@{}} % 4 columns
    &   \multicolumn{3}{c}{现在时}  \\
    & 单  & 双 & 复 \\
    1st & \cellcolor{light-gray}\fullpada{asmi} & \fullpada{svaḥ} & \fullpada{smaḥ}  \\
    2nd & \cellcolor{light-gray}\fullpada{a\textcolor{red}{s}i} & \fullpada{sthaḥ} & \fullpada{stha} \\
    3rd & \cellcolor{light-gray}\fullpada{asti} & \fullpada{staḥ} & \fullpada{santi} \\
  \end{tabular}   
\end{frame}

\begin{frame}[fragile]
  \frametitle{\verbroot{as (2)} 是}
  \small
  \begin{itemize}
    \item 未完成时:语干全是 \verbstem{ās},二单三单加 \pratyaya{\nobreakdash-ī-}
    \item 祈愿语气:全是弱语干 \verbstem{s}
  \end{itemize}
  \centering
  \begin{NiceTabular}{lllllll}
    \CodeBefore
      \rectanglecolor{light-gray}{3-2}{5-4}
    \Body % 7 columns
    &   \multicolumn{3}{c}{未完成时} & \multicolumn{3}{c}{祈愿语气} \\
    & 单  & 双 & 复 & 单  & 双 & 复 \\
    1st & \fullpada{āsam} & \fullpada{āsva} & \fullpada{āsma} & \fullpada{syām} & \fullpada{syāva} & \fullpada{syāma}\\
    2nd & \fullpada{ās\veryimportant{ī}ḥ}  & \fullpada{āstam} & \fullpada{āsta} & \fullpada{syāḥ}  & \fullpada{syātam} & \fullpada{syāta} \\
    3rd & \fullpada{ās\veryimportant{ī}t} & \fullpada{āstām} & \fullpada{āsan} & \fullpada{syāt} & \fullpada{syātām} & \fullpada{syuḥ} \\
  \end{NiceTabular}   
\end{frame}

\begin{frame}{\verbroot{i (2)} 走}
  %\small
  \begin{itemize}
    \item 未完成时:语干全是 \verbstem{ai}
  \end{itemize}
  \centering
  \begin{NiceTabular}{@{}llll@{}} % 4 columns
    \CodeBefore
      \rectanglecolor{light-gray}{3-2}{5-4}
    \Body 
    &   \multicolumn{3}{c}{未完成时}  \\
    & 单  & 双 & 复 \\
    1st & \fullpada{āyam} & \fullpada{aiva} & \fullpada{aima}  \\
    2nd & \fullpada{aiḥ} & \fullpada{aitam} & \fullpada{aita} \\
    3rd & \fullpada{ait} & \fullpada{aitām} & \fullpada{āyan} \\
  \end{NiceTabular}   
\end{frame}

\begin{frame}[fragile]
  \frametitle{\verbroot{han (2)} 杀}
  \small
  \begin{itemize}
    \item 单数和第一人称都是强语干 \verbstem{han}
    \item 弱语干:辅音前是 \verbstem{ha},元音前是 \verbstem{ghn}
  \end{itemize}
  \centering
  \footnotesize
  \begin{NiceTabular}{lllllll}
    \CodeBefore
      \rectanglecolor{light-gray}{3-2}{3-7}
      \rectanglecolor{light-gray}{4-2}{5-2}
      \rectanglecolor{light-gray}{4-5}{5-5}
    \Body % 7 columns
    &   \multicolumn{3}{c}{现在时} & \multicolumn{3}{c}{未完成时} \\
    & 单  & 双 & 复 & 单  & 双 & 复 \\
    1 & \fullpada{hanmi} & \fullpada{hanvaḥ} & \fullpada{hanmaḥ} & \fullpada{ahanam} & \fullpada{ahanva} & \fullpada{ahanma}\\
    2 & \fullpada{haṃsi}  & \fullpada{hathaḥ} & \fullpada{hatha} & \fullpada{ahan}  & \fullpada{ahatam} & \fullpada{ahata} \\
    3 & \fullpada{hanti} & \fullpada{hataḥ} & \fullpada{ghnanti} & \fullpada{ahan} & \fullpada{ahatām} & \fullpada{aghnan} \\
  \end{NiceTabular}   
\end{frame}

\begin{frame}%[fragile]
  \frametitle{\verbroot{brū (2)} 说}
  \small
  \begin{itemize}
    \item 强语干辅音语尾前加 \pratyaya{\nobreakdash-ī\nobreakdash-}
  \end{itemize}
  \centering
  \begin{NiceTabular}{llll}
    \CodeBefore
      \rectanglecolor{light-gray}{3-2}{5-2}
      \rectanglecolor{light-gray}{8-2}{10-2}
    \Body % 7 columns
    &   \multicolumn{3}{c}{现在时}  \\
    & 单  & 双 & 复  \\
    1 & \fullpada{bravīmi} & \fullpada{brūvah} & \fullpada{brūmaḥ} \\
    2 & \fullpada{bravīsi}  & \fullpada{brūthaḥ} & \fullpada{brūtha} \\
    3 & \fullpada{bravīti} & \fullpada{brūtaḥ} & \fullpada{bruvanti}  \\
    \bigskip \\
    &    \multicolumn{3}{c}{未完成时} \\
    1  & \fullpada{abravam} & \fullpada{abrūva} & \fullpada{abrūma} \\
    2 & \fullpada{abravīḥ}  & \fullpada{abrūtam} & \fullpada{abrūta} \\
    3  & \fullpada{abravīt} & \fullpada{abrūtām} & \fullpada{abruvan} \\
  \end{NiceTabular}   
\end{frame}

\begin{frame}%[fragile]
  \frametitle{\verbroot{dā (3)} 给,\verbroot{dhā (3)} 放 }
  \small
  \begin{itemize}
    \item 弱语干去掉 \veryimportant{\nobreakdash-ā\nobreakdash-},三复语尾是 \wordending{ati}
  \end{itemize}
  \centering
  \begin{NiceTabular}{llll}
    \CodeBefore
      \rectanglecolor{light-gray}{3-2}{5-2}
      \rectanglecolor{light-gray}{7-2}{9-2}
    \Body % 7 columns
    &   \multicolumn{3}{c}{现在时}  \\
    \verbroot{dā} & 单  & 双 & 复  \\
    1 & \fullpada{dadāmi} & \fullpada{dadvaḥ} & \fullpada{dadmaḥ} \\
    2 & \fullpada{dadāsi}  & \fullpada{datthaḥ} & \fullpada{dattha} \\
    3 & \fullpada{dadāti} & \fullpada{dattaḥ} & \fullpada{dadati} \\
    \verbroot{dhā} & 单  & 双 & 复  \\
    1  & \fullpada{dadhāmi} & \fullpada{dadhvaḥ} & \fullpada{dadhmaḥ} \\
    2 & \fullpada{dadhāsi}  & \fullpada{dhatthaḥ} & \fullpada{dhattha} \\
    3  & \fullpada{dadhāti} & \fullpada{dhattaḥ} & \fullpada{dadhati} \\
  \end{NiceTabular}   
\end{frame}

\begin{frame}%[fragile]
  \frametitle{\verbroot{kṛ (8)} 做}
  \small
  \begin{itemize}
    \item 强语干 \verbstem{karo},弱语干 \verbstem{kuru}
    \item 弱语干的 \verbstem{\nobreakdash-u} 在语尾的 \wordending{v\nobreakdash-} 和 \wordending{m\nobreakdash-} 前消失
  \end{itemize}
  \centering
  \begin{NiceTabular}{llll}
    \CodeBefore
      \rectanglecolor{light-gray}{2-2}{4-2}
      \rectanglecolor{light-gray}{6-2}{8-2}
    \Body % 7 columns
    现在时 & 单  & 双 & 复  \\
    1 & \fullpada{karomi} & \fullpada{kurvaḥ} & \fullpada{kurmaḥ} \\
    2 & \fullpada{karoṣi} & \fullpada{kuruthaḥ} & \fullpada{kurutha} \\
    3 & \fullpada{karoti} & \fullpada{kurutaḥ} & \fullpada{kurvanti} \\
    未完成时 & 单  & 双 & 复  \\
    1 & \fullpada{akaravam} & \fullpada{akurva} & \fullpada{akurma} \\
    2 & \fullpada{akaroḥ}  & \fullpada{akurutam} & \fullpada{akuruta} \\
    3 & \fullpada{akarot} & \fullpada{akurutām} & \fullpada{akurvan} \\
  \end{NiceTabular}   
\end{frame}

\begin{frame}{\verbroot{stu (2)} 赞颂}
  %\small
  \begin{itemize}
    \item 强语干三合
  \end{itemize}
  \centering
  \begin{tabular}{@{}llll@{}} % 4 columns
    &   \multicolumn{3}{c}{现在时}  \\
    & 单  & 双 & 复 \\
    1st & \cellcolor{light-gray}\fullpada{staumi} & \fullpada{stuvaḥ} & \fullpada{stumaḥ}  \\
    2nd & \cellcolor{light-gray}\fullpada{stauṣi} & \fullpada{stuthaḥ} & \fullpada{stutha} \\
    3rd & \cellcolor{light-gray}\fullpada{stauti} & \fullpada{stutaḥ} & \fullpada{stuvanti} \\
  \end{tabular}   
\end{frame}

\begin{frame}%[fragile]
  \frametitle{第二类不分强弱语干动词}
  \small
  \begin{itemize}
    \item 以 \veryimportant{\nobreakdash-ā} 结尾的动词,如 \verbroot{yā (2)} 去
    \item \verbroot{svap (2)} 睡:辅音语尾前加 \pratyaya{\nobreakdash-i\nobreakdash-}
  \end{itemize}
  \centering
  \begin{NiceTabular}{llll}
    \CodeBefore
      %\rectanglecolor{light-gray}{3-2}{5-2}
      %\rectanglecolor{light-gray}{7-2}{9-2}
    \Body % 7 columns
    %&   \multicolumn{3}{c}{现在时}  \\
    \verbroot{yā} & 单  & 双 & 复  \\
    1 & \fullpada{yāmi} & \fullpada{yāvaḥ} & \fullpada{yāmaḥ} \\
    2 & \fullpada{yāsi}  & \fullpada{yāthaḥ} & \fullpada{yātha} \\
    3 & \fullpada{yāti} & \fullpada{yātaḥ} & \fullpada{yānti} \\
    \verbroot{svap} & 单  & 双 & 复  \\
    1  & \fullpada{svapimi} & \fullpada{svapivaḥ} & \fullpada{svapimaḥ} \\
    2 & \fullpada{svapiṣi}  & \fullpada{svapithaḥ} & \fullpada{svapitha} \\
    3  & \fullpada{svapiti} & \fullpada{svapitaḥ} & \fullpada{svapanti} \\
  \end{NiceTabular}   
\end{frame}

\begin{frame}%[fragile]
  \frametitle{注意连声}
  \small
  \centering
  \begin{itemize}
    \item \verbroot{vac (2)} 说 ~~ \verbroot{dviṣ (2)} 恨
  \end{itemize}
  \begin{NiceTabular}{llll}
    \CodeBefore
      \rectanglecolor{light-gray}{3-2}{3-4}
      \rectanglecolor{light-gray}{6-2}{8-2}
    \Body % 7 columns 
    &   \multicolumn{3}{c}{现在时}  \\
    \verbroot{vac} & 1  & 2 & 3  \\
    单数 & \fullpada{vacmi} & \fullpada{vakṣi} & \fullpada{vakti} \\
    \vspace{0.5cm} \\
    \verbroot{dviṣ} & 单  & 双 & 复 \\
    1  & \fullpada{dveṣmi} & \fullpada{dviṣvaḥ} & \fullpada{dviṣmaḥ} \\
    2 & \fullpada{dvekṣi}  & \fullpada{dviṣṭhaḥ} & \fullpada{dviṣṭha} \\
    3  & \fullpada{dveṣṭi} & \fullpada{dviṣṭaḥ} & \fullpada{dviṣanti} \\
  \end{NiceTabular}   
\end{frame}

\section{本节作业}

\begin{frame}{\insertsection }
  \begin{itemize}
    \item
      第十九章练习3
    \item
      勘误:第3题括号后加\skt{nṛpāya}
    \bigskip
    %\item
    %  现在请做学习通\nobreakdash-章节\nobreakdash-课后问卷
  \end{itemize}
\end{frame}  

\end{document}	
