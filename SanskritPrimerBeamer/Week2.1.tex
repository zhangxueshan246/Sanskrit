%%% XeLaTeX-article %%%
%# -*- coding: utf-8 -*-
%!TEX encoding = UTF-8 Unicode
%!TEX TS-program = xelatex  
%---------------------虽然加了%还是要保留!

\documentclass[17pt]{beamer}
\mode<presentation>
{
\usetheme[width=40pt]{Hannover}
\usecolortheme[]{dove}
\usefonttheme[]{structurebold}
\setbeameroption{hide notes}
}

\usepackage{fontspec}
\setmainfont{Arial} %设置主字体
\newfontfamily\sanskritfont[Script=Devanagari,Mapping=romantodevanagari,Scale=1.15]{Sanskrit 2003}             %输出天城体
%\newfontfamily\sanskritfont[Mapping=tex-text]{Times New Roman}              %输出转写
\doublehyphendemerits=-10000
\newcommand{\skt}[1]{{\sanskritfont{#1}}} %输出天城体
\newcommand{\skttrans}[1]{{\skt{#1}~#1}}  %输出天城体和转写
%----------------------------------------------------设置梵文输入方法 danda । ॥

\usepackage[UTF8,fontset=windows]{ctex}
\usepackage{amsmath}
%----------------------------------------------------设置中文环境

\usepackage{graphicx}
\usepackage{flafter} 
\graphicspath{{pic/}}
\usepackage{booktabs} 
\usepackage{nicematrix}
%-----------------------------------------插图表格

\usepackage{hyperref} 
\usepackage[dvipsnames]{xcolor}
\usepackage{colortbl}
\definecolor{light-gray}{gray}{0.9}
%------------------------------颜色

\newcommand{\verbroot}[1]{{$\sqrt{#1}$}}
\newcommand{\sktroot}[1]{{\verbroot{}\skt{#1}}}
\newcommand{\skttransroot}[1]{{\sktroot{#1}~#1}}
%---------------------------------------------------------------词根

\title{{梵语入门}}
\subtitle{19. 不带插入元音的动词语干(续) }
\author[张雪杉]{文学院~~张雪杉 \\ zhangxueshan@sdnu.edu.cn}
\date{}
%\institute{}


\begin{document}


\begin{frame}
  \titlepage
\end{frame}

\begin{frame}
  \frametitle{本节内容}
  \tableofcontents
\end{frame}

\section{上节作业}

\begin{frame}{第十八章练习3}
  \small
  \raggedright
  \begin{verse}
    \skt{1) pāpāḥ puraṃ dagdhuṃ na śaknuvantīti kṣatriyā vidanti ।} \\
    \skt{2) api janairdviṣṭaṃ kṣatriyaṃ \textcolor{red}{vetsīti} pṛṣṭvā naro mitreṇa sahāpaiti ।} \\
    \skt{3) puramāptuṃ na śaknuva iti bālau cintayataḥ ।} \\
    \skt{4) kanyā gṛhametuṃ na śaknuvanti bibhyati ca ।} \\
  \end{verse}
\end{frame}

\begin{frame}{第十八章练习3}
  \small
  \raggedright
  \begin{verse}
    \skt{5) vane vyāghraṃ viditvā narau bibhītaḥ ।} \\
    \skt{6) paurā īśvarasya dānāni bhuñjantīti kumāro vetti ।} \\
    \skt{7) api vane vastuṃ bhunakṣīti bālā naramapṛcchan ।} \\
    \skt{8) apyannaṃ bhuṅktheti bālānapṛccham । na bhuñjma iti pratyavadan । apyannaṃ bhunakṣīti bālāmapṛccham । annaṃ bhunajmīti bālā pratyavadat ।}
  \end{verse}
\end{frame}


\section{不带插入元音的动词语干}
\begin{frame}{\insertsection }
    \tableofcontents[currentsection]
\end{frame}

\subsection{复习}
\begin{frame}{Athematic Verbs  语干变化}
  \small
  \raggedright
  \begin{itemize}
    \item 第二类:词根直接加词尾
    \item 第三类:重复语干加词尾
    \item 第五类:强语干加\nobreakdash-no\nobreakdash-,弱语干加\nobreakdash-nu\nobreakdash-
    \item 第七类:强加\nobreakdash-na\nobreakdash-弱加\nobreakdash-n\nobreakdash-,
    
    \hspace*{4em}加在落尾辅音前
    \item 第八类:强语干加\nobreakdash-o\nobreakdash-,弱语干加\nobreakdash-u\nobreakdash-
    \item 第九类:强语干加\nobreakdash-nā\nobreakdash-,弱语干加\nobreakdash-nī\nobreakdash-/\nobreakdash-n\nobreakdash-
  \end{itemize}
\end{frame}

\begin{frame}{现在时语尾}
  \centering
  \begin{tabular}{@{}llll@{}} % 4 columns
    & 单数  & 双数  & 复数 \\
    第一人称 & -mi & -vaḥ  & -maḥ  \\
    第二人称 & -si & -thaḥ & -tha  \\
    第三人称 & -ti & -taḥ & -\textcolor{red}{a}nti  \\
  \end{tabular}   
\end{frame}

\subsection{派生语尾}
\begin{frame}{\insertsubsection}
  \centering
  \begin{tabular}{@{}llll@{}} % 4 columns
    & 单数  & 双数  & 复数 \\
    第一人称 & -\textcolor{red}{a}m & -va  & -ma  \\
    第二人称 & -\textcolor{blue}{ḥ} & -tam & -ta  \\
    第三人称 & -\textcolor{blue}{t} & -tām & -\textcolor{red}{a}n/-\textcolor{OliveGreen}{uḥ}  \\
  \end{tabular}   
\end{frame}

\subsection{未完成时}
\begin{frame}{\insertsubsection}
  \small
  \raggedright  
  \begin{tabular}{@{}llll@{}} % 4 columns
    \verbroot{}hu & 单数  & 双数  & 复数 \\
    第一人称 & \cellcolor{light-gray}ajuh\textcolor{red}{av}am & ajuhuva  & ajuhuma  \\
    第二人称 & \cellcolor{light-gray}ajuhoḥ & ajuhutam & ajuhuta  \\
    第三人称 & \cellcolor{light-gray}ajuhot & ajuhutām & ajuh\textcolor{red}{v}an  \\
  \end{tabular} 
  \bigskip
    
  \begin{tabular}{@{}llll@{}} % 4 columns
    \verbroot{}dviṣ & 单数  & 双数  & 复数 \\
    第一人称 & \cellcolor{light-gray}adveṣam & adviṣva & adviṣma \\
    第二人称 & \cellcolor{light-gray}adve\textcolor{blue}{ṭ} & adviṣṭam & adviṣṭa \\
    第三人称 & \cellcolor{light-gray}adve\textcolor{blue}{ṭ} & adviṣṭām & adviṣ\textcolor{OliveGreen}{an}/adviṣ\textcolor{OliveGreen}{ur} \\
  \end{tabular}   
\end{frame}

\subsection{祈愿语气}
\begin{frame}{\insertsubsection}
  \small
  \begin{itemize}
    \item 全部弱语干,词尾前加 \nobreakdash-yā\nobreakdash-
    \item 第三人称单数是 \nobreakdash-yuḥ
  \end{itemize}
  \centering
  \begin{tabular}{@{}lllll@{}} % 5 columns
    & 一单  & 二单 & 三单 & 三复 \\
    二 & iyām & iyāḥ & iyāt  & iyuḥ  \\
    三 & juhuyām & juhuyāḥ & juhuyāt & juhuyuḥ  \\
    五 & sunuyām & sunuyāḥ & sunuyāt & sunuyuḥ  \\
    七 & rundhyām & rundhyāḥ & rundhyāt & rundhyuḥ  \\
    八 & tanuyām & tanuyāḥ & tanuyāt & tanuyuḥ  \\
    九 & vṛṇīyām & vṛṇīyāḥ & vṛṇīyāt & vṛṇīyuḥ  \\
  \end{tabular}   
\end{frame}


\section{常用不规则动词}
\begin{frame}{\insertsection }
    \tableofcontents[currentsection]
\end{frame}

\begin{frame}{\verbroot{}as (II) 是}
  %\small
  \begin{itemize}
    \item 现在时:强语干as\nobreakdash-,弱语干s\nobreakdash-
  \end{itemize}
  \centering
  \begin{tabular}{@{}llll@{}} % 4 columns
    &   \multicolumn{3}{c}{现在时}  \\
    & 单  & 双 & 复 \\
    1st & \cellcolor{light-gray}asmi & svaḥ & smaḥ  \\
    2nd & \cellcolor{light-gray}a\textcolor{red}{s}i & sthaḥ & stha \\
    3rd & \cellcolor{light-gray}asti & staḥ & santi \\
  \end{tabular}   
\end{frame}

\begin{frame}[fragile]
  \frametitle{\verbroot{}as (II) 是}
  \small
  \begin{itemize}
    \item 未完成时:全都是ās\nobreakdash-,二单三单加\nobreakdash-ī\nobreakdash-
    \item 祈愿语气:全都是弱语干s\nobreakdash-
  \end{itemize}
  \centering
  \begin{NiceTabular}{lllllll}
    \CodeBefore
      \rectanglecolor{light-gray}{3-2}{5-4}
    \Body % 7 columns
    &   \multicolumn{3}{c}{未完成时} & \multicolumn{3}{c}{祈愿语气} \\
    & 单  & 双 & 复 & 单  & 双 & 复 \\
    1st & āsam & āsva & āsma & syām & syāva & syāma\\
    2nd & ās\textcolor{red}{ī}ḥ  & āstam & āsta & syāḥ  & syātam & syāta \\
    3rd & ās\textcolor{red}{ī}t & āstām & āsan & syāt & syātām & syuḥ \\
  \end{NiceTabular}   
\end{frame}

\begin{frame}{\verbroot{}i (II) 走}
  %\small
  \begin{itemize}
    \item 未完成时:全都是ai\nobreakdash-
  \end{itemize}
  \centering
  \begin{NiceTabular}{@{}llll@{}} % 4 columns
    \CodeBefore
      \rectanglecolor{light-gray}{3-2}{5-4}
    \Body 
    &   \multicolumn{3}{c}{未完成时}  \\
    & 单  & 双 & 复 \\
    1st & āyam & aiva & aima  \\
    2nd & aiḥ & aitam & aita \\
    3rd & ait & aitām & āyan \\
  \end{NiceTabular}   
\end{frame}

\begin{frame}[fragile]
  \frametitle{\verbroot{}han (II) 杀}
  \small
  \begin{itemize}
    \item 单数和第一人称都是强语干han\nobreakdash-
    \item 弱语干:辅音前是ha\nobreakdash-,元音前是ghn\nobreakdash-
  \end{itemize}
  \centering
  \footnotesize
  \begin{NiceTabular}{lllllll}
    \CodeBefore
      \rectanglecolor{light-gray}{3-2}{3-7}
      \rectanglecolor{light-gray}{4-2}{5-2}
      \rectanglecolor{light-gray}{4-5}{5-5}
    \Body % 7 columns
    &   \multicolumn{3}{c}{现在时} & \multicolumn{3}{c}{未完成时} \\
    & 单  & 双 & 复 & 单  & 双 & 复 \\
    1 & hanmi & hanvaḥ & hanmaḥ & ahanam & ahanva & ahanma\\
    2 & haṃsi  & hathaḥ & hatha & ahan  & ahatam & ahata \\
    3 & hanti & hataḥ & ghnanti & ahan & ahatām & aghnan \\
  \end{NiceTabular}   
\end{frame}

\begin{frame}%[fragile]
  \frametitle{\verbroot{}brū (II) 说}
  \small
  \begin{itemize}
    \item 强语干辅音语尾前加\nobreakdash-ī\nobreakdash-
  \end{itemize}
  \centering
  \begin{NiceTabular}{llll}
    \CodeBefore
      \rectanglecolor{light-gray}{3-2}{5-2}
      \rectanglecolor{light-gray}{8-2}{10-2}
    \Body % 7 columns
    &   \multicolumn{3}{c}{现在时}  \\
    & 单  & 双 & 复  \\
    1 & bravīmi & brūvah & brūmaḥ \\
    2 & bravīsi  & brūthaḥ & brutha \\
    3 & bravīti & brūtaḥ & bruvanti  \\
    \bigskip \\
    &    \multicolumn{3}{c}{未完成时} \\
    1  & abravam & abrūva & abrūma\\
    2 & abravīḥ  & abrūtam & abrūta \\
    3  & abravīt & abrūtām & abruvan \\
  \end{NiceTabular}   
\end{frame}

\begin{frame}%[fragile]
  \frametitle{\verbroot{}dā (III) 给,\verbroot{}dhā (III) 放 }
  \small
  \begin{itemize}
    \item 弱语干去掉\nobreakdash-ā\nobreakdash-,三单语尾是\nobreakdash-ati
  \end{itemize}
  \centering
  \begin{NiceTabular}{llll}
    \CodeBefore
      \rectanglecolor{light-gray}{3-2}{5-2}
      \rectanglecolor{light-gray}{7-2}{9-2}
    \Body % 7 columns
    &   \multicolumn{3}{c}{现在时}  \\
    \verbroot{}dā & 单  & 双 & 复  \\
    1 & dadāmi & dadvaḥ & dadmaḥ \\
    2 & dadāsi  & datthaḥ & dhattha \\
    3 & dadāti & dattaḥ & dadati \\
    \verbroot{}dhā & 单  & 双 & 复  \\
    1  & dadhāmi & dadhvaḥ & dadhmaḥ \\
    2 & dadhāsi  & dhatthaḥ & dhattha \\
    3  & dadhāti & dhattaḥ & dadhati \\
  \end{NiceTabular}   
\end{frame}

\begin{frame}%[fragile]
  \frametitle{\verbroot{}kṛ (VIII) 做}
  \small
  \begin{itemize}
    \item 强语干karo\nobreakdash-,弱语干kuru\nobreakdash-
    \item 弱语干的\nobreakdash-u\nobreakdash-在语尾的\nobreakdash-v\nobreakdash-和\nobreakdash-m\nobreakdash-前消失
  \end{itemize}
  \centering
  \begin{NiceTabular}{llll}
    \CodeBefore
      \rectanglecolor{light-gray}{3-2}{5-2}
      \rectanglecolor{light-gray}{8-2}{10-2}
    \Body % 7 columns
    &   \multicolumn{3}{c}{现在时}  \\
    & 单  & 双 & 复  \\
    1 & karomi & kurvaḥ & kurmaḥ \\
    2 & karoṣi & kuruthaḥ & kurutha \\
    3 & karoti & kurutaḥ & kurvanti  \\
    &    \multicolumn{3}{c}{未完成时} \\
    1 & akaravam & akurva & akurma\\
    2 & akaroḥ  & akurutam & akuruta \\
    3 & akarot & akurutām & akurvan \\
  \end{NiceTabular}   
\end{frame}

\begin{frame}{\verbroot{}stu (II) 赞颂}
  %\small
  \begin{itemize}
    \item 强语干三合
  \end{itemize}
  \centering
  \begin{tabular}{@{}llll@{}} % 4 columns
    &   \multicolumn{3}{c}{现在时}  \\
    & 单  & 双 & 复 \\
    1st & \cellcolor{light-gray}staumi & stuvaḥ & stumaḥ  \\
    2nd & \cellcolor{light-gray}stausi & stuthaḥ & stutha \\
    3rd & \cellcolor{light-gray}stauti & stutaḥ & stuvanti \\
  \end{tabular}   
\end{frame}

\begin{frame}%[fragile]
  \frametitle{第二类不分强弱语干动词}
  \small
  \begin{itemize}
    \item 以 \nobreakdash-ā 结尾的动词,如 \verbroot{}yā (II) 去
    \item \verbroot{}svap (II) 睡:辅音语尾前加 \nobreakdash-i\nobreakdash-
  \end{itemize}
  \centering
  \begin{NiceTabular}{llll}
    \CodeBefore
      %\rectanglecolor{light-gray}{3-2}{5-2}
      %\rectanglecolor{light-gray}{7-2}{9-2}
    \Body % 7 columns
    %&   \multicolumn{3}{c}{现在时}  \\
    \verbroot{}yā & 单  & 双 & 复  \\
    1 & yāmi & yāvaḥ & yāmaḥ \\
    2 & yāsi  & yāthaḥ & yātha \\
    3 & yāti & yātaḥ & yānti \\
    \verbroot{}svap &  & &  \\
    1  & svapimi & svapivaḥ & svapimaḥ \\
    2 & svapiṣi  & svapithaḥ & svapitha \\
    3  & svapiti & svapitaḥ & svapanti \\
  \end{NiceTabular}   
\end{frame}

\begin{frame}%[fragile]
  \frametitle{注意连声}
  \small
  \centering
  \begin{itemize}
    \item \verbroot{}vac (II) 说 ~~ \verbroot{}dviṣ (II) 恨
  \end{itemize}
  \begin{NiceTabular}{llll}
    \CodeBefore
      %\rectanglecolor{light-gray}{3-2}{5-2}
      %\rectanglecolor{light-gray}{7-2}{9-2}
    \Body % 7 columns 
    &   \multicolumn{3}{c}{现在时}  \\
    \verbroot{}vac & 单  & 双 & 复  \\
    1 & vacmi & vacvaḥ & vacmaḥ \\
    2 & vakṣi  & vakthaḥ & vaktha \\
    3 & vakti & vaktaḥ & vacanti \\
    \verbroot{}dviṣ &   &  &  \\
    1  & dveṣmi & dviṣvaḥ & dviṣmaḥ \\
    2 & dvekṣi  & dviṣṭhaḥ & dviṣṭhaṣ̄ \\
    3  & dveṣṭi & dviṣṭaḥ & dviṣanti \\
  \end{NiceTabular}   
\end{frame}

\section{本节作业}

\begin{frame}{\insertsection }
  \begin{itemize}
    \item
      第十九章练习3
    \item
      勘误:第3题括号后加\skt{nṛpāya}
    \bigskip
    %\item
    %  现在请做学习通\nobreakdash-章节\nobreakdash-课后问卷
  \end{itemize}
\end{frame}  

\end{document}	
