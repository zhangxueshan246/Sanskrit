%%% XeLaTeX-article %%%
%# -*- coding: utf-8 -*-
%!TEX encoding = UTF-8 Unicode
%!TEX TS-program = xelatex  
%---------------------虽然加了%还是要保留!

\documentclass[17pt]{beamer}
\mode<presentation>
{
\usetheme[width=40pt]{Hannover}
\usecolortheme[]{dove}
\usefonttheme[]{structurebold}
\setbeameroption{hide notes}
}

\usepackage{fontspec}
\usepackage{polyglossia}
\setmainfont{Arial} %设置主字体
\newfontfamily\sanskritfont[Script=Devanagari,Mapping=romantodevanagari,Scale=1.15]{Sanskrit 2003}             %输出天城体
%\newfontfamily\sanskritfont[Mapping=tex-text]{Times New Roman}              %输出转写
\doublehyphendemerits=-10000
\newcommand{\skt}[1]{{\sanskritfont{#1}}} %输出天城体
\newcommand{\skttrans}[1]{{\skt{#1}~#1}}  %输出天城体和转写
%----------------------------------------------------设置梵文输入方法 danda । ॥

\usepackage[UTF8,fontset=windows]{ctex}
\usepackage{amsmath}
%----------------------------------------------------设置中文环境

\usepackage{graphicx}
\usepackage{flafter} 
\graphicspath{{pic/}}
\usepackage{booktabs} 
\usepackage{nicematrix}
\newenvironment{indentlist}
  {\begin{list}{}{\setlength{\leftmargin}{2em}\setlength{\itemsep}{0.5em}}}
  {\end{list}}
%-----------------------------------------插图表格

\usepackage{hyperref} 
\usepackage[dvipsnames]{xcolor}
\usepackage{colortbl}
\definecolor{light-gray}{gray}{0.9}
%------------------------------颜色

\newcommand{\verbroot}[1]{\textcolor{red}{$\sqrt{}$#1}}
\newcommand{\sktroot}[1]{{\verbroot{\skt{#1}}}}
\newcommand{\skttransroot}[1]{{\sktroot{#1}~\textcolor{red}{#1}}}

\newcommand{\nounstem}[1]{\textcolor{red}{#1\nobreakdash-}}
\newcommand{\sktnounstem}[1]{{\textcolor{red}{\skt{#1\nobreakdash-}}}}
\newcommand{\skttransnounstem}[1]{{\sktnounstem{#1}~\nounstem{#1}}}

\newcommand{\verbstem}[1]{\textcolor{blue}{#1\nobreakdash-}}
\newcommand{\sktverbstem}[1]{{\textcolor{blue}{\skt{#1\nobreakdash-}}}}
\newcommand{\skttransverbstem}[1]{{\sktverbstem{#1}~\verbstem{#1}}}

\newcommand{\wordending}[1]{\textcolor{Orange}{\nobreakdash-#1}}
\newcommand{\sktending}[1]{{\textcolor{Orange}{\skt{-#1}}}}
\newcommand{\skttransending}[1]{{\sktending{#1}~\wordending{#1}}}

\newcommand{\fullpada}[1]{\textcolor{OliveGreen}{#1}}
\newcommand{\sktpada}[1]{{\textcolor{OliveGreen}{\skt{#1}}}}
\newcommand{\skttranspada}[1]{{\sktpada{#1}~\fullpada{#1}}}

\newcommand{\pratyaya}[1]{\textcolor{Plum}{#1}}
\newcommand{\sktpratyaya}[1]{{\textcolor{Plum}{\skt{#1}}}}
\newcommand{\skttranspratyaya}[1]{{\sktpratyaya{#1}~\pratyaya{#1}}}

\newcommand{\reconstruction}[1]{\textcolor{gray}{*#1}}
\newcommand{\fullsentence}[1]{\textcolor{MidnightBlue}{#1}}
\newcommand{\sktsentence}[1]{\textcolor{MidnightBlue}{\skt{#1}}}

\newcommand{\veryimportant}[1]{\textcolor{red}{#1}}
\newcommand{\important}[1]{\textcolor{blue}{#1}}
\newcommand{\notsoimportant}[1]{\textcolor{gray}{#1}}
%-------------------------------------------词根等标颜色

\title{{梵语入门}}
\subtitle{29. ṛ/n 变格与迂回将来时}
\author[张雪杉]{文学院~~张雪杉 \\ zhangxueshan@sdnu.edu.cn}
\date{}
%\institute{}

\begin{document}	

\begin{frame}
  \titlepage
\end{frame}

\begin{frame}
  \frametitle{本节内容}
  \small
  \tableofcontents
\end{frame}

\section{上节作业}

\begin{frame}{第28章练习4}
  \small
  \begin{verse}
    \skt{1) śūrā arīndadṛśuḥ । nagaraṃ pālayitumicchantaḥ pradudruvuḥ ।}\\
    \skt{2) bahuvasornarasya gṛhaṃ nanāśa । vasu taṃ na duḥkhādrarakṣeti bubudhima ।}\\
    \skt{3) īśvaraṃ dṛṣṭvā namaścakrire ।}\\
    \skt{4) uṣasi kumārāḥ priye gṛhanagare tatyajuḥ ।}\\
    \skt{5) yataḥ kāmaḥ kumārasya hṛdaye vavardha tataḥ kanyāyai ratnaṃ dadau ।}\\
    \skt{6) bālau gurorvacanāni śuśruvatuḥ । sa dhīmāneveti tuṣṭuvatuḥ ।}
  \end{verse}
\end{frame}

\begin{frame}{第28章练习4}
  \small
  \begin{verse}
    \skt{7) kumārā āpadi hataṃ mitraṃ śuśucire । guṇavānmahābalaścāsīditi sasmaruḥ ।}\\
    \skt{8) uditasya sūryasya prabhayā prasannaḥ siṃho 'cintaḥ suṣvāpa ।}\\
    \skt{9) yato nṛpaḥ paurānna rarakṣa tataḥ śatravaḥ \mbox{puramabhidudruvuḥ sarvāṇi ratnāni cāpajahruḥ ।}}\\
    \skt{10) kāle tu paurāḥ puraṃ tyaktuṃ śekuḥ ।}\\
    \skt{11) api bibhyeti pṛṣṭāḥ । ciraṃ bibhīmādhunaiva tu na bibhīma iti pratyūcima ।}
  \end{verse}
\end{frame}

\begin{frame}{第28章练习4}
  \small
  \begin{verse}
    \skt{12) ciraṃ mahānvṛkṣaḥ kṣatriyagaṇasya chāyāṃ \mbox{dadau ।} tadā tu chinnaḥ ।}\\
    \skt{13) yadannaṃ suhṛdaḥ pecustadbubhujimahe ।}\\
    \skt{14) nṛpaḥ kumārāya rūpavatīṃ kanyāṃ saratnāṃ dadau । mahāsukhau kumāraḥ kanyā cāpajagmatuḥ ।}\\
    \skt{15) bahava eva kṣatriyā yuddhe mamruḥ । eke tu svāṃ nagarīṃ pratyājagmuḥ ।}\\
    \skt{16) ciraṃ vasumānāsa । tadā tu sarvaṃ tatyāja vane ca gatvā tatra cintayituṃ sasāda ।}
  \end{verse}
\end{frame}

\begin{frame}{第28章练习4}
  \small
  \begin{verse}
    \skt{17) smayatā sakhīgaṇena parivṛtā nāryapi jahāsa ।}\\
    \skt{18) kṣatriyā iṣubhirhatā bhuvi petuḥ ।}\\
    \skt{19) pure jagmima । api kadā citpure \mbox{jagma ।} jagama sa ca jagāma te tu na jagmuḥ ।}\\
    \skt{20) naraḥ striyā darśanena tutoṣa । tasyāḥ prabhā devyā iveti mene ।}
  \end{verse}
\end{frame}

\begin{frame}{第28章练习4}
  \small
  \begin{verse}
    \skt{21) mṛtasya śatro rathamiṣūṃśca sadhanuṣo yuddhakṣetrādlebhimahe । kṣatriyebhyaḥ svabandhubhyo dadima ।}\\
    \skt{22) vyāghro haṃsaṃ jagrāha । kṣaṇenāpadudrāva ।}\\
    \skt{23) yataḥ kopāduvacitha tato na śuśruma ।}\\
    \skt{24) kṣatriyo na kadā \mbox{canārīnhantumiyeṣa ।} adhunā tu dharmavittānabhibabhūva । tānsarvāñjaghāna ।}
  \end{verse}
\end{frame}

\section[ṛ的变格]{以ṛ结尾的名词变格}
\begin{frame}{\insertsection}
  \small
  \tableofcontents[currentsection]
\end{frame}

\begin{frame}{ṛ的变格~~意义层面}
  \small
  \begin{itemize}
    \item 亲属名词,表示亲属关系。\\
    \nounstem{pitṛ}, \nounstem{mātṛ}\\
    \item 行动名词,\\
    词根二合+ \pratyaya{\nobreakdash-tṛ\nobreakdash-},表示动作发出者。\\
  \end{itemize}
  \centering
  \begin{tabular}{@{}ll@{}} % 6 columns
    \textbf{词根} & \textbf{行动名词} \\    
    \verbroot{nī} & \nounstem{netṛ} \\
    \verbroot{labh} & \nounstem{labdhṛ} \\    
    \verbroot{rakṣ} & \nounstem{rakṣitṛ} \\
  \end{tabular}
\end{frame}

\begin{frame}{ṛ的变格~~形态层面}
  \small
  \begin{itemize}
    \item 要区分强弱语干,词尾大多正常。\\
    \item 弱语干都是ṛ,单数依格都是ar。\\
    \item 强语干分两类:
    \begin{itemize}
      \item 父母兄女人用ar,\\
      \nounstem{pitṛ}, \nounstem{mātṛ}, \nounstem{bhrātṛ}, \nounstem{duhitṛ}, \nounstem{nṛ}\\
      \item 其他都用ār。
    \end{itemize}
  \end{itemize}
\end{frame}

\begin{frame}{以ṛ结尾的名词变格}
  \small
  \nounstem{pitṛ} m. 父亲,~~~~\nounstem{netṛ} m. 向导
  \bigskip

  \resizebox{\textwidth}{!}{
    \centering
    \begin{NiceTabular}{|c|c|c|c|c|c|c|}[hvlines, rules/width=0.3pt, rules/color=gray]
       & \Block{1-2}{单数} & & \Block{1-2}{双数} & & \Block{1-2}{复数}  \\
      主 & \cellcolor{light-gray}\fullpada{pitā} & \cellcolor{light-gray}\fullpada{netā} & \cellcolor{light-gray}\Block{3-1}{\fullpada{pitarau}} & \cellcolor{light-gray}\Block{3-1}{\fullpada{netārau}} & \cellcolor{light-gray}\Block{2-1}{\fullpada{pitaraḥ}} & \cellcolor{light-gray}\Block{2-1}{\fullpada{netaraḥ}}  \\
      呼 & \cellcolor{light-gray}\fullpada{pitaḥ} & \cellcolor{light-gray}\fullpada{netaḥ} & \cellcolor{light-gray}  & \cellcolor{light-gray} & \cellcolor{light-gray} & \cellcolor{light-gray} \\
      业 & \cellcolor{light-gray}\fullpada{pitaram} & \cellcolor{light-gray}\fullpada{netāram} & \cellcolor{light-gray} & \cellcolor{light-gray} & \fullpada{pitṝn} & \fullpada{netṝn} \\
      具 & \fullpada{pitrā} & \fullpada{netrā} & \Block{3-1}{\fullpada{pitṛbhyām}} & \Block{3-1}{\fullpada{netṛbhyām}} & \fullpada{pitṛbhiḥ} & \fullpada{netṛbhiḥ} \\
      为 & \fullpada{pitre} & \fullpada{netre} & & & \Block{2-1}{\fullpada{pitṛbhyaḥ}}  & \Block{2-1}{\fullpada{netṛbhyaḥ}} \\
      从 & \Block{2-1}{\fullpada{pituḥ}} & \Block{2-1}{\fullpada{netuḥ}} & &  & & \\
      属 & & & \Block{2-1}{\fullpada{pitroḥ}} & \Block{2-1}{\fullpada{netroḥ}} & \fullpada{pitṝṇām} & \fullpada{netṝṇām} \\
      依 & \cellcolor{light-gray}\fullpada{pitari} & \cellcolor{light-gray}\fullpada{netari} & & & \fullpada{pitṛṣu} & \fullpada{netṛṣu}\\
    \end{NiceTabular}
  }
  \bigskip

  注:阴性复数业格结尾是 \wordending{-ṝḥ}。\\
  ~~~~\nounstem{mātṛ} m. 母亲,复数业格:\fullpada{mātṝḥ}
\end{frame}


\section[n的变格]{以n结尾的名词变格}
\begin{frame}{\insertsection}
  \small
  \tableofcontents[currentsection]
\end{frame}

\begin{frame}{以n结尾的名词}
  \small
  \begin{itemize}
    \item 只有阳性和中性,使用标准辅音格尾。\\
    \item an结尾阳性词区分强弱语干\\
    \begin{itemize}
      \item 强语干三合用ān,\\
      \item 弱语干零级用n,\\
      \item -an前是复合辅音的词元音语尾前\\弱语干二合用an。
    \end{itemize}      
    \item in结尾只看语尾开头音
    \begin{itemize}
      \item 元音语尾前用in,\\
      \item 辅音语尾前用i。\\
    \end{itemize}
  \end{itemize}
\end{frame}

\subsection{an词干}
\begin{frame}{以an结尾的名词变格}
  \small
  \nounstem{rājan} m. 国王,~~~~\nounstem{ātman} m. 自我/灵魂
  \bigskip

  \resizebox{\textwidth}{!}{
    \centering
    \begin{NiceTabular}{|c|c|c|c|c|c|c|}[hvlines, rules/width=0.3pt, rules/color=gray]
       & \Block{1-2}{单数} & & \Block{1-2}{双数} & & \Block{1-2}{复数}  \\
      主 & \cellcolor{light-gray}\fullpada{rājā} & \cellcolor{light-gray}\fullpada{ātmā} & \cellcolor{light-gray}\Block{3-1}{\fullpada{rājānau}} & \cellcolor{light-gray}\Block{3-1}{\fullpada{ātmānau}} & \cellcolor{light-gray}\Block{2-1}{\fullpada{rājānaḥ}} & \cellcolor{light-gray}\Block{2-1}{\fullpada{ātmānaḥ}}  \\
      呼 & \cellcolor{light-gray}\fullpada{rājan} & \cellcolor{light-gray}\fullpada{ātman} & \cellcolor{light-gray}  & \cellcolor{light-gray} & \cellcolor{light-gray} & \cellcolor{light-gray} \\
      业 & \cellcolor{light-gray}\fullpada{rājānam} & \cellcolor{light-gray}\fullpada{ātmānam} & \cellcolor{light-gray} & \cellcolor{light-gray} & \fullpada{rājñaḥ} & \fullpada{ātmanaḥ} \\
      具 & \fullpada{rājnā} & \fullpada{ātmanā} & \Block{3-1}{\fullpada{rājabhyām}} & \Block{3-1}{\fullpada{ātmabhyām}} & \fullpada{rājabhiḥ} & \fullpada{ātmabhiḥ} \\
      为 & \fullpada{rājñe} & \fullpada{ātmane} & & & \Block{2-1}{\fullpada{rājabhyaḥ}}  & \Block{2-1}{\fullpada{ātmabhyaḥ}} \\
      从 & \Block{2-1}{\fullpada{rājñaḥ}} & \Block{2-1}{\fullpada{ātmanaḥ}} & &  & & \\
      属 & & & \Block{3-1}{\fullpada{rājñoḥ}} & \Block{3-1}{\fullpada{ātmanoḥ}} & \fullpada{rājñām} & \fullpada{ātmanām} \\
      \Block{2-1}{依} & \fullpada{rājñi} & \Block{2-1}{\fullpada{ātmani}} & & & \Block{2-1}{\fullpada{rājasu}} & \Block{2-1}{\fullpada{ātmaṣu}}\\
       & \cellcolor{light-gray}\fullpada{rājani} &  & & &  & \\
    \end{NiceTabular}
  }
  \bigskip

  \nounstem{nāman} n. 名字
  \bigskip

  { \footnotesize
  \centering 
    \begin{NiceTabular}{|c|c|c|c|c|c|c|}[hvlines, rules/width=0.3pt, rules/color=gray]
       & 单数 & 双数 & 复数  \\
      主呼业 & \fullpada{nāma} & \fullpada{nāmanī/nāmnī} & \fullpada{nāmāni} \\
    \end{NiceTabular}
  }
\end{frame}

\subsection{in词干}
\begin{frame}{以in结尾的名词变格}
  \small
  \nounstem{balin} adj. 强壮,有力的
  \bigskip

  \centering
  {\footnotesize
    \begin{NiceTabular}{|c|c|c|c|c|c|c|}[hvlines, rules/width=0.3pt, rules/color=gray]
      阳性 & 单数 & 双数 & 复数  \\
      主 & \fullpada{balī}  & \Block{3-1}{\fullpada{balinau}} & \Block{3-1}{\fullpada{balinaḥ}}   \\
      呼 & \fullpada{balin} &  &  \\
      业 & \fullpada{balinam} &  &  \\
      具 & \fullpada{balinā} & \Block{3-1}{\fullpada{balibhyām}} & \fullpada{balibhiḥ} \\
      为 & \fullpada{baline} &  & \Block{2-1}{\fullpada{balibhyaḥ}} \\
      从 & \Block{2-1}{\fullpada{balinaḥ}} &  &  \\
      属 &  & \Block{2-1}{\fullpada{balinoḥ}} & \fullpada{balinām} \\
      依 & \fullpada{balini} &  & \fullpada{baliṣu} \\
      中性 & 单数 & 双数 & 复数  \\
      主业 & \fullpada{bali}  & \Block{2-1}{\fullpada{balinī}} & \Block{2-1}{\fullpada{balīni}}   \\
      呼 & \fullpada{bali(n)} &  &  \\
    \end{NiceTabular}
  }
\end{frame}

\begin{frame}{in相关构词法}
  \small
  \begin{itemize}
    \item \pratyaya{\nobreakdash-in\nobreakdash-} 和 \pratyaya{\nobreakdash-vin\nobreakdash-} 加在名词后表所有。\\
    以a结尾的名词在加 \pratyaya{\nobreakdash-in\nobreakdash-} 时去掉a。\\
  \end{itemize} 
  \centering 
  \begin{tabular}{@{}ll@{}} % 6 columns
    \textbf{名词} & \textbf{加in/vin} \\    
    \nounstem{bala} 力量 & \nounstem{balin} 有力的 \\
    \nounstem{veda} 知识 & \nounstem{vedin} 知者 \\    
    \nounstem{hasta} 手 & \nounstem{hastin} 大象 \\
    \nounstem{tejas} 光芒 & \nounstem{tejasvin} 光辉的 \\
    \nounstem{tapas} 苦行 & \nounstem{tapasvin} 苦行者 \\
  \end{tabular}
\end{frame}

\begin{frame}{in相关构词法}
  \small
  \begin{itemize}
    \item \pratyaya{\nobreakdash-in\nobreakdash-} 加在词根后表doing。\\
    词根要二合或三合。\\
  \end{itemize} 
  \centering 
  \begin{tabular}{@{}ll@{}} % 6 columns
    \textbf{词根} & \textbf{加in} \\    
    \verbroot{kṛ} 做 & \nounstem{karin\nobreakdash-/kārin} 做,做者 \\
    \verbroot{ji} 战胜 & \nounstem{jayin} 胜利 \\    
  \end{tabular}
  \begin{itemize}
    \item in结尾名词的阴性加\pratyaya{\nobreakdash-ī\nobreakdash-}。\\
    \nounstem{balin} mn. 强壮 $\to$ \nounstem{balinī} f. 强壮\\
  \end{itemize} 
\end{frame}

\section{迂回将来时}
\begin{frame}{\insertsection}
  \small
  \tableofcontents[currentsection]
\end{frame}

\begin{frame}{迂回将来时}
  \small
  \begin{itemize}
    \item 迂回的意思是用两个词而非一个词表达。\\
    \notsoimportant{迂回完成时在第34章。}
    \item 迂回将来时\\=行动名词(\pratyaya{\nobreakdash-tṛ\nobreakdash-})主格 + \verbroot{as} 现在时。\\
    \mbox{一二人称用行动名词单数主格加 \verbroot{as} 变位,}\\
    第三人称只出现行动名词主格。
    \item 常与时间副词连用。\\
    \fullsentence{śvo 'smi hantā Jayadratham} \\~~~明天我将要杀掉胜车王。
  \end{itemize}
\end{frame}

\begin{frame}{迂回将来时}
  %\small
  \verbroot{nī} 带领 $\to$ \nounstem{netṛ} 向导
  \bigskip

  \centering
  \begin{tabular}{@{}llll@{}} % 6 columns
    & 单 & 双 & 复 \\    
    1st & \fullpada{netāsmi} & \fullpada{netāsvaḥ}  & \fullpada{netāsmaḥ} \\
    2nd & \fullpada{netāsi} & \fullpada{netāsthaḥ} & \fullpada{netāstha} \\    
    3rd & \fullpada{netā} & \fullpada{netārau} & \fullpada{netāraḥ} \\
  \end{tabular}
\end{frame}

\section{本节作业}

\begin{frame}{\insertsection }
  \begin{itemize}
    \item
      第29章练习4
  \end{itemize}
\end{frame}  

\end{document}