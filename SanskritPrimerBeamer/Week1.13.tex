%%% XeLaTeX-article %%%
%# -*- coding: utf-8 -*-
%!TEX encoding = UTF-8 Unicode
%!TEX TS-program = xelatex  
%---------------------虽然加了%还是要保留!

\documentclass[17pt]{beamer}
\mode<presentation>
{
\usetheme[width=40pt]{Hannover}
\usecolortheme[]{dove}
\usefonttheme[]{structurebold}
\setbeameroption{hide notes}
}

\usepackage{fontspec}
\usepackage{polyglossia}
\setmainfont{Arial} %设置主字体
\newfontfamily\sanskritfont[Script=Devanagari,Mapping=romantodevanagari,Scale=1.15]{Sanskrit 2003}             %输出天城体
%\newfontfamily\sanskritfont[Mapping=tex-text]{Times New Roman}              %输出转写
\doublehyphendemerits=-10000
\newcommand{\skt}[1]{{\sanskritfont{#1}}} %输出天城体
\newcommand{\skttrans}[1]{{\skt{#1}~#1}}  %输出天城体和转写
%----------------------------------------------------设置梵文输入方法 danda । ॥

\usepackage[UTF8,fontset=windows]{ctex}
\usepackage{amsmath}
%----------------------------------------------------设置中文环境

\usepackage{graphicx}
\usepackage{flafter} 
\graphicspath{{pic/}}
\usepackage{booktabs} 
\usepackage{nicematrix}
\newenvironment{indentlist}
  {\begin{list}{}{\setlength{\leftmargin}{2em}\setlength{\itemsep}{0.5em}}}
  {\end{list}}
%-----------------------------------------插图表格

\usepackage{hyperref} 
\usepackage[dvipsnames]{xcolor}
\usepackage{colortbl}
\definecolor{light-gray}{gray}{0.9}
%------------------------------颜色

\newcommand{\verbroot}[1]{\textcolor{red}{$\sqrt{}$#1}}
\newcommand{\sktroot}[1]{{\verbroot{\skt{#1}}}}
\newcommand{\skttransroot}[1]{{\sktroot{#1}~\textcolor{red}{#1}}}

\newcommand{\nounstem}[1]{\textcolor{red}{#1\nobreakdash-}}
\newcommand{\sktnounstem}[1]{{\textcolor{red}{\skt{#1\nobreakdash-}}}}
\newcommand{\skttransnounstem}[1]{{\sktnounstem{#1}~\nounstem{#1}}}

\newcommand{\verbstem}[1]{\textcolor{blue}{#1\nobreakdash-}}
\newcommand{\sktverbstem}[1]{{\textcolor{blue}{\skt{#1\nobreakdash-}}}}
\newcommand{\skttransverbstem}[1]{{\sktverbstem{#1}~\verbstem{#1}}}

\newcommand{\wordending}[1]{\textcolor{Orange}{\nobreakdash-#1}}
\newcommand{\sktending}[1]{{\textcolor{Orange}{\skt{-#1}}}}
\newcommand{\skttransending}[1]{{\sktending{#1}~\wordending{#1}}}

\newcommand{\fullpada}[1]{\textcolor{OliveGreen}{#1}}
\newcommand{\sktpada}[1]{{\textcolor{OliveGreen}{\skt{#1}}}}
\newcommand{\skttranspada}[1]{{\sktpada{#1}~\fullpada{#1}}}

\newcommand{\pratyaya}[1]{\textcolor{Plum}{#1}}
\newcommand{\sktpratyaya}[1]{{\textcolor{Plum}{\skt{#1}}}}
\newcommand{\skttranspratyaya}[1]{{\sktpratyaya{#1}~\pratyaya{#1}}}

\newcommand{\reconstruction}[1]{\textcolor{gray}{*#1}}
\newcommand{\fullsentence}[1]{\textcolor{MidnightBlue}{#1}}
\newcommand{\sktsentence}[1]{\textcolor{MidnightBlue}{\skt{#1}}}

\newcommand{\veryimportant}[1]{\textcolor{red}{#1}}
\newcommand{\important}[1]{\textcolor{blue}{#1}}
\newcommand{\notsoimportant}[1]{\textcolor{gray}{#1}}
%-------------------------------------------词根等标颜色

\title{{梵语入门}}
\subtitle{14. 复合词}
\author[张雪杉]{文学院~~张雪杉 \\ zhangxueshan@sdnu.edu.cn}
\date{}
%\institute{}

\begin{document}	

\begin{frame}
  \titlepage
\end{frame}

\begin{frame}
  \frametitle{本节内容}
  \small
  \tableofcontents
\end{frame}

\section{上节作业}

\begin{frame}{止音连声}
  \begin{verse}
    	\skt{1) naraḥ + tatra}\\
    	\skt{2) naraḥ + uktavān}\\
    	\skt{3) bālāḥ + gacchanti}\\
    	\skt{4) kaviḥ + vadati}\\
    	\skt{5) guroḥ + āśramaḥ}\\
    	\skt{6) naraḥ + ayam}\\
    	\skt{7) punaḥ + rāmaḥ}
  \end{verse}
\end{frame}

\subsection{复习}
\begin{frame}{ī ū 变格}  
  \centering
  \includegraphics[width=\textwidth]{longiudeclension.png} %
\end{frame}

\section{复合词}
\begin{frame}{\insertsection}
  \small
  \tableofcontents[currentsection]
\end{frame}

\subsection{基本概念}
\begin{frame}{\insertsection \insertsubsection }
  \small 
  \begin{itemize}
    \item 名词性复合词\\是由两个或多个词干组合而成的名词。
    \item 
      通常只有最后一个词带格尾,\\前面的词都使用词干形式。\\
      \bigskip


      短语:\fullsentence{rājñaḥ putraḥ} (国王的儿子)\\
      复合词:\fullpada{rāja\nobreakdash-putraḥ} (王子)
  \end{itemize}  
\end{frame}

\subsection{传统分类}
\begin{frame}{\insertsection \insertsubsection }
  \small
  传统上,梵语复合词分为三大类:
  \begin{itemize}
    \item 并列复合词 (相违释 / Dvandva)
    \item 限定复合词 (依主释 / Tatpuruṣa)
    \begin{itemize}
      \item 同位复合词\\ (持业释 / Karmadhāraya)
    \end{itemize}
    \item 定语复合词 (多财释 / Bahuvrīhi)
  \end{itemize}
\end{frame}

\subsection{~~~相违释}
\begin{frame}{并列复合词 (相违释)}
  \small
  \begin{itemize}
    \item 所有词是并列关系,可以用“和”拆分。
    \item 指两个东西是双数,指多个是复数。\\
    \bigskip

    \fullpada{rāma-lakṣmaṇau} 
      $=$ \fullsentence{rāmaḥ lakṣmaṇaḥ ca}\\
    \fullpada{sūrya-candrau} $=$ \fullsentence{sūryaḥ candraḥ ca}\\
    \fullpada{putra-pautrāḥ} \\
    \fullpada{deva\nobreakdash-gandharva\nobreakdash-mānuṣa\nobreakdash-uraga\nobreakdash-rākṣasān} \\
  \end{itemize}
\end{frame}

\subsection{~~~依主释}
\begin{frame}{限定复合词 (依主释)}
  \small
  \begin{itemize}
    \item 后词表达的事物,由前词加以限定。
    \item 拆开前词与后词不同格的称狭义依主释。\\
    最常见的是属格,其他格也都可能。\\
  \end{itemize}
   \centering
  \begin{tabular}{@{}ll@{}} % 6 columns
    \textbf{前词} & \textbf{例} \\    
    属格 & \fullpada{rāja\nobreakdash-putraḥ} $=$ \fullsentence{rājñaḥ putraḥ} \\
    业格 &\fullpada{grāma\nobreakdash-gataḥ} $=$ \fullsentence{grāmaṃ gataḥ} \\
    具格 & \fullpada{deva\nobreakdash-dattaḥ} $=$ \fullsentence{devena dattaḥ} \\
    为格 & \fullpada{karṇa\nobreakdash-sukhaḥ} $=$ \fullsentence{karṇāya sukhaḥ} \\
    从格 & \fullpada{svarga\nobreakdash-patitaḥ} $=$ \fullsentence{svargāt patitaḥ} \\
    依格 & \fullpada{yudhi\nobreakdash-sthiraḥ} $=$ \fullsentence{yudhi sthiraḥ} \\
  \end{tabular}
\end{frame}

\subsection{~~~持业释}
\begin{frame}{同位复合词 (持业释)}
  \small
  \begin{itemize}
    \item 拆开前后词同格的依主释称为持业释。
    \item 前词以状语、同位语或者喻体的形式限定后词。
    \bigskip

    \fullpada{nīlotpalam} $=$ \fullsentence{nīlam utpalam} \\
    \fullpada{megha\nobreakdash-dūtaḥ} $=$ \fullsentence{megha eva dūtaḥ} \\
    \fullpada{megha\nobreakdash-śyāmaḥ} $=$ \fullsentence{megha iva śyāmaḥ} \\
  \end{itemize}
\end{frame}

\subsection{~~~多财释}
\begin{frame}{定语复合词 (多财释)}
  \small
  \begin{itemize}
    \item 后词是名词,但整个词是形容词。
    \item 后词不是中心词,全词修饰另一词。
  \end{itemize}
   \centering
  \begin{tabular}{@{}ll@{}} % 6 columns
    \textbf{前词} & \textbf{例} \\
    形容词 & \nounstem{mahā\nobreakdash-mukha}  \\
    分词 & \nounstem{nata\nobreakdash-mukha}  \\
    名词 & \nounstem{aśva\nobreakdash-mukha}  \\
    介词 & \nounstem{prati\nobreakdash-mukha}  \\
    数词 & \nounstem{catur\nobreakdash-mukha}  \\
    其他 & \nounstem{a\nobreakdash-mukha}  \\
  \end{tabular}
\end{frame}

\begin{frame}{区分依主释与多财释}
  \small
  \begin{itemize}
    \item 看词性
    \begin{itemize}
      \item 依主释的词性跟随中心词。\\
        \fullpada{pīta-ambaram} (n.) 黄色的衣服。
      
      \item 多财释的词性随被修饰词变化。\\
        \fullsentence{pītāmbaraḥ viṣṇuḥ} (m.) 黄衣毗湿奴。\\
    \end{itemize}
    \item 看上下文
  \end{itemize}
\end{frame}

\begin{frame}{总结}
  \small
  \centering
  \begin{tabular}{lll}
    \textbf{术语} & \textbf{汉译} & \textbf{例子} \\
    Dvandva & 相违释 & \fullpada{sūrya-candrau} \\
    Tatpuruṣa & 依主释 & \fullpada{rāja-putraḥ} \\
    Karmadhāraya & 持业释 & \fullpada{nīlotpalam} \\
    Bahuvrīhi & 多财释 & \fullpada{mahā-mukhaḥ} \\

  \end{tabular}
\end{frame}

\begin{frame}{更长的复合词}
  \small
  \begin{itemize}
    \item 复合词可以嵌套,形成长复合词。\\分析时需层层拆解。
    \bigskip

    \fullpada{janma\nobreakdash-bandha\nobreakdash-vinirmuktaḥ}\\
    生死~~~~束缚~~~~解脱
    \fullpada{śrama\nobreakdash-vitata\nobreakdash-mukha\nobreakdash-bhraṃśī}\\
    劳累~~~张开~~~口~~~~~掉下\\
    \bigskip

    \fullpada{udgamana\nobreakdash-upaniveśana\nobreakdash-śayana-}\\
    站起~~~~~~~~坐下~~~~~~~~~~躺卧\\
    \fullpada{parāvṛtti\nobreakdash-valana\nobreakdash-calaneṣu}\\
    返回~~~~~~转圈~~~~行走
  \end{itemize}
\end{frame}

\section{课堂练习}

\begin{frame}{分析复合词类型}
  \begin{verse}
    \skt{1) lokapālāḥ} \\
    \skt{2) sukhaduḥkhayoḥ} \\
    \skt{3) gatamatiḥ} \\
    \skt{4) kṛtāñjaliḥ} \\
    \skt{5) putrasnehena}\\
    \skt{6) krodhabhayāt} \\
    \skt{7) krodhabhaye} \\
    \skt{8) naranāryau} \\
  \end{verse}
\end{frame}

\begin{frame}{复合动词找词根}
  %\small
  \raggedright
  例: \skt{samāgacchati} = \pratyaya{sam} + \pratyaya{ā} + \verbroot{gam}
  \bigskip  
  \begin{verse}
    \skt{1) paryanunayāmaḥ} \\
    \skt{2) saṃnibhṛtaḥ} \\
    \skt{3) anūttiṣṭhati} \\
    \skt{4) pratyupadravasi} \\
    \skt{5) samutkṣipataḥ} \\
    \skt{6) prodgacchanti । vyapāgacchāma} \\
    \skt{7) apātiṣṭhāvaḥ । apātiṣṭhāva}
  \end{verse}
\end{frame}

\section{本节作业}

\begin{frame}{\insertsection }
  \begin{itemize}
    \item
      第14章练习4
    \item
      阅读教材第14课相关内容
    \bigskip
    \item
      现在请做学习通\nobreakdash-章节\nobreakdash-课后问卷
  \end{itemize}
\end{frame}

\end{document}