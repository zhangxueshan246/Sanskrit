%%% XeLaTeX-article %%%
%# -*- coding: utf-8 -*-
%!TEX encoding = UTF-8 Unicode
%!TEX TS-program = xelatex  
%---------------------虽然加了%还是要保留!

\documentclass[17pt]{beamer}
\mode<presentation>
{
\usetheme[width=40pt]{Hannover}
\usecolortheme[]{dove}
\usefonttheme[]{structurebold}
\setbeameroption{hide notes}
}

\usepackage{fontspec}
\setmainfont{Arial} %设置主字体
\newfontfamily\sanskritfont[Script=Devanagari,Mapping=romantodevanagari,Scale=1.15]{Sanskrit 2003}             %输出天城体
%\newfontfamily\sanskritfont[Mapping=tex-text]{Times New Roman}              %输出转写
\doublehyphendemerits=-10000
\newcommand{\skt}[1]{{\sanskritfont{#1}}} %输出天城体
\newcommand{\skttrans}[1]{{\skt{#1}~#1}}  %输出天城体和转写
%----------------------------------------------------设置梵文输入方法 danda । ॥

\usepackage[UTF8,fontset=windows]{ctex}
\usepackage{amsmath}
%----------------------------------------------------设置中文环境

\usepackage{graphicx}
\usepackage{flafter} 
\graphicspath{{pic/}}
\usepackage{booktabs} 
\usepackage{nicematrix}
\usepackage{diagbox}
%-----------------------------------------插图表格

\usepackage{hyperref} 
\usepackage[dvipsnames]{xcolor}
\usepackage{colortbl}
\definecolor{light-gray}{gray}{0.85}
%------------------------------颜色

\newcommand{\verbroot}[1]{\textcolor{red}{$\sqrt{}$#1}}
\newcommand{\sktroot}[1]{{\verbroot{\skt{#1}}}}
\newcommand{\skttransroot}[1]{{\sktroot{#1}~\textcolor{red}{#1}}}

\newcommand{\nounstem}[1]{\textcolor{red}{#1\nobreakdash-}}
\newcommand{\sktnounstem}[1]{{\textcolor{red}{\skt{#1\nobreakdash-}}}}
\newcommand{\skttransnounstem}[1]{{\sktnounstem{#1}~\nounstem{#1}}}

\newcommand{\verbstem}[1]{\textcolor{blue}{#1\nobreakdash-}}
\newcommand{\sktverbstem}[1]{{\textcolor{blue}{\skt{#1\nobreakdash-}}}}
\newcommand{\skttransverbstem}[1]{{\sktverbstem{#1}~\verbstem{#1}}}

\newcommand{\wordending}[1]{\textcolor{Orange}{\nobreakdash-#1}}
\newcommand{\sktending}[1]{{\textcolor{Orange}{\skt{-#1}}}}
\newcommand{\skttransending}[1]{{\sktending{#1}~\wordending{#1}}}

\newcommand{\fullpada}[1]{\textcolor{OliveGreen}{#1}}
\newcommand{\sktpada}[1]{{\textcolor{OliveGreen}{\skt{#1}}}}
\newcommand{\skttranspada}[1]{{\sktpada{#1}~\fullpada{#1}}}

\newcommand{\pratyaya}[1]{\textcolor{Plum}{#1}}
\newcommand{\sktpratyaya}[1]{{\textcolor{Plum}{\skt{#1}}}}
\newcommand{\skttranspratyaya}[1]{{\sktpratyaya{#1}~\pratyaya{#1}}}

\newcommand{\reconstruction}[1]{\textcolor{gray}{*#1}}
\newcommand{\fullsentence}[1]{\textcolor{MidnightBlue}{#1}}

\newcommand{\veryimportant}[1]{\textcolor{red}{#1}}
\newcommand{\important}[1]{\textcolor{blue}{#1}}
\newcommand{\notsoimportant}[1]{\textcolor{gray}{#1}}
%-------------------------------------------词根等标颜色

\title{{梵语提高}}
\subtitle{24. s的变格和命令语气}
\author[张雪杉]{文学院~~张雪杉 \\ zhangxueshan@sdnu.edu.cn}
\date{}
%\institute{}

\begin{document}	

\begin{frame}
  \titlepage
\end{frame}

\begin{frame}
  \frametitle{本节内容}
  %\small
  \tableofcontents
\end{frame}

\section{上节作业}

\begin{frame}{第23章练习1}
  \raggedright
  \small
  \begin{verse}
    \skt{1) yasmātpurātsarve te kumārā āgacchaṃstasminvastumicchāmi ।}   \\
    \skt{2) yānkṣatriyānapadravamāṇānpaśyasi tānhantuṃ na śakṣyāmaḥ ।}   \\
    \skt{3) yāṃ bālāṃ vana āsīnāṃ naro 'paśyattasyā annamadadāt }\\
    \skt{4) yāvatsūryaṃ dyotamānamīkṣe tāvatsmaye ।}   \\
    \skt{5) kutra te kumārāḥ । atra nagaryāṃ na santīti pṛṣṭā yoddhumapāgacchanniti nārī duḥkhaṃ punarabravīt ।}   \\
    \skt{6) yo yo jāyate sa mariṣyati ।}   \\
  \end{verse}
\end{frame}

\begin{frame}{第23章练习1}
  \small
  \raggedright
  \begin{verse}
    \skt{7) yato yuddhātpratyāgataṃ putraṃ paśyati tato hṛṣyati ।}   \\
    \skt{8) kadā cana bālo nagarīṃ gacchati tadā tasmai bhoktumannaṃ dīyate ।}   \\
    \skt{9) yāni vacanānyaśṛṇavaṃ tāni staumi ।}\\
    \skt{10) kadā citkaṃ cana vedavidaṃ paśyasi tadā tasya vacanāni śrotumarhasi ।}  \\
    \skt{11) yatra yatra mitrāṇi tatra sukham ।}  \\
    \skt{12) katarastayorvṛkṣayoruttaro 'sti । ya uttaro 'sti taṃ rohitumicchāmi ।}   \\
  \end{verse}
\end{frame}

\begin{frame}{第23章 Sudarśana}
  \small
  \raggedright
  \begin{verse}
    \skt{sudarśanaṃ pravakṣyāmi dvīpaṃ te kurunandana ।}   \\
    \skt{parimaṇḍalo mahārāja dvīpo 'sau cakrasaṃsthitaḥ ॥ 13 ॥}   \\
    \skt{nadījalapraticchannaḥ parvataiścābhrasaṃnibhaiḥ ।}\\
    \mbox{\skt{puraiśca vividhākārai ramyairjanapadaistathā ॥ 14 ॥}}  \\
    \skt{vṛkṣaiḥ puṣpaphalopetaiḥ saṃpannadhanadhānyavān ।}  \\
    \mbox{\skt{lavaṇena samudreṇa samantātparivāritaḥ ॥ 15 ॥}}  \\
    \skt{yathā hi puruṣaḥ paśyedādarśe mukhamātmanaḥ ।}  \\
    \mbox{\skt{evaṃ sudarśanadvīpo dṛśyate candramaṇḍale ॥ 16 ॥}}  \\
  \end{verse}
\end{frame}


\begin{frame}{第23章 Sudarśana}  
  \centering
  \includegraphics[width=\textwidth]{sudarśana.png} %
\end{frame}


\section{s的变格}
\begin{frame}{\insertsection }
    \small
    \tableofcontents[currentsection]
\end{frame}

\begin{frame}{\insertsection }
  \begin{itemize}
    \item s 结尾的名词一般是中性
    \item 还有以中性名词为后词的多财释
    \item 加正常辅音格尾,主要是连声问题。
  \end{itemize}
\end{frame}

\begin{frame}{辅音格尾表复习}
  \small
  \centering
    \begin{NiceTabular}{|c|c|c|c|c|c|c|}[hvlines, rules/width=0.3pt, rules/color=gray]
       & \Block{1-2}{单数} & & \Block{1-2}{双数} & & \Block{1-2}{复数} &  \\
       & 阳阴 & 中 & 阳阴 & 中 & 阳阴 & 中  \\
      主 & \wordending{s} & \Block{3-1}{\wordending{}} & \Block{3-1}{\wordending{au}}  & \Block{3-1}{\wordending{ī}}  & \Block{3-1}{\wordending{aḥ}} &  \Block{3-1}{\wordending{ni}}   \\
      呼 & \wordending{} & & & & & \\
      业 & \wordending{am} &  & & & & \\
      具 & \Block{1-2}{\wordending{ā}} &  & \Block{3-2}{\wordending{bhyām}}  & & \Block{1-2}{\wordending{bhiḥ}} & \\
      为 & \Block{1-2}{\wordending{e}} &  & & & \Block{2-2}{\wordending{bhyaḥ}}  & \\
      从 & \Block{2-2}{\wordending{aḥ}} &  & & & & \\
      属 & &  & \Block{2-2}{\wordending{oḥ}} & &  \Block{1-2}{\wordending{ām}} & \\
      依 & \Block{1-2}{\wordending{i}} &  & & & \Block{1-2}{\wordending{su}}&  \\
    \end{NiceTabular}
\end{frame}

\subsection{as结尾名词}
\begin{frame}{\nounstem{-as} 结尾的中性名词}
  \small
  \nounstem{manas} n. 心,意
  \bigskip

  \centering
    \begin{NiceTabular}{|c|c|c|c|c|c|c|}[hvlines, rules/width=0.3pt, rules/color=gray]
       & 单数 & 双数 & 复数  \\
      主 & \Block{3-1}{\fullpada{manaḥ}}  & \Block{3-1}{\fullpada{manasī}} & \Block{3-1}{\fullpada{manāṃsi}}   \\
      呼 &  &  &  \\
      业 &  &  &  \\
      具 & \fullpada{manasā} & \Block{3-1}{\fullpada{manobhyām}} & \fullpada{manobhiḥ} \\
      为 & \fullpada{manase} &  & \Block{2-1}{\fullpada{manobhyaḥ}} \\
      从 & \Block{2-1}{\fullpada{manasaḥ}} &  &  \\
      属 &  & \Block{2-1}{\fullpada{manasoḥ}} & \fullpada{manasām} \\
      依 & \fullpada{manasi} &  & \fullpada{manaḥsu} \\
    \end{NiceTabular}
\end{frame}

\begin{frame}{\nounstem{-as} 结尾的多财释}
  \small
  \nounstem{sumanas} adj. 好心的
  \begin{itemize}
    \item 中性变化同 \nounstem{manas}
    \item 阳性阴性具格以下也一样
    \item 阳性阴性主呼业格如下表
  \end{itemize}
  
  \bigskip

  \centering
    \begin{NiceTabular}{|c|c|c|c|c|c|c|}[hvlines, rules/width=0.3pt, rules/color=gray]
       & 单数 & 双数 & 复数  \\
      主 & \fullpada{sumanāḥ}  & \Block{3-1}{\fullpada{sumanasau}} & \Block{3-1}{\fullpada{sumanasaḥ}}  \\
      呼 & \fullpada{sumanaḥ} &  &  \\
      业 & \fullpada{sumanasam} &  &  \\
    \end{NiceTabular}
\end{frame}

\subsection{is/us结尾名词}
\begin{frame}{\nounstem{-us} 结尾的名词}
  \small
  \nounstem{cakṣus} n. 眼睛 ~~ \nounstem{acakṣus} adj. 失明的
  \bigskip

  \centering
    \begin{NiceTabular}{|c|c|c|c|c|c|c|}[hvlines, rules/width=0.3pt, rules/color=gray]
      阳阴 & 单数 & 双数 & 复数  \\
      主呼 & \fullpada{acakṣuḥ}  & \Block{2-1}{\fullpada{acakṣuṣau}} & \Block{2-1}{\fullpada{acakṣuṣaḥ}}  \\
      业 & \fullpada{acakṣuṣam} &  &  \\
      中性 & \Block{2-1}{\fullpada{cakṣuḥ}}  & \Block{2-1}{\fullpada{cakṣuṣī}} & \Block{2-1}{\fullpada{cakṣūṃṣi}}   \\
      主呼业 &  &  &  \\
      具 & \fullpada{cakṣuṣā} & \Block{3-1}{\fullpada{cakṣurbhyām}} & \fullpada{cakṣurbhiḥ} \\
      为 & \fullpada{cakṣuṣe} &  & \Block{2-1}{\fullpada{cakṣurbhyaḥ}} \\
      从 & \Block{2-1}{\fullpada{cakṣuṣaḥ}} &  &  \\
      属 &  & \Block{2-1}{\fullpada{cakṣuṣoḥ}} & \fullpada{cakṣuṣām} \\
      依 & \fullpada{cakṣuṣi} &  & \fullpada{cakṣuḥṣu} \\
    \end{NiceTabular}
\end{frame}

\section{命令语气}
\subsection{形式}
\begin{frame}{命令语气语尾}  
  \centering
  \includegraphics[width=\textwidth]{imperativeendings.png} %
\end{frame}

\begin{frame}{\verbstem{a} 语干 }
  \small
  \verbroot{bhṛ (1)} ~持
  \bigskip

  \centering 
  \begin{tabular}{@{}llll@{}} % 4 columns
    P. & 单数  & 双数  & 复数 \\
    1st & \fullpada{bharāṇi} & \cellcolor{yellow!30}\fullpada{bharāva}  & \cellcolor{yellow!30}\fullpada{bharāma}  \\
    2nd & \fullpada{bhara} & \cellcolor{yellow!30}\fullpada{bharatam} & \cellcolor{yellow!30}\fullpada{bharata}  \\
    3rd & \fullpada{bharatu} & \cellcolor{yellow!30}\fullpada{bharatām} & \fullpada{bharantu}  \\
  \end{tabular} 
  \bigskip
    
  \begin{tabular}{@{}llll@{}} % 4 columns
    Ā. & 单数  & 双数  & 复数 \\
    1st & \fullpada{bharai} & \fullpada{bharāvahai} & \fullpada{bharāmahai} \\
    2nd & \fullpada{bharasva} & \cellcolor{yellow!30}\fullpada{bharethām} & \cellcolor{yellow!30}\fullpada{bharadhvam} \\
    3rd & \fullpada{bharatām} & \cellcolor{yellow!30}\fullpada{bharetām} & \fullpada{bharantām} \\
  \end{tabular}   
\end{frame}

\begin{frame}{非 \verbstem{a} 语干 }
  \small
  \verbroot{bhuj (7)} ~享用
  \bigskip
  
  \centering 
  \begin{tabular}{@{}llll@{}} % 4 columns
    P. & 单数  & 双数  & 复数 \\
    1st & \cellcolor{light-gray}\fullpada{bhunajāṇi} & \cellcolor{light-gray}\fullpada{bhunajāva}  & \cellcolor{light-gray}\fullpada{bhunajāma}  \\
    2nd & \fullpada{bhuṅgdhi} & \fullpada{bhuṅktam} & \fullpada{bhuṅkta}  \\
    3rd & \cellcolor{light-gray}\fullpada{bhunaktu} & \fullpada{bhuṅktām} & \fullpada{bhuñjantu}  \\
  \end{tabular} 
  \bigskip
    
  \begin{tabular}{@{}llll@{}} % 4 columns
    Ā. & 单数  & 双数  & 复数 \\
    1st & \cellcolor{light-gray}\fullpada{bhunajai} & \cellcolor{light-gray}\fullpada{bhunajāvahai} & \cellcolor{light-gray}\fullpada{bhunajāmahai} \\
    2nd & \fullpada{bhuṅkṣva} & \fullpada{bhuñjāthām} & \fullpada{bhuṅgdhvam} \\
    3rd & \fullpada{bhuṅktām} & \fullpada{bhuñjātām} & \fullpada{bhuñjatām} \\
  \end{tabular}   
\end{frame}

\begin{frame}{第二人称单数}
  \begin{itemize}
    \item \verbstem{a} 语干主动:等于现在时语干\\
    \verbroot{bhṛ}  \fullpada{bhara}  ~~ \verbroot{sthā}  \fullpada{uttiṣṭḥa}
    \item 非 \verbstem{a} 语干主动:语尾是 \wordending{dhi} 或 \wordending{hi}\\
    \verbroot{bhuj}  \fullpada{bhuṅgdhi}  ~~ \verbroot{brū}  \fullpada{brūhi}
    \item 第五类动词:等于弱语干\\
    \verbroot{śru}  \fullpada{śṛṇu}  
    \item 中间语态:语尾是 \wordending{sva}\\
    \verbroot{yudh}  \fullpada{yudhyasva}  
  \end{itemize}
\end{frame}

\subsection{意义和用法}
\begin{frame}{\insertsubsection }
  \begin{itemize}
    \item 第二人称表示命令
    \item 第三人称表示允许、建议等意义
    \item 第一人称实际少用
  \end{itemize}
\end{frame}

\subsection{表达禁止}
\begin{frame}{禁止的表达方式}
  \begin{itemize}
    \item mā + 命令语气/祈愿语气\\
    \fullsentence{mā gaccha} ~~ \fullsentence{mā gacchet}
    \item alam + 具格\\
    \fullsentence{alaṃ bhayena} ~~ \fullsentence{alaṃ krodhena}
    \item na + 必要分词 (33章) \\
    \fullsentence{naraḥ na hantavyaḥ}   
  \end{itemize}
\end{frame}

\section{本节作业}

\begin{frame}{\insertsection }
  \begin{itemize}
    \item
      第24章练习5
    \item 所有的Readings
  \end{itemize}
\end{frame}  

\end{document}	
