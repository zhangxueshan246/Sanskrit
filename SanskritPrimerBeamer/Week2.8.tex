%%% XeLaTeX-article %%%
%# -*- coding: utf-8 -*-
%!TEX encoding = UTF-8 Unicode
%!TEX TS-program = xelatex  
%---------------------虽然加了%还是要保留!

\documentclass[17pt]{beamer}
\mode<presentation>
{
\usetheme[width=40pt]{Hannover}
\usecolortheme[]{dove}
\usefonttheme[]{structurebold}
\setbeameroption{hide notes}
}

\usepackage{fontspec}
\setmainfont{Arial} %设置主字体
\newfontfamily\sanskritfont[Script=Devanagari,Mapping=romantodevanagari,Scale=1.15]{Sanskrit 2003}             %输出天城体
%\newfontfamily\sanskritfont[Mapping=tex-text]{Times New Roman}              %输出转写
\doublehyphendemerits=-10000
\newcommand{\skt}[1]{{\sanskritfont{#1}}} %输出天城体
\newcommand{\skttrans}[1]{{\skt{#1}~#1}}  %输出天城体和转写
%----------------------------------------------------设置梵文输入方法 danda । ॥

\usepackage[UTF8,fontset=windows]{ctex}
\usepackage{amsmath}
%----------------------------------------------------设置中文环境

\usepackage{graphicx}
\usepackage{flafter} 
\graphicspath{{pic/}}
\usepackage{booktabs} 
\usepackage{nicematrix}
\usepackage{diagbox}
%-----------------------------------------插图表格

\usepackage{hyperref} 
\usepackage[dvipsnames]{xcolor}
\usepackage{colortbl}
\definecolor{light-gray}{gray}{0.9}
%------------------------------颜色

\newcommand{\verbroot}[1]{\textcolor{red}{$\sqrt{}$#1}}
\newcommand{\sktroot}[1]{{\verbroot{\skt{#1}}}}
\newcommand{\skttransroot}[1]{{\sktroot{#1}~\textcolor{red}{#1}}}

\newcommand{\nounstem}[1]{\textcolor{red}{#1\nobreakdash-}}
\newcommand{\sktnounstem}[1]{{\textcolor{red}{\skt{#1\nobreakdash-}}}}
\newcommand{\skttransnounstem}[1]{{\sktnounstem{#1}~\nounstem{#1}}}

\newcommand{\verbstem}[1]{\textcolor{blue}{#1\nobreakdash-}}
\newcommand{\sktverbstem}[1]{{\textcolor{blue}{\skt{#1\nobreakdash-}}}}
\newcommand{\skttransverbstem}[1]{{\sktverbstem{#1}~\verbstem{#1}}}

\newcommand{\wordending}[1]{\textcolor{Orange}{\nobreakdash-#1}}
\newcommand{\sktending}[1]{{\textcolor{Orange}{\skt{-#1}}}}
\newcommand{\skttransending}[1]{{\sktending{#1}~\wordending{#1}}}

\newcommand{\fullpada}[1]{\textcolor{OliveGreen}{#1}}
\newcommand{\sktpada}[1]{{\textcolor{OliveGreen}{\skt{#1}}}}
\newcommand{\skttranspada}[1]{{\sktpada{#1}~\fullpada{#1}}}

\newcommand{\pratyaya}[1]{\textcolor{Plum}{#1}}
\newcommand{\sktpratyaya}[1]{{\textcolor{Plum}{\skt{#1}}}}
\newcommand{\skttranspratyaya}[1]{{\sktpratyaya{#1}~\pratyaya{#1}}}

\newcommand{\reconstruction}[1]{\textcolor{gray}{*#1}}
\newcommand{\fullsentence}[1]{\textcolor{MidnightBlue}{#1}}

\newcommand{\veryimportant}[1]{\textcolor{red}{#1}}
\newcommand{\important}[1]{\textcolor{blue}{#1}}
\newcommand{\notsoimportant}[1]{\textcolor{gray}{#1}}
%-------------------------------------------词根等标颜色

\title{{梵语提高}}
\subtitle{23. 关系从句}
\author[张雪杉]{文学院~~张雪杉 \\ zhangxueshan@sdnu.edu.cn}
\date{}
%\institute{}

\begin{document}	

\begin{frame}
  \titlepage
\end{frame}

\begin{frame}
  \frametitle{本节内容}
  %\small
  \tableofcontents
\end{frame}

\section{上节作业}

\begin{frame}{第22章练习4}
  \raggedright
  \small
  \begin{verse}
    \skt{1) sarve paurā nadyāṃ praveṣṭumaicchan ।}   \\
    \skt{2) ahaṃ pūrva iti bālo hṛṣyamāṇo 'bravīt ।}   \\
    \skt{3) api tau vṛkṣau paśyasīti pṛṣṭa ekaṃ vṛkṣaṃ }\\
    ~~~~~~\mbox{\skt{paśyāmyanyaṃ tu na paśyāmīti punarabravīt ।}}   \\
    \skt{4) kumāraṃ yuddhe hataṃ pariśucyānye vīrā etasmindeśe na santīti nṛpo manyate ।}   \\
    \mbox{\skt{5) sarveṣāṃ devānāmuttamo 'sīti bālo bhāṣate ।}}   \\
    \skt{6) sve mukhe alaṃkṛtya striyau nagarīṃ gantumaicchatām ।}   \\
  \end{verse}
\end{frame}

\begin{frame}{第22章练习4}
  \footnotesize
  \raggedright
  \begin{verse}
    \skt{7) kiṃ vidyuto bibheṣi। sarvaṃ dagdhuṃ śaknoti ।}   \\
    \skt{8) viśve parāṃ senāmabhidrotumicchāmaḥ ।}   \\
    \skt{9) tasyāḥ sundaryā nāryā mukhaṃ draṣṭuṃ na śakṣyāmīti }\\
    ~~~~~~\mbox{\skt{kumāro 'vadat । sarvāmāśāṃ tyaktvāpagantuṃ vṛṇīte ।}}   \\
    \mbox{\skt{10) svānmaraṇānna bibhemi suhṛdo maraṇāttvatibibhemi ।}}   \\
    \mbox{\skt{11) śobhamānā nārī smayamānātkumārātphalamāharate ।}}  \\
    \skt{12) anyeṣāmannaṃ bhoktuṃ nārhasi svakaṃ tu ।}   \\
  \end{verse}
\end{frame}

\subsection{复习}
\begin{frame}{\insertsubsection}
  \begin{itemize}
    \item 中间语态分词\\  
    现在时语干 + \pratyaya{\nobreakdash-māna\nobreakdash-/\nobreakdash-āna\nobreakdash-} \\
    将来时语干 + \pratyaya{\nobreakdash-māna\nobreakdash-} 
    \item 按照第三人称代词变格的
    
    更多代词、代词类形容词
  \end{itemize}
\end{frame}

\section{关系从句}
\begin{frame}{\insertsection }
    \small
    \tableofcontents[currentsection]
\end{frame}

\subsection{关系代词从句}
\begin{frame}{\insertsubsection }
  \begin{itemize}
    \item yad...tad... 各成一句
    \item yad是从句,tad是主句
    \item 需要变格,没有固定语序
  \end{itemize}
\end{frame}

\begin{frame}{\insertsubsection ~~例句}
  \small
  \begin{tabular}{@{}cccccc@{}} % 6 columns
    \fullsentence{yaḥ}  & \fullsentence{ātmanā} & \fullsentence{apatrapate} & \fullsentence{bhṛśaṃ} & \fullsentence{naraḥ}   \\
    who &  by the self & turns away & especially & man \\
  \end{tabular}
  \begin{tabular}{@{}cccccc@{}} % 6 columns
    \fullsentence{sa}  & \fullsentence{sarvalokasya} & \fullsentence{guruḥ} & \fullsentence{bhavati} & \fullsentence{uta}   \\
    he &  of the whole world & teacher & becomes & even \\
  \end{tabular}

  \bigskip
  The man who is especially modest about himself, becomes the teacher of the whole world.\\
  一个深深懂得自我警惕的人,\\能成为一切世人的老师。
\end{frame}

\begin{frame}{\insertsubsection ~~例句}
  \small
  \begin{tabular}{@{}cccccc@{}} % 6 columns
    \fullsentence{nṛpeṇa}  & \fullsentence{avamato} & \fullsentence{yas} & \fullsentence{tu}   \\
    by the king &  looked down & he & but \\
  \end{tabular}
  \begin{tabular}{@{}cccccc@{}} % 6 columns
    \fullsentence{sa}  & \fullsentence{sarvair} & \fullsentence{avamanyate}  \\
    he & by all & is looked down  \\
  \end{tabular}
  
  \bigskip
  But he who is looked down by the king, is looked down by all.
\end{frame}

\subsection{关系副词从句}

\begin{frame}{关系副词表}
  \resizebox{\textwidth}{!}{
  \begin{NiceTabular}{@{}cccccc@{}} % 6 columns
    \Block{2-1}{\diagbox{缀}{干}}  & \Block{1-2}{指示} & & 关系 & 疑问 & 其他 \\
     & \nounstem{a}  & \nounstem{ta}  & \nounstem{ya}  & \nounstem{ku}  &   \\
    \pratyaya{\nobreakdash-tra} & \fullpada{atra}~这里 & \fullpada{tatra}~那里 & \fullpada{yatra} & \fullpada{kutra}~哪里 & \fullpada{ekatra}~在一处  \\
    &  &  &  &  & \fullpada{sarvatra} 处处 \\
    \pratyaya{\nobreakdash-taḥ} & \fullpada{ataḥ}~从这 & \fullpada{tataḥ}~从那 & \fullpada{yataḥ} & \fullpada{kutaḥ}~从哪 & \fullpada{itaḥ}~从此 \\
    &  &  &  &  & \fullpada{sarvataḥ} 到处 \\
    \pratyaya{\nobreakdash-dā} &  & \fullpada{tadā}~那时 & \fullpada{yadā} & \fullpada{kadā}~何时 & \fullpada{ekadā}~一次 \\
    &  &  &  &  & \fullpada{sarvadā} 每次 \\
    \pratyaya{\nobreakdash-thā} &  & \fullpada{tathā}~那样 & \fullpada{yathā} &  & \fullpada{sarvathā}~不管怎样 \\
  \end{NiceTabular}
  }
\end{frame}

\begin{frame}{\insertsubsection ~~例句}
  \footnotesize
  \resizebox{\textwidth}{!}{
  \begin{tabular}{@{}cccccccc@{}} % 6 columns
    \fullsentence{vāsāṃsi}  & \fullsentence{jīrṇāni} & \fullsentence{yathā} & \fullsentence{vihāya} & \fullsentence{navāni} & \fullsentence{gṛhṇāti} & \fullsentence{naro}  & \fullsentence{'parāṇi} \\
    clothes &  old & \important{just as} & take off  &  new & takes & man & other \\
    n.pl.A & n.pl.A & \veryimportant{Adv} & Abs  & n.pl.A & 3.Sg.Pres & m.Sg.N & n.pl.A \\
  \end{tabular}
  }
  \resizebox{\textwidth}{!}{
  \begin{tabular}{@{}cccccccc@{}} % 6 columns
    \fullsentence{tathā}  & \fullsentence{śarīrāṇi} & \fullsentence{vihāya} & \fullsentence{jīrṇāny} & \fullsentence{anyāni} & \fullsentence{saṃyāti} & \fullsentence{navāni}  & \fullsentence{dehī} \\
    \important{so} &  bodies & take off & old & other & takes & new & soul \\
    \veryimportant{Adv} & n.pl.A & Abs & n.pl.A  & n.pl.A & 3.Sg.Pres & n.pl.A & m.Sg.N \\
  \end{tabular}
  }

  \bigskip
  Just as a man takes off old clothes and puts on other new ones, so the soul takes off old bodies and puts on new ones.\\
  正如有人脱掉了旧衣,另外换上一件新衣,\\同样,灵魂解脱了旧身,另入一个新体。
\end{frame}



\subsection{其他相关句法}
\begin{frame}{不定和全称代词}
  \small
  \centering
  \begin{itemize}
    \item 疑问代词加上\fullpada{cit}、\fullpada{api}或\fullpada{cana}
  \end{itemize}
  \begin{tabular}{@{}ll@{}} % 6 columns
    \fullpada{kaḥ} 谁 & \fullsentence{kaś cit / ko 'pi} 某人,不管是谁 \\
    \fullpada{kasya} 谁的 & \fullsentence{kasya cana} 某人的,不管谁的 \\
    \fullpada{kadā} 何时 & \fullsentence{kadā cit} 某一次 \\
  \end{tabular}
  \begin{itemize} 
    \item 重复形式 ~~\fullsentence{yena yena} 不管是谁 
    \item 加na表示全部否定\\    
    \fullsentence{na kaś cit} 没有人 ~
    \fullsentence{na kadā cit} 永远不会
  \end{itemize}  
\end{frame}

\begin{frame}{关系的其他表示方法}
  \small
  \centering
  \begin{itemize}
    \item 关系从句在梵语中不如英语中常见
    \item 相似的意义可以用分词表达\\
    \fullsentence{āgataṃ naram apaśyat} \\he saw the man who had come
    \item 还有复合词\\    
    \fullsentence{nārī hataputrā} \\the woman whose son is killed\\
    \fullsentence{nārī vedavit} \\the woman who knows the Vedas
  \end{itemize}  
\end{frame}

\section{本节作业}

\begin{frame}{\insertsection }
  \begin{itemize}
    \item
      第23章练习1
  \end{itemize}
\end{frame}  

\end{document}	
