%%% XeLaTeX-article %%%
%# -*- coding: utf-8 -*-
%!TEX encoding = UTF-8 Unicode
%!TEX TS-program = xelatex  
%---------------------虽然加了%还是要保留!

\documentclass[17pt]{beamer}
\mode<presentation>
{
\usetheme[width=40pt]{Hannover}
\usecolortheme[]{dove}
\usefonttheme[]{structurebold}
\setbeameroption{hide notes}
}

\usepackage{fontspec}
\setmainfont{Arial} %设置主字体
\newfontfamily\sanskritfont[Script=Devanagari,Mapping=romantodevanagari,Scale=1.15]{Sanskrit 2003}             %输出天城体
%\newfontfamily\sanskritfont[Mapping=tex-text]{Times New Roman}              %输出转写
\doublehyphendemerits=-10000
\newcommand{\skt}[1]{{\sanskritfont{#1}}} %输出天城体
\newcommand{\skttrans}[1]{{\skt{#1}~#1}}  %输出天城体和转写
%----------------------------------------------------设置梵文输入方法

\usepackage[UTF8,fontset=windows]{ctex}
\usepackage{amsmath}
%----------------------------------------------------设置中文环境

\usepackage{graphicx}
\usepackage{flafter} 
\graphicspath{{pic/}}
\usepackage{booktabs} 
%-----------------------------------------插图表格

\usepackage{hyperref} 
\usepackage{xcolor}
%------------------------------颜色

\newcommand{\verbroot}[1]{{$\sqrt{#1}$}}
\newcommand{\sktroot}[1]{{\verbroot{}\skt{#1}}}
\newcommand{\skttransroot}[1]{{\sktroot{#1}~#1}}
%---------------------------------------------------------------词根

\title{{梵语入门}}
\subtitle{3. 动词概述}
\author[张雪杉]{文学院~~张雪杉 \\ zhangxueshan@sdnu.edu.cn}
\date{}
%\institute{}



\begin{document}	


\begin{frame}
  \titlepage
\end{frame}

\begin{frame}
  \frametitle{本节内容}
  \tableofcontents
\end{frame}

\section{上节作业}

\begin{frame}{\insertsubsection ~~第一章练习3}
  \small
  \centering
  \begin{tabular}{@{}lllll@{}} % 6 columns
    a) & \skt{naraḥ}  & \skt{aditiḥ}  & \skt{karma} & \skt{devaḥ}    \\
    & \skt{uṣāḥ} & \skt{śivaḥ}  & \skt{āyurvedaḥ}  & \skt{ekāya}  \\
    & \skt{sarasvatī} & \skt{nāma} & \skt{avatāraḥ}  & \skt{bhārataḥ}   \\
    & \skt{namaste} & \skt{agniḥ} & \skt{priyau} & \skt{ātmā}  \\
    & \skt{rākṣasaḥ}  & \skt{bhagavadgītā} & \skt{cakrāṇi} & \skt{manuḥ} \\
  \end{tabular}
\end{frame}

\begin{frame}{\insertsubsection ~~第一章练习3}
  \small
  \centering
  \begin{tabular}{@{}lllll@{}} % 6 columns
    b) & \skt{kṛṣṇaḥ}  & \skt{devī}  & \skt{gaṇeśaḥ} & \skt{rāmaḥ}  \\
    & \skt{kaliyugam} & \skt{lakṣmī}  & \skt{mitram}  & \skt{muniḥ}  \\
    & \skt{pūjā} & \skt{ṛgvedaḥ} & \skt{hanumān}  & \skt{śāntiḥ}  \\
    & \textcolor{red}{\skt{tantram}} & \skt{yogī} & \skt{amṛtā} & \skt{brāhmaṇaḥ}  \\
    & \skt{itihāsaḥ}  & \skt{mahātmā} & \textcolor{red}{\skt{putraiḥ}} & \skt{saṃsāraḥ} \\
    & \skt{nirvāṇam}  & \skt{paṇḍitaḥ}  & \skt{āśramaḥ} & \skt{īśvarebhyaḥ}   \\
    & \skt{saṃskṛtam} &   &  &   \\
  \end{tabular}
\end{frame}
         
\begin{frame}{\insertsubsection ~~第一章练习3}
  \small
  \centering
  \begin{tabular}{@{}llllll@{}} % 6 columns
    c) & \skt{siṃhaḥ}  & \skt{svāmī}  & \skt{viṣṇuḥ} & \skt{ācāryaḥ}  \\
    & \skt{mahāyānam}  & \skt{ṛṣiḥ}  & \skt{vākṣu}  & \skt{arhaḥ}  \\
    & \skt{iṣṭam} & \skt{annam} & \textcolor{red}{\skt{uktvā}}  & \skt{añjaliḥ}  \\
    & \textcolor{red}{\skt{guptaḥ}} & \skt{akṣaḥ} & \skt{gantum} & \skt{yajñaḥ} \\ 
    & \skt{aśvaḥ}  & \skt{svastiḥ} & \skt{patsu} & \skt{kva} \\
  \end{tabular}
\end{frame}  

\begin{frame}{\insertsubsection ~~第一章练习3}
  \small
  \centering
  \begin{tabular}{@{}llllll@{}} % 6 columns
    d) & \skt{vāgbhiḥ}  & \skt{uttamaḥ}  & \skt{uktam} & \skt{atra}  \\
    & \skt{vatsaḥ} & \skt{strī}  & \skt{bodhisattvaḥ}  & \skt{itthā} \\
    & \textcolor{red}{\skt{vākyam}} & \skt{adya}  & \skt{padbhiḥ}  & \skt{tattat} \\
    & \skt{rātryā} & \skt{labdhāyai} & \skt{mantram}  & \skt{vidyut}  \\
    & \skt{buddhaḥ} & \skt{brahma} &  &  \\
  \end{tabular}
\end{frame}      
   
\section{动词概述}
\begin{frame}{\insertsection }
    \tableofcontents[currentsection]
\end{frame}

\begin{frame}{词类}
  \begin{itemize}
    \item 梵语是一种屈折语
    \item
      按照屈折变化的方式分类:
      
      动词、名词、不变词
  \end{itemize}
  
\end{frame}

\begin{frame}{\insertsection ~~相关概念}
  \begin{itemize}
    \item
      形态层面:
      
      词根、词缀、词干、词尾
    \item
      意义层面:
      
      时态和语气、人称、数、语态
  \end{itemize}
  
\end{frame}

\subsection{动词的形态}

\begin{frame}{\insertsubsection }
  \small
  \centering
  \begin{tabular}{@{}llllll@{}} % 6 columns
    词根 & 词干 & 词尾 & 词  \\
    & (语干) & (语尾) &   \\
    Root & Stem & Ending & Full form   \\    
    \skttransroot{as} & \skttrans{as-} & \skttrans{-ti} & \skttrans{asti}  \\
    \skttransroot{viś}& \skttrans{viśa-} & \skttrans{-ti} & \skttrans{viśati}  \\
    \skttransroot{dā} & \skttrans{dadā-} & \skttrans{-ti} & \skttrans{dadāti} \\
  \end{tabular}
\end{frame}

\subsection{动词的意义}

\begin{frame}{\insertsubsection ~~时态和语气}
  \begin{itemize}
    \item
      时态 Tense
      \begin{itemize}
        \item 时 Time:过去、现在、将来
        \item 体 Aspect:一般、完成、进行
      \end{itemize}
    \item
      语气 Mood:
      
    陈述、祈使、虚拟、条件等
    \item
      主要体现在语干上
  \end{itemize}
\end{frame}

\begin{frame}{\insertsubsection ~~时态和语气}
  \small
  \centering
  \resizebox{\textwidth}{!}{
    \begin{tabular}{@{}lllll@{}} % 7 columns: type, length/type, and 5 vowels
      形态 &   & 意义 &   \\
      时态名称 & 时  & 体 & 语气   \\
      \midrule
      现在时  & 现在  & 一般/进行 & 陈述 \\
      \textcolor{blue}{未完成时}  & 过去  & 进行 & 陈述 \\
      命令语气  & 现在  & 一般/进行 & 祈使 \\
      祈愿语气  & 现在  & 一般/进行 & 可能 \\
      \textcolor{gray}{虚拟语气}  & 将来  & 一般/进行 & 祈求 \\
      将来时  & 将来  & 一般/进行 & 陈述 \\
      迂回将来时  & 将来  & 一般/进行 & 陈述 \\
      条件式  & 将来/过去  & 一般/进行 & 可能/条件 \\
      \textcolor{blue}{完成时}  & 现在 & 完成 & 陈述 \\
      \textcolor{blue}{不定过去时}  & 过去 & 一般 & 陈述 \\
      祈求式  & 现在 & 一般 & 祈使 \\
    \end{tabular} 
  }
\end{frame}

\begin{frame}{\insertsubsection ~~人称、数、语态}
  \begin{itemize}
    \item
      人称 Person:第一、第二、第三
    \item
      数 Number:单数、双数、复数
    \item
      语态 Voice:主动、\textcolor{red}{被动}、中间
    \item
      主要体现在语尾上
  \end{itemize}
\end{frame}

\begin{frame}{动词概念总结}
  \centering
  \begin{tabular}{@{}lllll@{}} % 7 columns: type, length/type, and 5 vowels
    形态 & 意义 \\
    语干 & 时态和语气  \\
    语尾  & 人称 \\
      & 数 \\
      & 语态 \\
  \end{tabular} 
\end{frame}


\section{本节作业}

\begin{frame}{\insertsection }
  \begin{itemize}
    \item
      第二章练习2,第三章练习1
    \item
      阅读教材第3课相关内容
    \bigskip
    \item
      现在请做学习通\nobreakdash-章节\nobreakdash-课后问卷
  \end{itemize}
\end{frame}  

\end{document}	
