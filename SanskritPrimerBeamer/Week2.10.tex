%%% XeLaTeX-article %%%
%# -*- coding: utf-8 -*-
%!TEX encoding = UTF-8 Unicode
%!TEX TS-program = xelatex  
%---------------------虽然加了%还是要保留!

\documentclass[17pt]{beamer}
\mode<presentation>
{
\usetheme[width=40pt]{Hannover}
\usecolortheme[]{dove}
\usefonttheme[]{structurebold}
\setbeameroption{hide notes}
}

\usepackage{fontspec}
\setmainfont{Arial} %设置主字体
\newfontfamily\sanskritfont[Script=Devanagari,Mapping=romantodevanagari,Scale=1.15]{Sanskrit 2003}             %输出天城体
%\newfontfamily\sanskritfont[Mapping=tex-text]{Times New Roman}              %输出转写
\doublehyphendemerits=-10000
\newcommand{\skt}[1]{{\sanskritfont{#1}}} %输出天城体
\newcommand{\skttrans}[1]{{\skt{#1}~#1}}  %输出天城体和转写
%----------------------------------------------------设置梵文输入方法 danda । ॥

\usepackage[UTF8,fontset=windows]{ctex}
\usepackage{amsmath}
%----------------------------------------------------设置中文环境

\usepackage{graphicx}
\usepackage{flafter} 
\graphicspath{{pic/}}
\usepackage{booktabs} 
\usepackage{nicematrix}
\usepackage{diagbox}
%-----------------------------------------插图表格

\usepackage{hyperref} 
\usepackage[dvipsnames]{xcolor}
\usepackage{colortbl}
\definecolor{light-gray}{gray}{0.85}
%------------------------------颜色

\newcommand{\verbroot}[1]{\textcolor{red}{$\sqrt{}$#1}}
\newcommand{\sktroot}[1]{{\verbroot{\skt{#1}}}}
\newcommand{\skttransroot}[1]{{\sktroot{#1}~\textcolor{red}{#1}}}

\newcommand{\nounstem}[1]{\textcolor{red}{#1\nobreakdash-}}
\newcommand{\sktnounstem}[1]{{\textcolor{red}{\skt{#1\nobreakdash-}}}}
\newcommand{\skttransnounstem}[1]{{\sktnounstem{#1}~\nounstem{#1}}}

\newcommand{\verbstem}[1]{\textcolor{blue}{#1\nobreakdash-}}
\newcommand{\sktverbstem}[1]{{\textcolor{blue}{\skt{#1\nobreakdash-}}}}
\newcommand{\skttransverbstem}[1]{{\sktverbstem{#1}~\verbstem{#1}}}

\newcommand{\wordending}[1]{\textcolor{Orange}{\nobreakdash-#1}}
\newcommand{\sktending}[1]{{\textcolor{Orange}{\skt{-#1}}}}
\newcommand{\skttransending}[1]{{\sktending{#1}~\wordending{#1}}}

\newcommand{\fullpada}[1]{\textcolor{OliveGreen}{#1}}
\newcommand{\sktpada}[1]{{\textcolor{OliveGreen}{\skt{#1}}}}
\newcommand{\skttranspada}[1]{{\sktpada{#1}~\fullpada{#1}}}

\newcommand{\pratyaya}[1]{\textcolor{Plum}{#1}}
\newcommand{\sktpratyaya}[1]{{\textcolor{Plum}{\skt{#1}}}}
\newcommand{\skttranspratyaya}[1]{{\sktpratyaya{#1}~\pratyaya{#1}}}

\newcommand{\reconstruction}[1]{\textcolor{gray}{*#1}}
\newcommand{\fullsentence}[1]{\textcolor{MidnightBlue}{#1}}

\newcommand{\veryimportant}[1]{\textcolor{red}{#1}}
\newcommand{\important}[1]{\textcolor{blue}{#1}}
\newcommand{\notsoimportant}[1]{\textcolor{gray}{#1}}
%-------------------------------------------词根等标颜色

\title{{梵语提高}}
\subtitle{25. 多语干名词变格}
\author[张雪杉]{文学院~~张雪杉 \\ zhangxueshan@sdnu.edu.cn}
\date{}
%\institute{}

\begin{document}	

\begin{frame}
  \titlepage
\end{frame}

\begin{frame}
  \frametitle{本节内容}
  %\small
  \tableofcontents
\end{frame}

\section{上节作业}

\begin{frame}{第24章练习5}
  \raggedright
  \small
  \begin{verse}
    \skt{1) parvatasya śira āruhya viśvaṃ lokaṃ draṣṭuṃ śakṣyāmīti cintayitvā kumāraḥ puramatyajat ।
  śirastvāptuṃ nāśaknot ।}   \\
    \skt{2) yatra cakṣūṃṣi tatra vapuḥ ।}   \\
    \mbox{\skt{3) alaṃ bhayena । uttiṣṭha paraṃtapa । parānyudhyasva ।}}\\
    \skt{4) yaśo namaśca sarvasmai kṣatriyāya yo dharmavijjitendriyaśca maraṇānna bibheti ।}   \\
    \skt{5) ojasā nṛpaḥ parānabhibhavitumaśaknot ।}   \\
  \end{verse}
\end{frame}

\begin{frame}{第24章练习5}
  \raggedright
  \small
  \begin{verse}
    \skt{6) bālaḥ kanyāyai phalāni dadātu । kanye tāni phalāni bhuṅkṣva ।}   \\
    \skt{7) namo devebhyaḥ kuru sadā ca tattvaṃ brūhi ।}   \\
    \skt{8) rajasi sucakṣuṣo 'pyacakṣuṣaḥ ।}\\
    \skt{9) sumanasaḥ kanyāyā annamadaduḥ ।}   \\
    \skt{10) kṣatriyā hataṃ mitraṃ labhadhvam ।  svaṃ gṛhaṃ pratyābharata ।}   \\
  \end{verse}
\end{frame}

\begin{frame}{第24章练习5}
  \raggedright
  \small
  \begin{verse}
    \skt{11) vacobhiḥ stuvanti manaḥsu tu kupyanti ।}   \\
    \skt{12) yatkumārā nṛpasya gṛhe 'kurvaṃstatkathayantu ।}   \\
    \skt{13) nṛpasya teja īkṣamāṇāḥ sarve paurāstaṃ namasā pūjayanti ।}\\
    \mbox{\skt{14) naro duḥkhaṃ mitraṃ dṛṣṭvā tadvaco 'bravīt ।}}\\
    ~~~~~~\skt{śṛṇu mitra alaṃ cintayeti । śvo 'nyaṃ yatnaṃ kuru ।}   \\
    \skt{15) kutrottamāḥ parvatā iti pṛṣṭvā kumārastāndraṣṭuṃ gacchati ।}   \\
  \end{verse}
\end{frame}

\begin{frame}{佛教书第29页前5题}
  \small
  %\raggedright
    1. tasmin pradeśe tathāgataḥ viharati | \\
    2. ekasmin samaye buddhaḥ jetavane anāthapiṇḍadasya ārāme viharati | 
prajñapte eva āsane niṣīdati sma |   \\
    3. bodhisattvaḥ āsanāt ut-tiṣṭhati | tathāgatam ca evam vadati | \\
    4. puruṣaḥ dine dine ātma-bhāvān parityajati |  \\
    5. evam etad yathā vadasi |  \\
\end{frame}

\subsection{复习}

\begin{frame}{辅音格尾表}
  \small
  \centering
    \begin{NiceTabular}{|c|c|c|c|c|c|c|}[hvlines, rules/width=0.3pt, rules/color=gray]
       & \Block{1-2}{单数} & & \Block{1-2}{双数} & & \Block{1-2}{复数} &  \\
       & 阳阴 & 中 & 阳阴 & 中 & 阳阴 & 中  \\
      主 & \wordending{s} & \Block{3-1}{\wordending{}} & \Block{3-1}{\wordending{au}}  & \Block{3-1}{\wordending{ī}}  & \Block{3-1}{\wordending{aḥ}} &  \Block{3-1}{\wordending{ni}}   \\
      呼 & \wordending{} & & & & & \\
      业 & \wordending{am} &  & & & & \\
      具 & \Block{1-2}{\wordending{ā}} &  & \Block{3-2}{\wordending{bhyām}}  & & \Block{1-2}{\wordending{bhiḥ}} & \\
      为 & \Block{1-2}{\wordending{e}} &  & & & \Block{2-2}{\wordending{bhyaḥ}}  & \\
      从 & \Block{2-2}{\wordending{aḥ}} &  & & & & \\
      属 & &  & \Block{2-2}{\wordending{oḥ}} & &  \Block{1-2}{\wordending{ām}} & \\
      依 & \Block{1-2}{\wordending{i}} &  & & & \Block{1-2}{\wordending{su}}&  \\
    \end{NiceTabular}
\end{frame}

\begin{frame}{命令语气语尾}  
  \centering
  \includegraphics[width=\textwidth]{imperativeendings.png} %
\end{frame}

\section{多语干名词}
\begin{frame}{\insertsection }
    \small
    \tableofcontents[currentsection]
\end{frame}

\begin{frame}{\insertsection }
  \begin{itemize}
    \item \nounstem{\nobreakdash-vant}, \nounstem{\nobreakdash-mant}, \nounstem{\nobreakdash-ant}
    \item 词内连声和以 t 结尾的名词一样
    \item 但他们还有语干变化
    \item 强语干是横着背表前五个
    \item 弱语干是 \nounstem{\nobreakdash-vat}, \nounstem{\nobreakdash-mat}, \nounstem{\nobreakdash-at}
  \end{itemize}
\end{frame}

\subsection{vant,mant}

\begin{frame}{\nounstem{\nobreakdash-vant}, \nounstem{\nobreakdash-mant} 变格}
  \small
  \nounstem{guṇavant} adj. 有德 
  \bigskip

  \centering
  \resizebox{\textwidth}{!}{
    \begin{NiceTabular}{|c|c|c|c|c|c|c|}[hvlines, rules/width=0.3pt, rules/color=gray]
      阳性 & 单数 & 双数 & 复数  \\
      主 & \cellcolor{light-gray}\fullpada{guṇavān}  & \cellcolor{light-gray}\Block{3-1}{\fullpada{guṇavantau}} & \cellcolor{light-gray}\Block{2-1}{\fullpada{guṇavantaḥ}}  \\
      呼 & \cellcolor{light-gray}\fullpada{guṇavan} & \cellcolor{light-gray} & \cellcolor{light-gray} \\
      业 & \cellcolor{light-gray}\fullpada{guṇavantam} & \cellcolor{light-gray} & \fullpada{guṇavataḥ} \\
      中性 & \Block{2-1}{\fullpada{guṇavat}}  & \Block{2-1}{\fullpada{guṇavatī}} & \Block{2-1}{\fullpada{guṇavanti}}   \\
      主呼业 &  &  &  \\
      具 & \fullpada{guṇavatā} & \Block{3-1}{\fullpada{guṇavadbhyām}} & \fullpada{guṇavadbhiḥ} \\
      为 & \fullpada{guṇavate} &  & \Block{2-1}{\fullpada{guṇavadbhyaḥ}} \\
      从 & \Block{2-1}{\fullpada{guṇavataḥ}} &  &  \\
      属 &  & \Block{2-1}{\fullpada{guṇavatoḥ}} & \fullpada{guṇavatām} \\
      依 & \fullpada{guṇavati} &  & \fullpada{guṇavatsu} \\
    \end{NiceTabular}
  }
\end{frame}

\begin{frame}{\nounstem{\nobreakdash-vant}, \nounstem{\nobreakdash-mant} 意义}
  \begin{itemize}
    \item \pratyaya{\nobreakdash-vant\nobreakdash-}, \pratyaya{\nobreakdash-mant\nobreakdash-} 词缀表所有或性质\\
    \nounstem{guṇavant} 有德的~~ \nounstem{dhīmant} 聪明的
    \item 阴性是弱语干加 \pratyaya{\nobreakdash-ī\nobreakdash-}
    \item \pratyaya{\nobreakdash-vat\nobreakdash-} 作副词表示方式\\
    \fullpada{cakravat} ~~ 像车轮一样
    \item \pratyaya{\nobreakdash-tavant\nobreakdash-} 构成过去主动分词\\
    \nounstem{likhita} 被写的~~ \nounstem{likhitavant} 写过的
  \end{itemize}
\end{frame}

\subsection{ant}
\begin{frame}{\nounstem{\nobreakdash-ant} 分词}
  \begin{itemize}
    \item \pratyaya{\nobreakdash-ant\nobreakdash-} 词缀构成现在/将来主动分词\\
    \begin{itemize}
      \item \nounstem{\nobreakdash-a} 语干加 \pratyaya{\nobreakdash-nt\nobreakdash-} \nounstem{gacchant}
      \item 非 \nounstem{\nobreakdash-a} 弱语干加 \pratyaya{\nobreakdash-ant\nobreakdash-}  \nounstem{ghnant}
    \end{itemize}
    \item 变格基本同 \nounstem{\nobreakdash-vant} 
    \begin{itemize}
      \item 阳性单数主格是 \pratyaya{\nobreakdash-an} \fullpada{paśyan}
      \item 第三类动词全用弱语干 \fullpada{juhvat}
    \end{itemize}
    \item 阴性都是加 \pratyaya{\nobreakdash-ī\nobreakdash-}\\
    \begin{itemize}
      \item 现在分词 \nounstem{\nobreakdash-a} 语干用强语干 \nounstem{paśyantī}
      \item 非 \nounstem{\nobreakdash-a} 语干用弱语干 \nounstem{bibhyatī}
      \item 将来分词多用强语干,有时也用弱语干 \nounstem{bhariṣyantī\nobreakdash- / bhariṣyatī}
    \end{itemize}
  \end{itemize}
\end{frame}

\subsection{mahānt}

\begin{frame}{\nounstem{mahānt} 变格}
  %\small
  \nounstem{mahānt} adj. 大 

  \begin{itemize}
      \item 强语干三合 
      \item 常见形式是 \nounstem{mahā}
    \end{itemize}
  \centering
  \bigskip
  {\small
    \begin{NiceTabular}{|c|c|c|c|c|c|c|}[hvlines, rules/width=0.3pt, rules/color=gray]
      阳性 & 单数 & 双数 & 复数  \\
      主 & \cellcolor{light-gray}\fullpada{mahān}  & \cellcolor{light-gray}\Block{3-1}{\fullpada{mahāntau}} & \cellcolor{light-gray}\Block{2-1}{\fullpada{mahāntaḥ}}  \\
      呼 & \cellcolor{light-gray}\fullpada{mahan} & \cellcolor{light-gray} & \cellcolor{light-gray} \\
      业 & \cellcolor{light-gray}\fullpada{mahāntam} & \cellcolor{light-gray} & \fullpada{mahataḥ} \\
      中性 & \Block{2-1}{\fullpada{mahat}}  & \Block{2-1}{\fullpada{mahatī}} & \Block{2-1}{\fullpada{mahānti}}   \\
      主呼业 &  &  &  \\
    \end{NiceTabular}
  }
\end{frame}

\section{本节作业}

\begin{frame}{\insertsection }
  \begin{itemize}
    \item
      第25章练习4,注意勘误
    \item 佛教书第29页6-11题
  \end{itemize}
\end{frame}  

\end{document}	
