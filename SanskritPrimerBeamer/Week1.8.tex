%%% XeLaTeX-article %%%
%# -*- coding: utf-8 -*-
%!TEX encoding = UTF-8 Unicode
%!TEX TS-program = xelatex  
%---------------------虽然加了%还是要保留!

\documentclass[17pt]{beamer}
\mode<presentation>
{
\usetheme[width=40pt]{Hannover}
\usecolortheme[]{dove}
\usefonttheme[]{structurebold}
\setbeameroption{hide notes}
}

\usepackage{fontspec}
\setmainfont{Arial} %设置主字体
\newfontfamily\sanskritfont[Script=Devanagari,Mapping=romantodevanagari,Scale=1.15]{Sanskrit 2003}             %输出天城体
%\newfontfamily\sanskritfont[Mapping=tex-text]{Times New Roman}              %输出转写
\doublehyphendemerits=-10000
\newcommand{\skt}[1]{{\sanskritfont{#1}}} %输出天城体
\newcommand{\skttrans}[1]{{\skt{#1}~#1}}  %输出天城体和转写
%----------------------------------------------------设置梵文输入方法 danda । ॥

\usepackage[UTF8,fontset=windows]{ctex}
\usepackage{amsmath}
%----------------------------------------------------设置中文环境

\usepackage{graphicx}
\usepackage{flafter} 
\graphicspath{{pic/}}
\usepackage{booktabs} 
\usepackage{nicematrix}
%-----------------------------------------插图表格

\usepackage{hyperref} 
\usepackage[dvipsnames]{xcolor}
\usepackage{colortbl}
\definecolor{light-gray}{gray}{0.9}
%------------------------------颜色

\newcommand{\verbroot}[1]{\textcolor{red}{$\sqrt{}$#1}}
\newcommand{\sktroot}[1]{{\verbroot{\skt{#1}}}}
\newcommand{\skttransroot}[1]{{\sktroot{#1}~\textcolor{red}{#1}}}

\newcommand{\nounstem}[1]{\textcolor{red}{#1\nobreakdash-}}
\newcommand{\sktnounstem}[1]{{\textcolor{red}{\skt{#1\nobreakdash-}}}}
\newcommand{\skttransnounstem}[1]{{\sktnounstem{#1}~\nounstem{#1}}}

\newcommand{\verbstem}[1]{\textcolor{blue}{#1\nobreakdash-}}
\newcommand{\sktverbstem}[1]{{\textcolor{blue}{\skt{#1\nobreakdash-}}}}
\newcommand{\skttransverbstem}[1]{{\sktverbstem{#1}~\verbstem{#1}}}

\newcommand{\wordending}[1]{\textcolor{Orange}{\nobreakdash-#1}}
\newcommand{\sktending}[1]{{\textcolor{Orange}{\skt{-#1}}}}
\newcommand{\skttransending}[1]{{\sktending{#1}~\wordending{#1}}}

\newcommand{\fullpada}[1]{\textcolor{OliveGreen}{#1}}
\newcommand{\sktpada}[1]{{\textcolor{OliveGreen}{\skt{#1}}}}
\newcommand{\skttranspada}[1]{{\sktpada{#1}~\fullpada{#1}}}

\newcommand{\pratyaya}[1]{\textcolor{Plum}{#1}}
\newcommand{\sktpratyaya}[1]{{\textcolor{Plum}{\skt{#1}}}}
\newcommand{\skttranspratyaya}[1]{{\sktpratyaya{#1}~\pratyaya{#1}}}

\newcommand{\reconstruction}[1]{\textcolor{gray}{*#1}}
\newcommand{\fullsentence}[1]{\textcolor{MidnightBlue}{#1}}

\newcommand{\veryimportant}[1]{\textcolor{red}{#1}}
\newcommand{\important}[1]{\textcolor{blue}{#1}}
\newcommand{\notsoimportant}[1]{\textcolor{gray}{#1}}
%-------------------------------------------词根等标颜色

\title{{梵语入门}}
\subtitle{9. 以ā结尾的名词变格}
\author[张雪杉]{文学院~~张雪杉 \\ zhangxueshan@sdnu.edu.cn}
\date{}
%\institute{}

\begin{document}	

\begin{frame}
  \titlepage
\end{frame}

\begin{frame}
  \frametitle{本节内容}
  \small
  \tableofcontents
\end{frame}

\section{期中考试}

\begin{frame}{\insertsection }
  \large
  \begin{itemize}
    \item
      默写梵语字母表\\包括天城体和转写体
  \end{itemize}
\end{frame}

\section{上节作业}

\begin{frame}{第八章练习4}
  \small
  \raggedright
  \begin{verse}
    \skt{1) pāpān yuddhe jitvā śūrāḥ hṛṣyanti ।}   \\
    \skt{2) tava mitrāṇi grāme sthitāni paśyāmi ।}   \\
    \skt{3) vyāghrāt bālaṃ pālayitvā aśvaḥ bālena saha īśvarasya gṛhaṃ gacchati ।}   \\
    \skt{4) siṃhaḥ iha kiṃ karoti iti cintayitvā bālaḥ gṛhaṃ dravati ।}  \\
    \mbox{\skt{5) priyaṃ kulaṃ tyaktvā kumāraḥ mitrāṇi yuddhaṃ nayati ।}}  \\
    \skt{6) dūtaḥ vanaṃ gataḥ । kiṃ tatra gacchasi iti puruṣeṇa pṛṣṭaḥ ।}   \\
  \end{verse}
\end{frame}  

\begin{frame}{第八章练习4}
  \small
  \raggedright
  \begin{verse}
    \skt{7) gṛhaṃ mitrābhyāṃ saha tyaktvā bālaḥ purāt vanaṃ dravati ।}  \\
    \skt{8) śūraiḥ jitasya nagarasya kathāḥ janān harṣayanti । janāḥ śūrebhyaḥ eva namanti ।}   \\
    \skt{9) pure vasāmi iti uktvā bālaḥ naraṃ kiṃ vane vasasi iti pṛcchati ।}   \\
    \skt{10) devāḥ guṇān paśyanti iti vadāmaḥ ।}   \\
    \mbox{\skt{11) api īśvarasya gṛhaṃ gatvā janāḥ īśvarāya natāḥ ।}} \\
    \skt{12) gṛhaṃ kulaṃ ca tyaktuṃ na icchāmi iti uktvā kumāraḥ sīdati ।}   \\
  \end{verse}
\end{frame}  

\begin{frame}{第八章练习4}
  \footnotesize
  \raggedright
  \begin{verse}
    \skt{13) mitrāṇi draṣṭuṃ gacchāmaḥ iti uktvā bālāḥ gṛhaṃ tyajanti ।}   \\
    \skt{14) nagaraṃ pāpaiḥ jitaṃ dṛṣṭvā śūrāḥ tata nagaraṃ veṣṭuṃ na icchanti ।}   \\
    \skt{15) putraḥ vṛkṣāt patitāni phalāni hṛtvā nṛpāya dātum icchati ।}   \\
    \skt{16) putraḥ vṛkṣāt nareṇa pātitāni phalāni hṛtvā nṛpāya dātum icchati ।}   \\
    \skt{17) ahaṃ vanaṃ gantuṃ na icchāmi ।}  \\
    \mbox{\skt{18) dagdhaṃ kṣetraṃ puraṃ ca tyaktam iha dṛṣṭvā janāḥ śocanti ।}}   \\
  \end{verse}
\end{frame}  

\subsection{复习}

\begin{frame}{\insertsubsection}
  \begin{itemize}
    \item 独立式
    
    词根零级 + \pratyaya{\nobreakdash-tvā} 
    \item 过去分词
    
    词根零级 + \pratyaya{\nobreakdash-ta\nobreakdash-} \veryimportant{(按名词变格)}
    \item 不定式  
    
    词根二合 + \pratyaya{\nobreakdash-tum}
  \end{itemize}
\end{frame}

\section{名词相关}
\begin{frame}{\insertsection }
  \small
  \tableofcontents[currentsection]
\end{frame}

\subsection{ā的变格}

\begin{frame}{以ā结尾的名词是阴性}
  \small
  \centering
  \resizebox{\textheight}{!}{
    \begin{tabular}{@{}llllll@{}} % 6 columns
       & 单数 & 双数 & 复数  \\
      主 & \fullpada{senā}  & \fullpada{sene} & \fullpada{senāḥ}   \\
      呼 & \fullpada{sene} & \fullpada{sene} & \fullpada{senāḥ} \\
      业 & \fullpada{senām} & \fullpada{sene} & \fullpada{senāḥ} \\
      具 & \fullpada{senayā} & \fullpada{senābhyām} & \fullpada{senābhiḥ} \\
      为 & \fullpada{senāyai} & \fullpada{senābhyām} & \fullpada{senābhyaḥ} \\
      从 & \fullpada{senāyāḥ} & \fullpada{senābhyām} & \fullpada{senābhyaḥ} \\
      属 & \fullpada{senāyāḥ} & \fullpada{senayoḥ} & \fullpada{senānām} \\
      依 & \fullpada{senāyāṃ} & \fullpada{senayoḥ} & \fullpada{senāsu} \\
    \end{tabular}
  }
\end{frame}

\begin{frame}{复习 deva 阳性}
  \small
  \centering
  \resizebox{\textheight}{!}{
    \begin{tabular}{@{}llllll@{}} % 6 columns
       & 单数 & 双数 & 复数  \\
      主 & \fullpada{devaḥ}  & \fullpada{devau} & \fullpada{devāḥ}   \\
      呼 & \fullpada{deva} & \fullpada{devau} & \fullpada{devāḥ} \\
      业 & \fullpada{devam} & \fullpada{devau} & \fullpada{devān} \\
      具 & \fullpada{devena} & \fullpada{devābhyām} & \fullpada{devaiḥ} \\
      为 & \fullpada{devāya} & \fullpada{devābhyām} & \fullpada{devebhyaḥ} \\
      从 & \fullpada{devāt} & \fullpada{devābhyām} & \fullpada{devebhyaḥ} \\
      属 & \fullpada{devasya} & \fullpada{devayoḥ} & \fullpada{devānām} \\
      依 & \fullpada{deve} & \fullpada{devayoḥ} & \fullpada{deveṣu} \\
    \end{tabular}
  }
\end{frame}

\subsection{形容词}
\begin{frame}{a/ā结尾形容词}
  \small
  \begin{itemize}
    \item 形容词跟着它修饰的名词变格
    
    \item 以a结尾的形容词阴性以ā或ī结尾
  \end{itemize}
  \centering
    \begin{tabular}{@{}llllll@{}} % 6 columns
       & 阳性 & 阴性 & 中性  \\
      主 & \fullpada{priyaḥ}  & \fullpada{priyā} & \fullpada{priyam}   \\
      呼 & \fullpada{priya} & \fullpada{priye} & \fullpada{priya} \\
      业 & \fullpada{priyam} & \fullpada{priyām} & \fullpada{priyam} \\
       & 以下略 &  &  \\
    \end{tabular}
\end{frame}

\begin{frame}{比较级和最高级}
  \small
  \begin{itemize}
    \item 比较级和最高级有两套词缀,意义相同  
    \item \pratyaya{\nobreakdash-tara\nobreakdash-/\nobreakdash-tama\nobreakdash-} 比 \pratyaya{\nobreakdash-(ī)yas\nobreakdash-/\nobreakdash-iṣṭha\nobreakdash-} 更常用
    \item 具体某个形容词用哪一套没有规律
  \end{itemize}
  \centering
    \begin{tabular}{@{}llllll@{}} % 6 columns
      形容词 & 比较级 & 最高级  \\
      \nounstem{ugra}  & \nounstem{ugratara} & \nounstem{ugratama}   \\
      \nounstem{priya} & \nounstem{preyas} & \nounstem{preṣṭha} \\
    \end{tabular}
\end{frame}

\begin{frame}{比较级和最高级~~句法}
  \small
  \begin{itemize}
    \item 比较级与从格连用 
    
    \fullsentence{siṃhaḥ aśvāt śīghrataraḥ} \\The lion is faster than the horse.
    \item 最高级与属格或依格连用
    
    \fullsentence{preṣṭhaḥ narāṇām} dearest of men.\\
    \fullsentence{preṣṭhaḥ narēṣu} dearest among men.
    \item 不和其他成分连用时也可译为 very / exceedingly 等
  \end{itemize}
\end{frame}

\subsection{副词}
\begin{frame}{\insertsubsection }
  \small
  \begin{itemize}
    \item 形容词中性单数业格一般可用作副词
    
    \nounstem{nitya} \fullpada{nityam} always \\
    \nounstem{cira} \fullpada{cīram} for a long time \\
    \nounstem{sukha} \fullpada{sukham} happily
    \item 名词有时也可以按照同样方式形成副词
    
    \nounstem{nāman} \fullpada{nāma} called \\
    \nounstem{rahas} \fullpada{rahas} secretly
  \end{itemize}
\end{frame}

\begin{frame}{句法~~go + 抽象名词}
  \small
  \begin{itemize}
    \item go 加中性名词业格通常指动作终点\\     
    \fullsentence{paraṃ gacchati}  He goes to the city. 
    \item go 加抽象名词业格可译为变成 \\
    \fullsentence{siddhiṃ gacchati}  He becomes successful. \\
    \fullsentence{vismayaṃ gacchati}  He becomes astonished. 
  \end{itemize}  
\end{frame}

\section{词内连声}
\begin{frame}{\insertsection }
  \small
  \tableofcontents[currentsection]
\end{frame}

\subsection{ruki规则}
\small
\begin{frame}{\insertsection }
  \begin{itemize}
    \item 前音是非a/ā元音、半元音、h或者喉音时,s变成ṣ\\
    \nounstem{deva} + \pratyaya{\nobreakdash-su} = \fullpada{deveṣu}\\
    反例:a 后不变 \\ \nounstem{senā} + \pratyaya{\nobreakdash-su} = \fullpada{senāsu}
    \item 前音和s中间插入了ṃ、ḥ或者咝音仍要变化\\
    \nounstem{yajuḥ} + \pratyaya{\nobreakdash-su} = \fullpada{yajuḥṣu}
    \item 词尾的s不变 
    ~~\fullsentence{agnis tatra} 
  \end{itemize}
\end{frame}


\section{本节作业}

\begin{frame}{\insertsection }
  \begin{itemize}
    \item
      第九章练习4
    \item
      阅读教材第9课相关内容
    \bigskip
    \item
      现在请做学习通\nobreakdash-章节\nobreakdash-课后问卷
  \end{itemize}
\end{frame}  

\end{document}	