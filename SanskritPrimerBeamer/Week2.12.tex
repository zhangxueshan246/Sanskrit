%%% XeLaTeX-article %%%
%# -*- coding: utf-8 -*-
%!TEX encoding = UTF-8 Unicode
%!TEX TS-program = xelatex  
%---------------------虽然加了%还是要保留!

\documentclass[17pt]{beamer}
\mode<presentation>
{
\usetheme[width=40pt]{Hannover}
\usecolortheme[]{dove}
\usefonttheme[]{structurebold}
\setbeameroption{hide notes}
}

\usepackage{fontspec}
\setmainfont{Arial} %设置主字体
\newfontfamily\sanskritfont[Script=Devanagari,Mapping=romantodevanagari,Scale=1.15]{Sanskrit 2003}             %输出天城体
%\newfontfamily\sanskritfont[Mapping=tex-text]{Times New Roman}              %输出转写
\doublehyphendemerits=-10000
\newcommand{\skt}[1]{{\sanskritfont{#1}}} %输出天城体
\newcommand{\skttrans}[1]{{\skt{#1}~#1}}  %输出天城体和转写
%----------------------------------------------------设置梵文输入方法 danda । ॥

\usepackage[UTF8,fontset=windows]{ctex}
\usepackage{amsmath}
%----------------------------------------------------设置中文环境

\usepackage{graphicx}
\usepackage{flafter} 
\graphicspath{{pic/}}
\usepackage{booktabs} 
\usepackage{nicematrix}
\usepackage{diagbox}
%-----------------------------------------插图表格

\usepackage{hyperref} 
\usepackage[dvipsnames]{xcolor}
\usepackage{colortbl}
\definecolor{light-gray}{gray}{0.85}
%------------------------------颜色

\newcommand{\verbroot}[1]{\textcolor{red}{$\sqrt{}$#1}}
\newcommand{\sktroot}[1]{{\verbroot{\skt{#1}}}}
\newcommand{\skttransroot}[1]{{\sktroot{#1}~\textcolor{red}{#1}}}

\newcommand{\nounstem}[1]{\textcolor{red}{#1\nobreakdash-}}
\newcommand{\sktnounstem}[1]{{\textcolor{red}{\skt{#1\nobreakdash-}}}}
\newcommand{\skttransnounstem}[1]{{\sktnounstem{#1}~\nounstem{#1}}}

\newcommand{\verbstem}[1]{\textcolor{blue}{#1\nobreakdash-}}
\newcommand{\sktverbstem}[1]{{\textcolor{blue}{\skt{#1\nobreakdash-}}}}
\newcommand{\skttransverbstem}[1]{{\sktverbstem{#1}~\verbstem{#1}}}

\newcommand{\wordending}[1]{\textcolor{Orange}{\nobreakdash-#1}}
\newcommand{\sktending}[1]{{\textcolor{Orange}{\skt{-#1}}}}
\newcommand{\skttransending}[1]{{\sktending{#1}~\wordending{#1}}}

\newcommand{\fullpada}[1]{\textcolor{OliveGreen}{#1}}
\newcommand{\sktpada}[1]{{\textcolor{OliveGreen}{\skt{#1}}}}
\newcommand{\skttranspada}[1]{{\sktpada{#1}~\fullpada{#1}}}

\newcommand{\pratyaya}[1]{\textcolor{Plum}{#1}}
\newcommand{\sktpratyaya}[1]{{\textcolor{Plum}{\skt{#1}}}}
\newcommand{\skttranspratyaya}[1]{{\sktpratyaya{#1}~\pratyaya{#1}}}

\newcommand{\reconstruction}[1]{\textcolor{gray}{*#1}}
\newcommand{\fullsentence}[1]{\textcolor{MidnightBlue}{#1}}
\newcommand{\reduplicativesyllable}[1]{\textcolor{WildStrawberry}{#1}}

\newcommand{\veryimportant}[1]{\textcolor{red}{#1}}
\newcommand{\important}[1]{\textcolor{blue}{#1}}
\newcommand{\notsoimportant}[1]{\textcolor{gray}{#1}}
%-------------------------------------------词根等标颜色

\title{{梵语提高}}
\subtitle{27-28. 完成时}
\author[张雪杉]{文学院~~张雪杉 \\ zhangxueshan@sdnu.edu.cn}
\date{}
%\institute{}

\begin{document}	

\begin{frame}
  \titlepage
\end{frame}

\begin{frame}
  \frametitle{本节内容}
  %\small
  \tableofcontents
\end{frame}

\section{上节作业}

\begin{frame}{第26章练习4}
  \raggedright
  \small
  \begin{verse}
    \skt{1) gireḥ śiro dṛṣṭvā kumārāḥ kṣaṇena taṃ rohitumaicchan ।} \\
    \skt{2) kṣatriyo mahābalo 'pyarīṇāṃ bahubhiriṣubhirjitaḥ ।} \\
    \skt{3) sa yatnaḥ phalavānevāsīditi nandantaḥ kṣatriyā avadan ।}\\
    \skt{4) sumatayo durmatīnnayeyuḥ ।}   \\
    \skt{5) iṣubhiraraya ṛṣīnagnerapadrāvayanti ।}\\
  \end{verse}
\end{frame}

\begin{frame}{第26章练习4}
  \raggedright
  \small
  \begin{verse}
    \skt{6) vepamānasya bālasyāśrūṇi dṛṣṭvā śatrurannaṃ tasmānnāpāharat ।}\\
    \skt{7) nagarakṣetragiriṣu gatvā kumāro vṛddhabuddhiḥ priyaṃ nagaraṃ pratyāgacchat ।}\\
    \skt{8) gurave praṇatya narastaṃ gṛhe 'nayat ।}\\
    \skt{9) śatrorbahūni vasūnyapahṛtya tuṣyanto narāḥ svakāṃ nagarīṃ pratyāgacchan ।}\\
    \skt{10) grāmaṃ śīghraṃ tyaktvoṣasi mahābuddhi\veryimportant{rṛ}ṣirnamo 'karot ।}\\
  \end{verse}
\end{frame}

\subsection{复习}
\begin{frame}{i u 变格}  
  \centering
  \includegraphics[width=\textwidth]{iudeclension.png} %
\end{frame}


\section{完成时}
\begin{frame}{\insertsection }
    \small
    \tableofcontents[currentsection]
\end{frame}

\begin{frame}{\insertsection ~~例词}
  \small
  \verbroot{viś(1)} 进入
  \bigskip

  \centering 
  \begin{tabular}{@{}llll@{}} % 4 columns
    P. & 单数  & 双数  & 复数 \\
    1st & \cellcolor{light-gray}\fullpada{viveśa} & \fullpada{viviśiva}  & \fullpada{viviśima}  \\
    2nd & \cellcolor{light-gray}\fullpada{viveśitha} & \fullpada{viviśathuḥ}  & \fullpada{viviśa}  \\
    3rd & \cellcolor{light-gray}\fullpada{viveśa} & \fullpada{viviśatuḥ}  & \fullpada{viviśuḥ}  \\
  \end{tabular} 
  \bigskip
    
  \begin{tabular}{@{}llll@{}} % 4 columns
    Ā. & 单数  & 双数  & 复数 \\
    1st & \fullpada{viviśe} & \fullpada{viviśivahe} & \fullpada{viviśimahe} \\
    2nd & \fullpada{viviśiṣe} & \fullpada{viviśāthe} & \fullpada{viviśidhve} \\
    3rd & \fullpada{viviśe} & \fullpada{viviśāte} & \fullpada{viviśire} \\
  \end{tabular}   
\end{frame}

\begin{frame}{\insertsection }
  \small
  \begin{itemize}
    \item 完成时的构成:\\
  \end{itemize}
    \centering 
  \resizebox{\textwidth}{!}{
    \begin{NiceTabular}{@{}cccc@{}} % 4 columns
      重复音节  & 等级完成时语干  & 联系元音 & 完成时语尾 \\
      \\
      \cellcolor{light-gray}\fullpada{viveśitha} & P.2nd.Sg.  & & \\
      \reduplicativesyllable{vi} & \cellcolor{light-gray}\verbstem{veś} &  \pratyaya{i} & \wordending{tha}  \\
      \\
      \fullpada{viviśe} & Ā.1st/3rd.Sg. &  & \\      
      \reduplicativesyllable{vi} & \verbstem{viś} &   & \wordending{e}  \\
    \end{NiceTabular} 
  }
\end{frame}

\subsection{重复音节}
\begin{frame}{\insertsubsection ~~辅音}
  \small
  \begin{itemize}
    \item 重复音节的辅音\\和第三类动词现在时语干一样。\\
    \item 送气变不送气 ~~\verbroot{bhuj} \fullpada{bubhoja}
    \item 喉音变腭音 ~~\verbroot{kṛ} \fullpada{cakara}
    \item h(\reconstruction{gh})变j ~~\verbroot{has} \fullpada{jahasa}
    \item 复合辅音重复第一个 ~~\verbroot{kṣip} \fullpada{jikṣepa}\\
    \item 复合咝音不重复 ~~\verbroot{sthā} \fullpada{tasthau}  \\ 
  \end{itemize}
\end{frame}

\begin{frame}{\insertsubsection ~~元音}
  \small
  \begin{itemize}
    \item 重复音节的元音\\和第三类动词现在时语干不一样。\\
    \item 长短i/ī,u/ū和重复短i短u \\
    ~~\verbroot{kṣip} \fullpada{jikṣepa}
    ~~\verbroot{bhuj} \fullpada{bubhoja}\\
    \item 其他全都重复a\\
     ~~\verbroot{kṛ} \fullpada{cakara}
     ~~\verbroot{dā} \fullpada{dadau}  \\ 
    \item 复合元音在词中重复i u,在词尾重复a\\
    ~~\verbroot{sev} \fullpada{siṣeve}
    ~~\verbroot{gai} \fullpada{jagau}  \\ 
  \end{itemize}
\end{frame}

\begin{frame}{\insertsubsection ~~不规则}
  \small
  \begin{itemize}
    \item 元音开头的词根\\
    重复音节是对应零级元音, \\
    强语干加对应半元音。 \\

    \bigskip
    ~~\verbroot{iṣ} ~1st.Sg. \fullpada{iyeṣa} 3rd.Pl. \fullpada{īṣuḥ}\\
    ~~\verbroot{uṣ} 1st.Sg. \fullpada{uvoṣa} 3rd.Pl. \fullpada{ūṣuḥ} \\
    \item y和v开头的二合词根也重复零级元音。 \\
    
    \bigskip
    ~~\verbroot{yaj} 1st.Sg. \fullpada{iyaja} 3rd.Pl. \fullpada{ījuḥ}\\
    ~~\verbroot{vac} 1st.Sg. \fullpada{uvaca} 3rd.Pl. \fullpada{ūcuḥ} 
  \end{itemize}
\end{frame}

\subsection{等级完成时语干}
\begin{frame}{\insertsubsection ~~规则}
  \small
  \begin{itemize}
    \item 强语干二合,弱语干零级。\\
    ~~\verbroot{viś} 1st.Sg. \fullpada{viveśa} 3rd.Pl. \fullpada{viviśuḥ}\\
    ~~\verbroot{vac} 1st.Sg. \fullpada{uvaca} 3rd.Pl. \fullpada{ūcuḥ} \\
    \item 鼻音结尾的弱语干,虽然规则但不好认。\\ 
  \end{itemize}
    \centering 
  \begin{tabular}{@{}cccc@{}} % 4 columns
    P. & \verbroot{gam} ~~单数  & 双数  & 复数 \\
    1st & \cellcolor{light-gray}\fullpada{jagāma jagama} & \fullpada{jagmiva}  & \fullpada{jagmima}  \\
    2nd & \cellcolor{light-gray}\fullpada{jagamitha jagantha} & \fullpada{jagmathuḥ}  & \fullpada{jagma}  \\
    3rd & \cellcolor{light-gray}\fullpada{jagāma} & \fullpada{jagmatuḥ}  & \fullpada{jagmuḥ}  \\
  \end{tabular}
\end{frame}

\begin{frame}{\insertsubsection ~~不规则}
  \small
  \begin{itemize}
    \item a加单辅音结尾的词根强语干, \\
    第一人称可三合,第三人称必三合。\\
    \verbroot{nī} 1st.Sg. \fullpada{ninaya nināya} 3rd.Sg. \fullpada{nināya}\\
    \item 辅音重复自己,二合是a的词根,\\ 
    弱语干不重复,而是a变e。\\
    \verbroot{nam} 3rd.Pl. \fullpada{nemuḥ} 3rd.Sg. \fullpada{nanāma}\\
    \verbroot{sad} 3rd.Pl. \fullpada{seduḥ} 3rd.Sg. \fullpada{sasāda}\\
  \end{itemize}
  \centering 
  \begin{tabular}{@{}cccc@{}} % 4 columns
    P. & \verbroot{pat} ~~单数  & 双数  & 复数 \\
    1st & \cellcolor{light-gray}\fullpada{papāta papata} & \fullpada{petiva}  & \fullpada{petima paptima}  \\
    2nd & \cellcolor{light-gray}\fullpada{petitha papattha} & \fullpada{petathuḥ}  & \fullpada{peta}  \\
    3rd & \cellcolor{light-gray}\fullpada{papāta} & \fullpada{petatuḥ}  & \fullpada{petuḥ paptuḥ}  \\  
  \end{tabular}
\end{frame}

\begin{frame}{\insertsubsection ~~不规则}
  \small
  \begin{itemize}
    \item 以ā结尾的词根,
    弱语干ā消失。 \\
  \end{itemize}
  \centering 
  \begin{tabular}{@{}cccc@{}} % 4 columns
    P. & \verbroot{dā} ~~单数  & 双数  & 复数 \\
    1st & \cellcolor{light-gray}\fullpada{dadau} & \fullpada{dadiva}  & \fullpada{dadima}  \\
    2nd & \cellcolor{light-gray}\fullpada{dadātha daditha} & \fullpada{dadathuḥ}  & \fullpada{dada}  \\
    3rd & \cellcolor{light-gray}\fullpada{dadau} & \fullpada{dadatuḥ}  & \fullpada{daduḥ}  \\  
  \end{tabular}
  \begin{itemize}
    \item 有些词弱语干也二合。\\
    \verbroot{tyaj}  P.1st.Sg. \fullpada{tatyāja tatyaja} \\~~~~~~~~3rd.Pl. \fullpada{tatyajuḥ} \\ 
    \verbroot{smṛ}  P.1st.Sg. \fullpada{sasmāra sasmara} \\~~~~~~~~3rd.Sg. \fullpada{sasmaruḥ} \\ 
  \end{itemize}
  \centering 
  
\end{frame}

\begin{frame}{语干~~其他不规则}
  \footnotesize
    \begin{itemize}
      \item \verbroot{bhū} 语干全都用 \verbstem{babhūv}。\\
    \end{itemize}
    \centering
    \begin{tabular}{@{}cccc@{}} % 4 columns
      P. & 单数  & 双数  & 复数 \\
      1st & \cellcolor{light-gray}\fullpada{babhūva} & \fullpada{babhūviva}  & \fullpada{babhūvima}  \\
      2nd & \cellcolor{light-gray}\fullpada{babhūvitha} & \fullpada{babhūvathuḥ}  & \fullpada{babhūva}  \\
      3rd & \cellcolor{light-gray}\fullpada{babhūva} & \fullpada{babhūvatuḥ}  & \fullpada{babhūvuḥ}  \\  
    \end{tabular}
    \begin{itemize}
      \item \verbroot{as} 语干全都用 \verbstem{ās}。\\   
      ~~P.1st.Sg. \fullpada{āsa} 1st.Pl. \fullpada{āsima} \\ 
      \item \verbroot{vid} 所有语干都不重复\notsoimportant{,还可表示现在}。 \\
      ~~P.1st.Sg. \fullpada{veda} 3rd.Pl. \fullpada{viduḥ}  \\  
      \item \verbroot{han} 所有h都还原成gh。 \\
      ~~P.1st.Sg. \fullpada{jaghana} 3rd.Pl. \fullpada{jaghnuḥ}  \\ 
      ~~现在时 P.3rd.Pl. \fullpada{ghnanti}\\ 
    \end{itemize}
\end{frame}

\subsection{完成时语尾}
\begin{frame}{\insertsubsection }
  \centering 
  \begin{tabular}{@{}cccc@{}} % 4 columns
    P. & 单数  & 双数  & 复数 \\
    1st & \wordending{a} & \wordending{(i)-va}  & \wordending{(i)-ma}  \\
    2nd & \wordending{(i)-tha} & \wordending{athuḥ}  & \wordending{a}  \\
    3rd & \wordending{a} & \wordending{atuḥ}  & \wordending{uḥ}  \\ 
    \\
    Ā. & 单数  & 双数  & 复数 \\
    1st & \wordending{e} & \wordending{(i)-vahe}  & \wordending{(i)-mahe}  \\
    2nd & \wordending{se/-(i)-ṣe} & \wordending{āthe}  & \wordending{(i)-dhve}  \\
    3rd & \wordending{e} & \wordending{āte}  & \wordending{ire}  \\ 
  \end{tabular}
\end{frame}

\begin{frame}{\insertsubsection ~~不规则}
  \small
  \begin{itemize}
    \item 以ā结尾的词根,\\
    主动语态一三人称单数是 \wordending{au}。 \\
  \end{itemize}
  \centering 
  \begin{tabular}{@{}cccc@{}} % 4 columns
    P. & \verbroot{sthā} ~~单数  & 双数  & 复数 \\
    1st & \cellcolor{light-gray}\fullpada{tasthau} & \fullpada{tasthiva}  & \fullpada{tasthima}  \\
    2nd & \cellcolor{light-gray}\fullpada{tasthitha tasthātha} & \fullpada{tasthathuḥ}  & \fullpada{tastha}  \\
    3rd & \cellcolor{light-gray}\fullpada{tasthau} & \fullpada{tasthatuḥ}  & \fullpada{tasthuḥ}  \\  
    \\
    Ā. & 单数  & 双数  & 复数 \\
    1st & \fullpada{tasthe} & \fullpada{tasthivahe}  & \fullpada{tasthimahe}  \\
    2nd & \fullpada{tasthiṣe} & \fullpada{tasthāthe}  & \fullpada{tasthidhve}  \\
    3rd & \fullpada{tasthe} & \fullpada{tasthāte}  & \fullpada{tasthire}  \\  
  \end{tabular}
\end{frame}

\begin{frame}{词根和语尾间的词内连声}
  \footnotesize
  %\resizebox{\textwidth}{!}{
    \begin{itemize}
      \item 非 ā 音结尾的词根\\
      在元音语尾和联系元音i前要有半元音。\\
      \item u/ū永远变uv。\\   
      \verbroot{śru}  ~~P.1st.Pl. \fullpada{śuśruma} 3rd.Pl. \fullpada{śuśruvuḥ} \\ 
      \verbroot{hu}  ~~P.1st.Pl. \fullpada{juhuvima} 3rd.Sg. \fullpada{juhuvuḥ} \\ 
      \verbroot{dhū}  ~~P.1st.Pl. \fullpada{dudhuvima} 3rd.Sg. \fullpada{dudhuvuḥ} \\ 
      \item 单辅音后i/ī变y,复合辅音后变iy。\\
      \verbroot{nī} ~~P.1st.Pl. \fullpada{ninyima} 3rd.Pl. \fullpada{ninyuḥ}  \\ 
      \verbroot{krī} ~~P.1st.Pl. \fullpada{cikriyima} 3rd.Pl. \fullpada{cikriyuḥ} \\   
      \verbroot{śri} ~~P.1st.Pl. \fullpada{śiśriyima} 3rd.Pl. \fullpada{śiśriyuḥ}  \\  
      \item ṛ变r。\\
      \verbroot{kṛ} ~~P.1st.Pl. \fullpada{cakṛma} 3rd.Pl. \fullpada{cakruḥ}  \\  
    \end{itemize}
  %}
\end{frame}

\section{本节作业}

\begin{frame}{\insertsection }
  \begin{itemize}
    \item
      第28章练习4
  \end{itemize}
\end{frame}  

\end{document}	