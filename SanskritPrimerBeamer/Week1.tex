%%% XeLaTeX-article %%%
%# -*- coding: utf-8 -*-
%!TEX encoding = UTF-8 Unicode
%!TEX TS-program = xelatex  
%---------------------虽然加了%还是要保留!

\documentclass[17pt]{beamer}
\mode<presentation>
{
\usetheme[width=40pt]{Hannover}
\usecolortheme[]{dove}
\usefonttheme[]{structurebold}
\setbeameroption{hide notes}
}

\usepackage{fontspec}
\setmainfont{Arial} %设置主字体
\newfontfamily\sanskritfont[Script=Devanagari,Mapping=romantodevanagari,Scale=1.15]{Sanskrit 2003}             %输出天城体
%\newfontfamily\sanskritfont[Mapping=tex-text]{Times New Roman}              %输出转写
\doublehyphendemerits=-10000
\newcommand{\skt}[1]{{\sanskritfont{#1}}} %输出天城体
\newcommand{\skttrans}[1]{{\skt{#1}~#1}}  %输出天城体和转写
%----------------------------------------------------设置梵文输入方法

\usepackage[UTF8,fontset=windows]{ctex}
%\setCJKmainfont{SimHei}
%----------------------------------------------------设置中文字体:默认黑体,可用
%\usepackage{xeCJK}
%\setCJKmainfont{Source Han Sans SC VF}
%----------------------------------------------------设置中文字体:思源黑体粗体

\usepackage{graphicx}
\usepackage{flafter} 
\graphicspath{{pic/}}
\usepackage{booktabs} 
%-----------------------------------------插图表格

\usepackage{hyperref} 
%------------------------------自定义命令


\title[课程介绍与字母]{{梵语入门}}
\subtitle{1. 课程介绍与字母}
\author[张雪杉]{文学院~~张雪杉 \\ \texttt{zhangxueshan@sdnu.edu.cn}}
\date{}
%\institute{}



\begin{document}	


\begin{frame}
  \titlepage
\end{frame}

\begin{frame}
  \frametitle{本节内容}
  \tableofcontents
\end{frame}

\section{课程介绍}
\begin{frame}{\insertsection }
    \tableofcontents[currentsection]
\end{frame}

\subsection{课程平台}
\begin{frame}{\insertsubsection }
  \begin{itemize}
    \item
      课程网站:超星学习通\\ =信息门户慧课慧学
    \item
      课程号:
  \end{itemize}
\end{frame}

\subsection{教材与参考资料}
\begin{frame}{\insertsubsection }
  \begin{itemize}
    \item
      教材: The Cambridge Introduction to Sanskrit
    \item
      中文参考书:梵文基础读本
    \item
      资料:课件、教材勘误表、习题参考答案等,共享文件夹查看更多
  \end{itemize}
\end{frame}

\subsection{考核方式}
\begin{frame}{\insertsubsection }
  \begin{itemize}
    \item
      平时20:课后问卷\\ 上课需轮流做作业题~~但不好算分
    \item
      期中30:默写字母表\\第八周随堂测试~~现场提交
    \item
      期末50:教材第16课练习1\\学习通作业~~截止时间xxx
  \end{itemize}
\end{frame}

\section{梵语字母}
\begin{frame}{\insertsection }
    \tableofcontents[currentsection]
\end{frame}

\subsection{元音字母}

\begin{frame}{\insertsubsection }
  %\small
    \begin{tabular}{@{}llllllll@{}} % 7 columns: type, length/type, and 5 vowels
    %\toprule
    %\multicolumn{2}{c}{Item} \\ \cmidrule(r){1-2}
    %类别 & 长短/类型 &  & 1 & 2 & 3 \\
    %\midrule
    简单 & 短  & \skttrans{a} & \skttrans{i} & \skttrans{u} & \skttrans{ṛ} & \skttrans{ḷ}   \\
    元音  & 长  & \skttrans{ā} & \skttrans{ī} & \skttrans{ū} & \skttrans{ṝ} &  \\
    %  & 短  & & \skttrans{ṛ} & \skttrans{ḷ}  \\
    %  & 长  & & \skttrans{ṝ} &  \\
    复合 &    & & \skttrans{e} & \skttrans{o} & & \\
    元音  &   &  & \skttrans{ai} & \skttrans{au} & & \\
    %\bottomrule
    \end{tabular}
\end{frame}

\subsection{辅音}

\begin{frame}{\insertsubsection }
  \small
    \begin{tabular}{@{}llllll@{}} % 6 columns: type, length/type, and 4 vowels
    %\toprule
    %\multicolumn{2}{c}{Item} \\ \cmidrule(r){1-2}
    %类别 & 长短/类型 &  & 1 & 2 & 3 \\
    %\midrule
     & 不送气  & 送气 & 不送气 & 送气 &   \\
     & 清音  & 清音 & 浊音 & 浊音 & 鼻音  \\
    喉音  & \skttrans{ka}  & \skttrans{kha} & \skttrans{ga} & \skttrans{gha} & \skttrans{ṅa} \\
    腭音  & \skttrans{ca}  & \skttrans{cha} & \skttrans{ja} & \skttrans{jha} & \skttrans{ña} \\
    顶音  & \skttrans{ṭa}  & \skttrans{ṭha} & \skttrans{ḍa} & \skttrans{ḍha} & \skttrans{ṇa} \\
    齿音  & \skttrans{ta}  & \skttrans{tha} & \skttrans{da} & \skttrans{dha} & \skttrans{na} \\
    唇音  & \skttrans{pa}  & \skttrans{pha} & \skttrans{ba} & \skttrans{bha} & \skttrans{ma} \\
      &  & &  & &   \\
    半元音  & \skttrans{ya}  & \skttrans{ra} & \skttrans{la} & \skttrans{va} &  \\
    咝音  & \skttrans{śa}  & \skttrans{ṣa} & \skttrans{sa} &  & \skttrans{ha} \\
    %\bottomrule
    \end{tabular}
\end{frame}

\begin{frame}
\end{frame} % to enforce entries in the table of contents

\end{document}	
