%%% XeLaTeX-article %%%
%# -*- coding: utf-8 -*-
%!TEX encoding = UTF-8 Unicode
%!TEX TS-program = xelatex  
%---------------------虽然加了%还是要保留!

\documentclass[17pt]{beamer}
\mode<presentation>
{
\usetheme[width=40pt]{Hannover}
\usecolortheme[]{dove}
\usefonttheme[]{structurebold}
\setbeameroption{hide notes}
}

\usepackage{fontspec}
\setmainfont{Arial} %设置主字体
\newfontfamily\sanskritfont[Script=Devanagari,Mapping=romantodevanagari,Scale=1.15]{Sanskrit 2003}             %输出天城体
%\newfontfamily\sanskritfont[Mapping=tex-text]{Times New Roman}              %输出转写
\doublehyphendemerits=-10000
\newcommand{\skt}[1]{{\sanskritfont{#1}}} %输出天城体
\newcommand{\skttrans}[1]{{\skt{#1}~#1}}  %输出天城体和转写
%----------------------------------------------------设置梵文输入方法 danda । ॥

\usepackage[UTF8,fontset=windows]{ctex}
\usepackage{amsmath}
%----------------------------------------------------设置中文环境

\usepackage{graphicx}
\usepackage{flafter} 
\graphicspath{{pic/}}
\usepackage{booktabs} 
\usepackage{nicematrix}
\newenvironment{indentlist}
  {\begin{list}{}{\setlength{\leftmargin}{2em}\setlength{\itemsep}{0.5em}}}
  {\end{list}}
%-----------------------------------------插图表格

\usepackage{hyperref} 
\usepackage[dvipsnames]{xcolor}
\usepackage{colortbl}
\definecolor{light-gray}{gray}{0.9}
%------------------------------颜色

\newcommand{\verbroot}[1]{\textcolor{red}{$\sqrt{}$#1}}
\newcommand{\sktroot}[1]{{\verbroot{\skt{#1}}}}
\newcommand{\skttransroot}[1]{{\sktroot{#1}~\textcolor{red}{#1}}}

\newcommand{\nounstem}[1]{\textcolor{red}{#1\nobreakdash-}}
\newcommand{\sktnounstem}[1]{{\textcolor{red}{\skt{#1\nobreakdash-}}}}
\newcommand{\skttransnounstem}[1]{{\sktnounstem{#1}~\nounstem{#1}}}

\newcommand{\verbstem}[1]{\textcolor{blue}{#1\nobreakdash-}}
\newcommand{\sktverbstem}[1]{{\textcolor{blue}{\skt{#1\nobreakdash-}}}}
\newcommand{\skttransverbstem}[1]{{\sktverbstem{#1}~\verbstem{#1}}}

\newcommand{\wordending}[1]{\textcolor{Orange}{\nobreakdash-#1}}
\newcommand{\sktending}[1]{{\textcolor{Orange}{\skt{-#1}}}}
\newcommand{\skttransending}[1]{{\sktending{#1}~\wordending{#1}}}

\newcommand{\fullpada}[1]{\textcolor{OliveGreen}{#1}}
\newcommand{\sktpada}[1]{{\textcolor{OliveGreen}{\skt{#1}}}}
\newcommand{\skttranspada}[1]{{\sktpada{#1}~\fullpada{#1}}}

\newcommand{\pratyaya}[1]{\textcolor{Plum}{#1}}
\newcommand{\sktpratyaya}[1]{{\textcolor{Plum}{\skt{#1}}}}
\newcommand{\skttranspratyaya}[1]{{\sktpratyaya{#1}~\pratyaya{#1}}}

\newcommand{\reconstruction}[1]{\textcolor{gray}{*#1}}
\newcommand{\fullsentence}[1]{\textcolor{MidnightBlue}{#1}}

\newcommand{\veryimportant}[1]{\textcolor{red}{#1}}
\newcommand{\important}[1]{\textcolor{blue}{#1}}
\newcommand{\notsoimportant}[1]{\textcolor{gray}{#1}}
%-------------------------------------------词根等标颜色

\title{{梵语入门}}
\subtitle{10. 动词前缀}
\author[张雪杉]{文学院~~张雪杉 \\ zhangxueshan@sdnu.edu.cn}
\date{}
%\institute{}

\begin{document}	

\begin{frame}
  \titlepage
\end{frame}

\begin{frame}
  \frametitle{本节内容}
  \small
  \tableofcontents
\end{frame}

\section{上节作业}

\begin{frame}{第九章练习4}
  \small
  \raggedright
  \begin{verse}
    \mbox{\skt{1) bhāryāṃ bālāḥ ca dṛṣṭvā naraḥ tuṣṭaḥ pure gacchati ।}}  \\
    \mbox{\skt{2) pūjā amarebhyaḥ iti uktvā janāḥ namanti ।}}   \\
    \skt{3) yuddheṣu pṛtanāsu ca cintāḥ bhavanti ।}   \\
    \skt{4) amarāṇāṃ kathāḥ śrutvā narau kṣetre sthitaṃ vṛkṣaṃ prati gatvā sīdataḥ ।}  \\
    \skt{5) prajñā jarāyām iti prajā vadati ।}  \\
  \end{verse}
\end{frame}  

\begin{frame}{第九章练习4}
  \small
  \raggedright
  \begin{verse}
    \skt{6) īśvaraḥ \veryimportant{prajānāṃ} pālaḥ ।}   \\
    \skt{7) kanyāyāḥ prabhāṃ dṛṣṭvā kumāraḥ cintāḥ vismarati ।}  \\
    \skt{8) api śūraiḥ jitānām ugrāṇāṃ kathāṃ śrutvā tān śūrān pūjayasi ।}   \\
    \skt{9) bālā vṛddhā iti naraḥ cintayati ।}   \\
    \skt{10) prājñaḥ devaḥ iva sukhaṃ jīvati ।}   \\
  \end{verse}
\end{frame}  


\subsection{复习}

\begin{frame}{以ā结尾的名词}
  \small
  \centering
  \resizebox{\textheight}{!}{
    \begin{tabular}{@{}llllll@{}} % 6 columns
       & 单数 & 双数 & 复数  \\
      主 & \fullpada{senā}  & \fullpada{sene} & \fullpada{senāḥ}   \\
      呼 & \fullpada{sene} & \fullpada{sene} & \fullpada{senāḥ} \\
      业 & \fullpada{senām} & \fullpada{sene} & \fullpada{senāḥ} \\
      具 & \fullpada{senayā} & \fullpada{senābhyām} & \fullpada{senābhiḥ} \\
      为 & \fullpada{senāyai} & \fullpada{senābhyām} & \fullpada{senābhyaḥ} \\
      从 & \fullpada{senāyāḥ} & \fullpada{senābhyām} & \fullpada{senābhyaḥ} \\
      属 & \fullpada{senāyāḥ} & \fullpada{senayoḥ} & \fullpada{senānām} \\
      依 & \fullpada{senāyāṃ} & \fullpada{senayoḥ} & \fullpada{senāsu} \\
    \end{tabular}
  }
\end{frame}


\begin{frame}{ruki规则}
  \small
  \begin{itemize}
    \item 前音是非a/ā元音、半元音、h或者喉音时,s变成ṣ\\
    \nounstem{deva} + \pratyaya{\nobreakdash-su} = \fullpada{deveṣu}\\
    反例:a 后不变 \\ \nounstem{senā} + \pratyaya{\nobreakdash-su} = \fullpada{senāsu}
    \item 前音和s中间插入了ṃ、ḥ或者咝音仍要变化\\
    \nounstem{yajuḥ} + \pratyaya{\nobreakdash-su} = \fullpada{yajuḥṣu}
    \item 词尾的s不变 
    ~~\fullsentence{agnis tatra} 
  \end{itemize}
\end{frame}

\section{动词前缀}
\begin{frame}{\insertsection }
  \small
  \tableofcontents[currentsection]
\end{frame}

\begin{frame}{前缀 prepositions}
  %\small
  \begin{itemize}
    \item 介词在与名词连用时往往后置 \\
    \fullpada{prati} + 业格 \fullsentence{nagaraṃ prati} \\
    \fullpada{saha} + 具格 \fullsentence{mitraiḥ saha} \\
    \fullpada{vinā} + 具/业/从格 \fullsentence{bhayena vinā} \\
    \item 和动词连用时前置 \\   
    \fullsentence{nagaraṃ pratigacchati} \\
  \end{itemize}
\end{frame}

\subsection{常见前缀}
\begin{frame}{\insertsubsection }
  \small
  \centering
    \begin{tabular}{@{}rlrl@{}} % 6 columns
      \pratyaya{ati\nobreakdash-} & 从上面,从旁边 & \pratyaya{ud\nobreakdash-} & 在上面,来自  \\
      \pratyaya{adhi\nobreakdash-} & 从上面,在上面  & \pratyaya{upa\nobreakdash-} & 向着   \\
      \pratyaya{anu\nobreakdash-} & 在后面,沿着 & \pratyaya{ni\nobreakdash-} & 向下面,向里面 \\
      \pratyaya{antar\nobreakdash-} & 在中间 & \pratyaya{nis\nobreakdash-} & 出来 \\
      \pratyaya{apa\nobreakdash-} & 离开 & \pratyaya{parā\nobreakdash-} & 离开 \\
      \pratyaya{api\nobreakdash-} & 朝……而去 & \pratyaya{pari\nobreakdash-} & 围绕着 \\
      \pratyaya{abhi\nobreakdash-} & 朝着 & \pratyaya{pra\nobreakdash-} & 向前 \\
      \pratyaya{ava\nobreakdash-} & 从……往下 & \pratyaya{prati\nobreakdash-} & 阻挡,回来 \\
      \pratyaya{ā\nobreakdash-} & 向着,来 & \pratyaya{vi\nobreakdash-} & 离开,散开 \\
       &  & \pratyaya{sam\nobreakdash-} & 同,一起 \\
    \end{tabular}
\end{frame}

\subsection{形式}
\begin{frame}{前缀的形式}
  %\small
  \begin{itemize}
    \item 前缀加在动词前 \\
    \pratyaya{apa\nobreakdash-}\verbroot{gam}   \fullpada{apagacchati} 离开 \\
    ~~~\pratyaya{ā\nobreakdash-}\verbroot{gam}  \fullpada{āgacchati} 来 \\
    \pratyaya{prati\nobreakdash-}  \fullsentence{nagaraṃ pratigacchati} 回城 \\

    \item 前缀可以加好几个 \\   
    \pratyaya{ā\nobreakdash-}\verbroot{gam} 来  \\
    \pratyaya{pari\nobreakdash-ā\nobreakdash-}\verbroot{gam} 达到 \\
    \pratyaya{pari\nobreakdash-upa\nobreakdash-ā\nobreakdash-}\verbroot{gam} 包围 \\
  \end{itemize}
\end{frame}

\begin{frame}{前缀的形式}
  %\small
  \begin{itemize}
    \item 前缀和动词之间发生词内连声 \\
    \pratyaya{pra\nobreakdash-}\verbroot{nam}   \fullpada{praṇamati} 鞠躬 \\
    \pratyaya{upa\nobreakdash-ni\nobreakdash-}\verbroot{sad}  \fullpada{upaniṣīdati} 坐过来 \\
    \pratyaya{ud\nobreakdash-}\verbroot{sthā}  \fullpada{uttiṣṭhati} 站起 \\
    \item 加前缀的动词独立式加\pratyaya{\nobreakdash-tya}或\pratyaya{\nobreakdash-ya} \\   
    短元音加\pratyaya{\nobreakdash-tya}:\verbroot{dru} \fullpada{drutvā} 跑 \\
    ~~~\pratyaya{apa\nobreakdash-}\verbroot{dru} \fullpada{apadrutya} 跑开 \\
    其他加\pratyaya{\nobreakdash-ya}:\verbroot{nī} \fullpada{nītvā} 带领 \\
    ~~~\pratyaya{apa\nobreakdash-}\verbroot{nī} \fullpada{apanīya} 领走 \\
  \end{itemize}
\end{frame}

\subsection{意义}
\begin{frame}{前缀的意义}
  %\small
  \begin{itemize}
    \item 前缀意义不确定 \\
    \item 动词意义少则前缀意义多 \\
      \verbroot{gam} 走 ~~ \pratyaya{anu\nobreakdash-}\verbroot{gam} 跟随 \\
      \pratyaya{apa\nobreakdash-}\verbroot{gam} 离开 ~~ \pratyaya{sam\nobreakdash-}\verbroot{gam} 聚集  
    \item 动词意义多则前缀意义少 \\
      \verbroot{muc} 释放 ~~ \\
      \pratyaya{ava\nobreakdash-}/\pratyaya{ud\nobreakdash-}/\pratyaya{nir\nobreakdash-}/\pratyaya{pari\nobreakdash-}/\pratyaya{pra\nobreakdash-}\verbroot{muc} 释放 \\
      \pratyaya{ava\nobreakdash-}\verbroot{muc} 脱衣服 ~~ \pratyaya{pra\nobreakdash-}\verbroot{muc} 掉下
  \end{itemize}
\end{frame}

\begin{frame}{前缀的意义}
  %\small
  \begin{itemize}
    \item 最终还要查字典 \\
    \pratyaya{pari\nobreakdash-upa\nobreakdash-ā\nobreakdash-}\verbroot{gam} 包围 \\
    \pratyaya{ava\nobreakdash-}\verbroot{gam} 理解 \\ 
    \item 查字典时要查词根 \\
      \mbox{\fullpada{saṃgacchati} \verbroot{gam}\nobreakdash-\pratyaya{sam\nobreakdash-} / \verbroot{saṃgam}}\\
      \fullpada{praṇamanti} \verbroot{nam}\nobreakdash-\pratyaya{pra\nobreakdash-} / \verbroot{praṇam}\\  
      \fullpada{upaniṣīdati} \verbroot{sad}\nobreakdash-\pratyaya{upa\nobreakdash-ni\nobreakdash-} / \verbroot{upaniṣad}\\ 
  \end{itemize}
\end{frame}


\section{词内连声}
\begin{frame}{\insertsection }
  \small
  \tableofcontents[currentsection]
\end{frame}

\subsection{元音连声}

\begin{frame}{\insertsubsection }
 % \small
  \begin{itemize}
    \item 简单元音同类融合为长音\\
    \pratyaya{upa\nobreakdash-ā\nobreakdash-}\verbroot{nī} \fullpada{upānayati}  \\
    \item i/u加不同类时变成对应半元音\\
    \pratyaya{vi\nobreakdash-apa\nobreakdash-}\verbroot{nī} \fullpada{vyapanayati}  \\
    \pratyaya{adhi\nobreakdash-ā\nobreakdash-}\verbroot{gam} \fullpada{adhyāgacchati}  \\
    \item a/ā加不同类时升一级\\ 
    \pratyaya{pra\nobreakdash-ud\nobreakdash-}\verbroot{dhṛ} \fullpada{proddharati}  \\
  \end{itemize}
\end{frame}

\subsection{m的连声}

\begin{frame}{\insertsubsection }
 % \small
  \begin{itemize}
    \item 动词前缀的m在元音前不变\\
    \pratyaya{sam\nobreakdash-ā\nobreakdash-}\verbroot{gam} \fullpada{samāgacchati}  \\
    \item 在辅音前变成ṃ\\
    \pratyaya{sam\nobreakdash-}\verbroot{gam} \fullpada{saṃgacchati}  \\
  \end{itemize}
\end{frame}

\section{本节作业}

\begin{frame}{\insertsection }
  \begin{itemize}
    \item
      第十章练习2
    \item
      阅读教材第10课相关内容
    \bigskip
    \item
      现在请做学习通\nobreakdash-章节\nobreakdash-课后问卷
  \end{itemize}
\end{frame}  

\end{document}	