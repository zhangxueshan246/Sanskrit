%%% XeLaTeX-article %%%
%# -*- coding: utf-8 -*-
%!TEX encoding = UTF-8 Unicode
%!TEX TS-program = xelatex  
%---------------------虽然加了%还是要保留!

\documentclass[17pt]{beamer}
\mode<presentation>
{
\usetheme[width=40pt]{Hannover}
\usecolortheme[]{dove}
\usefonttheme[]{structurebold}
\setbeameroption{hide notes}
}

\usepackage{fontspec}
\setmainfont{Arial} %设置主字体
\newfontfamily\sanskritfont[Script=Devanagari,Mapping=romantodevanagari,Scale=1.15]{Sanskrit 2003}             %输出天城体
%\newfontfamily\sanskritfont[Mapping=tex-text]{Times New Roman}              %输出转写
\doublehyphendemerits=-10000
\newcommand{\skt}[1]{{\sanskritfont{#1}}} %输出天城体
\newcommand{\skttrans}[1]{{\skt{#1}~#1}}  %输出天城体和转写
%----------------------------------------------------设置梵文输入方法 danda । ॥

\usepackage[UTF8,fontset=windows]{ctex}
\usepackage{amsmath}
%----------------------------------------------------设置中文环境

\usepackage{graphicx}
\usepackage{flafter} 
\graphicspath{{pic/}}
\usepackage{booktabs} 
%-----------------------------------------插图表格

\usepackage{hyperref} 
\usepackage{xcolor}
%------------------------------颜色

\newcommand{\verbroot}[1]{{$\sqrt{#1}$}}
\newcommand{\sktroot}[1]{{\verbroot{}\skt{#1}}}
\newcommand{\skttransroot}[1]{{\sktroot{#1}~#1}}
%---------------------------------------------------------------词根

\title{{梵语入门}}
\subtitle{4. 现在时}
\author[张雪杉]{文学院~~张雪杉 \\ zhangxueshan@sdnu.edu.cn}
\date{}
%\institute{}



\begin{document}	


\begin{frame}
  \titlepage
\end{frame}

\begin{frame}
  \frametitle{本节内容}
  \tableofcontents
\end{frame}

\section{上节作业}

\begin{frame}{第二章练习2}
  \small
  \centering
  \begin{tabular}{@{}llllll@{}} % 6 columns
    a) & \skt{gacchati}  & \skt{tataḥ}  & \skt{gṛham} & \skt{kṣatriyaḥ}   \\
    & \skt{jñānam} & \skt{snihyati}  & \skt{yuddham}  & \skt{tiṣṭhati}  \\
    & \skt{pṛcchati} & &   &   \\
    b)& \skt{tuṣyati}  & \skt{dravati}  & \skt{rājñī} & \skt{kṛtsnam} \\
    & \skt{candraḥ} & \skt{vidyut}  & \skt{kṛtvā}  & \skt{suhṛt} \\
    & \skt{kṣudra} & \skt{adya} &  &  \\
    c) & \skt{krodhaḥ}  & \skt{aśvaḥ}  & \skt{śrī} & \skt{śaknoti}  \\
    & \skt{vṛṇoti} & \skt{dveṣṭi}  & \skt{gṛhṇāti}  & \skt{viśva}  \\
    & \skt{anyonya} & \skt{śrutvā} &  &   \\
  \end{tabular}
\end{frame}            

\begin{frame}{第二章练习2}
  \small
  \centering
  \begin{tabular}{@{}llllll@{}} % 6 columns
    d) & \skt{bandhuḥ}  & \skt{buddhiḥ}  & \skt{śatruḥ} & \skt{dṛṣṭiḥ} \\
    & \skt{aśru} & \skt{kīrtiḥ}  & \skt{snānam}  & \skt{antarikṣagaḥ}  \\
    & \skt{hṛcchayaḥ} & \skt{pṛthivī} &   &   \\
  \end{tabular}
  \bigskip
  
  e) \skt{jitendriya anantaratnaprabhava} 

  \bigskip

  \skt{balahākacchedavibhaktarāgā lāṅgūlavikṣepavisarpiśobha} 
\end{frame} 

\subsection{数字与标点}

\begin{frame}{第三章练习1}
  \small
  \centering
  \begin{tabular}{@{}ll@{}} % 6 columns
    \skt{1) gacchati viśati ca ।} & \skt{7) api gacchati viśati vā ।} \\
    \skt{2) kim icchati ।} & \skt{8) na smarati na ca bodhati ।}  \\
    \skt{3) punar likhati ।} & \skt{9) na smarati na bodhati ca ।}  \\
    \skt{4) atra tatra ca } & \skt{10) api na bodhati ।}  \\
    \skt{5) na gacchati ।} & \skt{11) kim harati ।}  \\
    \skt{6) smarati likhati ca ।} &  \\
  \end{tabular}
  \bigskip

  数字与标点: \skt{1 2 3 4 5 6 7 8 9 0 ~~~~। ॥} 
\end{frame} 



\section[动词 \\ ~~现在时]{动词现在时}
\begin{frame}{\insertsection }
    \tableofcontents[currentsection]
\end{frame}

\subsection{现在时语尾}
\begin{frame}{\insertsubsection ~~主动语态}
  \small
  \centering
  \begin{tabular}{@{}llll@{}} % 4 columns
    & 单数  & 双数  & 复数 \\
    第一人称 & \skttrans{-mi} & \skttrans{-vaḥ}  & \skttrans{-maḥ}  \\
    第二人称 & \skttrans{-si} & \skttrans{-thaḥ} & \skttrans{-tha}   \\
    第三人称 & \skttrans{-ti} & \skttrans{-taḥ} & \skttrans{-nti}  \\
  \end{tabular}   
\end{frame}

\subsection{现在时语干}
\begin{frame}{\insertsubsection ~~动词分类}
  \begin{itemize}
    \item
      带插入元音a的:
      
      第一、四、六、\textcolor{blue}{十}类动词
    \item
      不带插入元音的:
      
      第二、三、五、七、八、九类动词
    \item
      使用现在时语干的时态:
      
      现在时、未完成时、命令语气和祈愿语气
  \end{itemize}  
\end{frame}

\begin{frame}{\insertsubsection ~~第一类动词}
  \small
  \begin{itemize}
    \item
      词根元音前加\nobreakdash-a\nobreakdash-,词根后加\nobreakdash-a\nobreakdash-。
  \end{itemize} 

  \centering
  \begin{tabular}{@{}llllll@{}} % 6 columns
    词根 & 过程 & 语干  \\
    \skttransroot{bhṛ} & \textcolor{gray}{*bh-a-ṛ-a-} & \skttrans{bhara-}  \\
    \skttransroot{budh} & \textcolor{gray}{*b-a-udh-a-} & \skttrans{bodha-}  \\
  \end{tabular}

\end{frame}

\begin{frame}{\insertsubsection ~~第四类动词}
  \small
  \begin{itemize}
    \item
      词根不变,后加\nobreakdash-ya\nobreakdash-。
  \end{itemize} 

  \centering
  \begin{tabular}{@{}llllll@{}} % 6 columns
    词根 & 过程 & 语干  \\
    \skttransroot{hṛṣ} & \textcolor{gray}{*hṛṣ-ya-} & \skttrans{hṛṣya-}  \\
  \end{tabular}

\end{frame}

\begin{frame}{\insertsubsection ~~第六类动词}
  \small
  \begin{enumerate}
    \item
      词根不变,后加\nobreakdash-a\nobreakdash-。

      \begin{tabular}{@{}llllll@{}} % 6 columns
        词根 & 过程 & 语干  \\
        \skttransroot{viś} & \textcolor{gray}{*viś-a-} & \skttrans{viśa-}  \\
      \end{tabular}
   
    \item
      词根末尾辅音前加鼻音,后加\nobreakdash-a\nobreakdash-。
  \end{enumerate} 

  \centering
  \begin{tabular}{@{}llllll@{}} % 6 columns
    %词根 & 过程 & 语干  \\
    \skttransroot{muc} & \textcolor{gray}{*mu-ñ-c-a-} & \skttrans{muñca-}  \\
    \skttransroot{vid} & \textcolor{gray}{*vi-n-d-a-} & \skttrans{vinda-}  \\
    \skttransroot{lup} & \textcolor{gray}{*lu-m-p-a-} & \skttrans{lumpa-}  \\
  \end{tabular}

\end{frame}

\subsection{完整形式}

\begin{frame}{\insertsubsection }
  \footnotesize
  词根 ~\skttransroot{bhṛ} ~~现在时语干 ~ {\skt{bhara-}} bhara\nobreakdash-

  \centering
  \textcolor{gray}{
    \begin{tabular}{@{}llll@{}} % 4 columns
      %\multicolumn{2}{l}{词根 ~\skttransroot{bhṛ}} & \multicolumn{2}{l}{现在时语干~\skttrans{bhara-}} \\
      & 单数  & 双数  & 复数 \\
      第一人称 & \skttrans{-mi} & \skttrans{-vaḥ}  & \skttrans{-maḥ}  \\
      第二人称 & \skttrans{-si} & \skttrans{-thaḥ} & \skttrans{-tha}   \\
      第三人称 & \skttrans{-ti} & \skttrans{-taḥ} & \skttrans{-nti}  \\
    \end{tabular}
  }
  \bigskip
  
  \resizebox{\textwidth}{!}{
    \begin{tabular}{@{}llll@{}} % 4 columns
      & 单  & 双  & 复 \\
      1st & \skt{bharāmi}~{bhar\textcolor{red}{ā}mi} & \skt{bharāvaḥ}~{bhar\textcolor{red}{ā}vaḥ}  & \skt{bharāmaḥ}~{bhar\textcolor{red}{ā}maḥ}  \\
      2nd & \skttrans{bharasi} & \skttrans{bharathaḥ} & \skttrans{bharatha}   \\
      3rd & \skttrans{bharati} & \skttrans{bharataḥ} & \skttrans{bharanti}  \\
    \end{tabular}
  }
  
  \begin{itemize}
    \item
      第一人称语尾前a变长ā(一四六十)。
  \end{itemize}
\end{frame}


\section{连声}
\begin{frame}{\insertsection }
    \tableofcontents[currentsection]
\end{frame}

\subsection{词尾的m和s}

\begin{frame}{\insertsection  ~~\insertsubsection }
  \small
  \centering
  \begin{tabular}{@{}lllll@{}} % 7 columns: type, length/type, and 5 vowels
    原始形式 & 后词情况 & 最终 & 例 \\[0.2cm]
    词尾的m & 开头元音 & m & \skttrans{kim atra}  \\
      &  开头辅音 & ṃ & \skttrans{kiṃ tatra} \\[0.2cm]
        词尾的s & 无其他词 & ḥ & \skttrans{bharataḥ}  \\
  \end{tabular} 
\end{frame}


\section{本节作业}

\begin{frame}{\insertsection }
  \begin{itemize}
    \item
      第四章练习2,练习3
    \item
      阅读教材第4课相关内容
    \bigskip
    \item
      现在请做学习通\nobreakdash-章节\nobreakdash-课后问卷
  \end{itemize}
\end{frame}  

\end{document}	
