%%% XeLaTeX-article %%%
%# -*- coding: utf-8 -*-
%!TEX encoding = UTF-8 Unicode
%!TEX TS-program = xelatex  
%---------------------虽然加了%还是要保留!

\documentclass[17pt]{beamer}
\mode<presentation>
{
\usetheme[width=40pt]{Hannover}
\usecolortheme[]{dove}
\usefonttheme[]{structurebold}
\setbeameroption{hide notes}
}

\usepackage{fontspec}
\setmainfont{Arial} %设置主字体
\newfontfamily\sanskritfont[Script=Devanagari,Mapping=romantodevanagari,Scale=1.15]{Sanskrit 2003}             %输出天城体
%\newfontfamily\sanskritfont[Mapping=tex-text]{Times New Roman}              %输出转写
\doublehyphendemerits=-10000
\newcommand{\skt}[1]{{\sanskritfont{#1}}} %输出天城体
\newcommand{\skttrans}[1]{{\skt{#1}~#1}}  %输出天城体和转写
%----------------------------------------------------设置梵文输入方法 danda । ॥

\usepackage[UTF8,fontset=windows]{ctex}
\usepackage{amsmath}
%----------------------------------------------------设置中文环境

\usepackage{graphicx}
\usepackage{flafter} 
\graphicspath{{pic/}}
\usepackage{booktabs} 
\usepackage{nicematrix}
%-----------------------------------------插图表格

\usepackage{hyperref} 
\usepackage[dvipsnames]{xcolor}
\usepackage{colortbl}
\definecolor{light-gray}{gray}{0.9}
%------------------------------颜色

\newcommand{\verbroot}[1]{\textcolor{red}{$\sqrt{}$#1}}
\newcommand{\sktroot}[1]{{\verbroot{\skt{#1}}}}
\newcommand{\skttransroot}[1]{{\sktroot{#1}~\textcolor{red}{#1}}}

\newcommand{\nounstem}[1]{\textcolor{red}{#1\nobreakdash-}}
\newcommand{\sktnounstem}[1]{{\textcolor{red}{\skt{#1\nobreakdash-}}}}
\newcommand{\skttransnounstem}[1]{{\sktnounstem{#1}~\nounstem{#1}}}

\newcommand{\verbstem}[1]{\textcolor{blue}{#1\nobreakdash-}}
\newcommand{\sktverbstem}[1]{{\textcolor{blue}{\skt{#1\nobreakdash-}}}}
\newcommand{\skttransverbstem}[1]{{\sktverbstem{#1}~\verbstem{#1}}}

\newcommand{\wordending}[1]{\textcolor{Orange}{\nobreakdash-#1}}
\newcommand{\sktending}[1]{{\textcolor{Orange}{\skt{-#1}}}}
\newcommand{\skttransending}[1]{{\sktending{#1}~\wordending{#1}}}

\newcommand{\fullpada}[1]{\textcolor{OliveGreen}{#1}}
\newcommand{\sktpada}[1]{{\textcolor{OliveGreen}{\skt{#1}}}}
\newcommand{\skttranspada}[1]{{\sktpada{#1}~\fullpada{#1}}}

\newcommand{\pratyaya}[1]{\textcolor{Plum}{#1}}
\newcommand{\sktpratyaya}[1]{{\textcolor{Plum}{\skt{#1}}}}
\newcommand{\skttranspratyaya}[1]{{\sktpratyaya{#1}~\pratyaya{#1}}}

\newcommand{\reconstruction}[1]{\textcolor{gray}{*#1}}
\newcommand{\fullsentence}[1]{\textcolor{MidnightBlue}{#1}}

\newcommand{\veryimportant}[1]{\textcolor{red}{#1}}
\newcommand{\important}[1]{\textcolor{blue}{#1}}
\newcommand{\notsoimportant}[1]{\textcolor{gray}{#1}}
%-------------------------------------------词根等标颜色

\title{{梵语入门}}
\subtitle{8. 独立式分词等}
\author[张雪杉]{文学院~~张雪杉 \\ zhangxueshan@sdnu.edu.cn}
\date{}
%\institute{}

\begin{document}	

\begin{frame}
  \titlepage
\end{frame}

\begin{frame}
  \frametitle{本节内容}
  \small
  \tableofcontents
\end{frame}

\section{上节作业}

\begin{frame}{第七章练习7}
  \small
  \raggedright
  \begin{verse}
    \skt{1) priyaṃ gṛhaṃ tyajāmaḥ aśvān ca nagaraṃ prati nayāmaḥ ।}   \\
    \skt{2) mitra api devān vacanaiḥ ślokaiḥ ca pūjayasi ।}   \\
    \skt{3) īśvarāḥ puraṃ vardhayanti ।}   \\
    \mbox{\skt{4) bālau kim atra tiṣṭhathaḥ na ca gṛhe bhavathaḥ ।}}  \\
    \skt{5) bālaḥ mitreṇa saha gṛhāt dravati ।}   \\
  \end{verse}
\end{frame}  

\begin{frame}{第七章练习7}
  \small
  \raggedright
  \begin{verse}
    \skt{6) śūrāḥ narāḥ vyāghraṃ vanaṃ prati drāvayanti ।}   \\
    \skt{7) api kṣatriyāḥ pāpān yuddhe pātayanti ।}   \\
    \mbox{\skt{8) śūraḥ aśvaḥ vyāghrāt naraṃ rakṣati ।}} \\
    \skt{9) mitrāṇi eva smarāmi gṛhaṃ ca nayāmi ।}   \\
    \skt{10) kṣatriyāḥ api yuddhe pāpaṃ jayatha janān ca nagaraṃ prati nayatha ।}   \\
  \end{verse}
\end{frame}  
 
\subsection{复习}

\begin{frame}{元音的系统变化}
  %\small
  \centering
  %\resizebox{\textwidth}{!}{
    \begin{tabular}{@{}ccc@{}} % 6 columns
      零级 & 二合 & 三合  \\
      \fullpada{-} & \fullpada{a} & \fullpada{ā} \\
      \hline
      \fullpada{ṛ/ṝ}  & \fullpada{ar} & \fullpada{ār}   \\
      \fullpada{ḷ}  & \fullpada{al} & \fullpada{āl}   \\
      \hline
      \fullpada{i/ī} & \fullpada{ay/e} & \fullpada{āy/ai} \\
      \fullpada{u/ū} & \fullpada{av/o} & \fullpada{āv/au} \\
    \end{tabular}
  %}
\end{frame}

\begin{frame}{元音转换的应用}
  \begin{itemize}
    \item 第十类动词
    \begin{itemize}
      \item
        词根变哪级都可能,后加 \nobreakdash-aya\nobreakdash-。
      \item 名转动词:保留名词的级别
      \item 其他:能产生重音节的最低级
    \end{itemize}
    \item 致使动词
    \begin{itemize}
      \item
        其他类的动词按照第十类变化
    \end{itemize}
  \end{itemize}
\end{frame}

\section{独立式等}
\begin{frame}{\insertsection }
  \small
  \tableofcontents[currentsection]
\end{frame}

\begin{frame}{为什么要有这些形式}
  \begin{itemize}
    \item 梵语一般每句话只有一个动词。
    \item 其他的动作要用\veryimportant{非谓语形式}表达。
  \end{itemize}
  \bigskip
    {\footnotesize
    She \important{leaves} the house, \important{goes} into the forest and \important{sits} down.

    After she \important{left} the house and \important{went} into the forest, she \important{sat} down.

    \important{Having left} \textsubscript{\textbf{(独立式/分词)}}  the house and \important{having gone} \textsubscript{\textbf{(独立式/分词)}}  into the forest, 
she \important{sat} \textsubscript{\textbf{(主要动词)}} down.
    }
\end{frame}

\subsection{独立式}
\begin{frame}{\insertsubsection}
  \begin{itemize}
    \item 形式
    
    词根零级 + \pratyaya{\nobreakdash-tvā} 

    \notsoimportant{前缀 + 词根零级 + \pratyaya{\nobreakdash-tya} / \pratyaya{\nobreakdash-ya}}
    \item 意义 (having x\nobreakdash-ed)
    
    动作发生在主要动词之前 

    与主要动词主语一致
  \end{itemize}
\end{frame}

\begin{frame}{\insertsubsection  ~~例词}
  \centering
  \begin{tabular}{@{}ll@{}} % 6 columns
    词根 & 独立式  \\
    \skttransroot{dru} 跑 & \skttranspada{drutvā}  \\
    \skttransroot{kṛ} 做 & \skttranspada{kṛtvā}  \\
    \skttransroot{gam} 去 & \skttranspada{gatvā}  \\
  \end{tabular}
\end{frame}

\subsection{过去分词}
\begin{frame}{\insertsubsection }
  \begin{itemize}
    \item 形式
    
    词根零级 + \pratyaya{\nobreakdash-ta\nobreakdash-} \veryimportant{(按名词变格)}
    \item 意义
    \begin{itemize}
      \item ppp. 过去被动分词
      
      (having been x\nobreakdash-ed)
      \item 没法被动的词表主动含义
    \end{itemize}
  \end{itemize}
\end{frame}

\begin{frame}{\insertsubsection ~~例词}
  \centering
  \begin{tabular}{@{}lll@{}} % 6 columns
    & 词根 & 过去分词  \\
    被动含义 & \skttransroot{bhṛ} 扛 & \skttransnounstem{bhṛta}  \\
    & \skttransroot{kṛ} 做 & \skttransnounstem{kṛta}  \\
    主动含义 &  \skttransroot{bhū} 存在 & \skttransnounstem{bhūta} \\ 
    & \skttransroot{gam} 去 & \skttransnounstem{gata}\\
  \end{tabular}
\end{frame}

\begin{frame}{\insertsubsection ~~句法}
  \begin{itemize}
    \item 句中没有主要动词时\\过去分词可代替主要动词
    
    \fullsentence{puraṃ dṛṣṭaṃ} The city was seen. \\
    \fullsentence{narāḥ puraṃ gatāḥ} The men went to the city.
    \item 被动的逻辑主语用具格或属格
    
    \fullsentence{nareṇa / narasya aśvaḥ dṛṣṭaḥ} \\
    The horse was seen by the man.
  \end{itemize}
\end{frame}

\subsection{不定式}
\begin{frame}{\insertsubsection }
  \begin{itemize}
    \item 形式
    
    词根二合 + \pratyaya{\nobreakdash-tum} 

    \item 意义
    
    to x (to do)
    
    和表示期望/想要/去等动词连用
  \end{itemize}
\end{frame}

\begin{frame}{\insertsubsection ~~例词}
  \centering
  \begin{tabular}{@{}lll@{}} % 6 columns
    & 词根 & 不定式  \\
    & \skttransroot{nī} 带领 & \skttranspada{netum}  \\
    & \skttransroot{śru} 听 & \skttranspada{śrotum}  \\ 
  \end{tabular}
\end{frame}

\subsection{不规则形式}
\begin{frame}{\insertsubsection ~~联系元音 i}
  \begin{itemize}
    \item 某些以辅音结尾的词根\\和以 t 开头的词缀间加 i
    
    \verbroot{likh} 写 \nounstem{likhita} \fullpada{likhitvā}\\
    \verbroot{pat} 飞 \fullpada{patitum} 
    \item 第十类和致使动词一定加 i
    
    \verbroot{cint} 想 \fullpada{cintayitum}\\
    \verbroot{pat} 飞 \nounstem{pātayita} \fullpada{pātayitvā}

    \item 其他词加不加 i 没有规律

    \verbroot{bhū} 存在 \nounstem{bhūta} \fullpada{bhūtvā} \fullpada{bhavitum}
  \end{itemize}  
\end{frame}

\begin{frame}{\insertsubsection ~~词根二合}
  \begin{itemize}
    \item 某些词根二合构成独立式\\和过去分词

    \verbroot{rakṣ} 保护 \nounstem{rakṣita} \fullpada{rakṣitvā}\\
    \verbroot{pat} 飞 \nounstem{patita} \fullpada{patitvā}
    \item 哪些词这样变没有规律
  \end{itemize}  
\end{frame}

\section{词内连声}
\begin{frame}{\insertsection }
  \small
  \tableofcontents[currentsection]
\end{frame}

\subsection{t 前辅音}
\begin{frame}{\insertsubsection 连声}
  \begin{itemize}
    \item 浊音变清音
    ~~\verbroot{chid} 切 \fullpada{chittvā}

    \item 腭音变喉音

    \verbroot{muc} 放 \fullpada{muktvā} \\
    \verbroot{yuj} 联系 \fullpada{yuktvā}
    \item 鼻音变n   
    ~~\verbroot{gam} 去 \fullpada{gantum} 

    {\small
    \notsoimportant{扩展:鼻音的发音部位跟着后面的辅音变}}
    \item śt 变 ṣṭ
    ~~\verbroot{dṛś} 看 \fullpada{dṛṣṭvā}  
  \end{itemize}  
\end{frame}

\subsection{buddha 连声}
\begin{frame}{\insertsubsection}
  \begin{itemize}
    \item 前音是送气浊音时,\\前音变不送气,t变dh 
    
    \verbroot{budh} 醒 ~\nounstem{buddha} \\
    \verbroot{labh} ~得 ~\fullpada{labdhvā} \\
    \verbroot{dah} ~~烧 ~\nounstem{dagdha} \\
    ~~~\notsoimportant{h 历史上往往是 gh}
  \end{itemize}  
\end{frame}

\section{本节作业}

\begin{frame}{\insertsection }
  \begin{itemize}
    \item
      \veryimportant{下周期中考试:默写字母表}
    \item
      第八章练习4
    \item
      阅读教材第8课相关内容
    \bigskip
    \item
      现在请做学习通\nobreakdash-章节\nobreakdash-课后问卷
  \end{itemize}
\end{frame}  

\end{document}	