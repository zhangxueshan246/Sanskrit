%%% XeLaTeX-article %%%
%# -*- coding: utf-8 -*-
%!TEX encoding = UTF-8 Unicode
%!TEX TS-program = xelatex  
%---------------------虽然加了%还是要保留!

\documentclass[17pt]{beamer}
\mode<presentation>
{
\usetheme[width=40pt]{Hannover}
\usecolortheme[]{dove}
\usefonttheme[]{structurebold}
\setbeameroption{hide notes}
}

\usepackage{fontspec}
\usepackage{polyglossia}
\setmainfont{Arial} %设置主字体
\newfontfamily\sanskritfont[Script=Devanagari,Mapping=romantodevanagari,Scale=1.15]{Sanskrit 2003}             %输出天城体
%\newfontfamily\sanskritfont[Mapping=tex-text]{Times New Roman}              %输出转写
\doublehyphendemerits=-10000
\newcommand{\skt}[1]{{\sanskritfont{#1}}} %输出天城体
\newcommand{\skttrans}[1]{{\skt{#1}~#1}}  %输出天城体和转写
%----------------------------------------------------设置梵文输入方法 danda । ॥

\usepackage[UTF8,fontset=windows]{ctex}
\usepackage{amsmath}
%----------------------------------------------------设置中文环境

\usepackage{graphicx}
\usepackage{flafter} 
\graphicspath{{pic/}}
\usepackage{booktabs} 
\usepackage{nicematrix}
\newenvironment{indentlist}
  {\begin{list}{}{\setlength{\leftmargin}{2em}\setlength{\itemsep}{0.5em}}}
  {\end{list}}
%-----------------------------------------插图表格

\usepackage{hyperref} 
\usepackage[dvipsnames]{xcolor}
\usepackage{colortbl}
\definecolor{light-gray}{gray}{0.9}
%------------------------------颜色

\newcommand{\verbroot}[1]{\textcolor{red}{$\sqrt{}$#1}}
\newcommand{\sktroot}[1]{{\verbroot{\skt{#1}}}}
\newcommand{\skttransroot}[1]{{\sktroot{#1}~\textcolor{red}{#1}}}

\newcommand{\nounstem}[1]{\textcolor{red}{#1\nobreakdash-}}
\newcommand{\sktnounstem}[1]{{\textcolor{red}{\skt{#1\nobreakdash-}}}}
\newcommand{\skttransnounstem}[1]{{\sktnounstem{#1}~\nounstem{#1}}}

\newcommand{\verbstem}[1]{\textcolor{blue}{#1\nobreakdash-}}
\newcommand{\sktverbstem}[1]{{\textcolor{blue}{\skt{#1\nobreakdash-}}}}
\newcommand{\skttransverbstem}[1]{{\sktverbstem{#1}~\verbstem{#1}}}

\newcommand{\wordending}[1]{\textcolor{Orange}{\nobreakdash-#1}}
\newcommand{\sktending}[1]{{\textcolor{Orange}{\skt{-#1}}}}
\newcommand{\skttransending}[1]{{\sktending{#1}~\wordending{#1}}}

\newcommand{\fullpada}[1]{\textcolor{OliveGreen}{#1}}
\newcommand{\sktpada}[1]{{\textcolor{OliveGreen}{\skt{#1}}}}
\newcommand{\skttranspada}[1]{{\sktpada{#1}~\fullpada{#1}}}

\newcommand{\pratyaya}[1]{\textcolor{Plum}{#1}}
\newcommand{\sktpratyaya}[1]{{\textcolor{Plum}{\skt{#1}}}}
\newcommand{\skttranspratyaya}[1]{{\sktpratyaya{#1}~\pratyaya{#1}}}

\newcommand{\reconstruction}[1]{\textcolor{gray}{*#1}}
\newcommand{\fullsentence}[1]{\textcolor{MidnightBlue}{#1}}
\newcommand{\sktsentence}[1]{\textcolor{MidnightBlue}{\skt{#1}}}

\newcommand{\veryimportant}[1]{\textcolor{red}{#1}}
\newcommand{\important}[1]{\textcolor{blue}{#1}}
\newcommand{\notsoimportant}[1]{\textcolor{gray}{#1}}
%-------------------------------------------词根等标颜色

\title{{梵语入门}}
\subtitle{16. 词间元音连声}
\author[张雪杉]{文学院~~张雪杉 \\ zhangxueshan@sdnu.edu.cn}
\date{}
%\institute{}

\begin{document}	

\begin{frame}
  \titlepage
\end{frame}

\begin{frame}
  \frametitle{本节内容}
  \small
  \tableofcontents
\end{frame}

\section{上节作业}

\begin{frame}{第15章练习5}
  \small
  \begin{verse}
    \skt{1) suhṛdbālāyai·  annamudakaṃ ca·  ābharat ।}\\
    \skt{2) suhṛdo vācaṃ śrutvā bālā·  annodake āhṛtya prāṇamat ।}\\
    \skt{3) nṛpasya vācaṃ śrutvā narā vacanāni na vismaranti ।}\\
    \skt{4) dharmavidaṃ kṣatriyaṃ pūjayema । kṣatriyasya balaṃ kṛtsne jagati pāpānabhibhavet ।}
  \end{verse}
\end{frame}

\begin{frame}{第15章练习5}
  \small
  \begin{verse}
    \skt{5) akṣānkṣiptvā jitāḥ śūrā vane vastuṃ pratyāgacchan ।}\\
    \skt{6) yadā rathastho nṛpaḥ puramagacchattadā kṣatriyagaṇaḥ kṣaṇādapaśyat ।}\\
    \skt{7) marudiva kṛtsnaṃ jagattartumicchāmi ।}\\
    \skt{8) udakabhujaṃ bālaṃ dṛṣṭvā tuṣyāmi ।}\\
    \skt{9) lokakṛddevo bhūtāni kṛtsnāyāṃ bhuvi toṣayati ।}
  \end{verse}
\end{frame}

\subsection{复习}
\begin{frame}{标准格尾}
  \small
  \centering
    \begin{NiceTabular}{|c|c|c|c|c|c|c|}[hvlines, rules/width=0.3pt, rules/color=gray]
       & \Block{1-2}{单数} & & \Block{1-2}{双数} & & \Block{1-2}{复数} &  \\
       & 阳阴 & 中 & 阳阴 & 中 & 阳阴 & 中  \\
      主 & \wordending{s} & \Block{3-1}{\wordending{}} & \Block{3-1}{\wordending{au}}  & \Block{3-1}{\wordending{ī}}  & \Block{3-1}{\wordending{aḥ}} &  \Block{3-1}{\wordending{ni}}   \\
      呼 & \wordending{} & & & & & \\
      业 & \wordending{am} &  & & & & \\
      具 & \Block{1-2}{\wordending{ā}} &  & \Block{3-2}{\wordending{bhyām}}  & & \Block{1-2}{\wordending{bhiḥ}} & \\
      为 & \Block{1-2}{\wordending{e}} &  & & & \Block{2-2}{\wordending{bhyaḥ}}  & \\
      从 & \Block{2-2}{\wordending{aḥ}} &  & & & & \\
      属 & &  & \Block{2-2}{\wordending{oḥ}} & &  \Block{1-2}{\wordending{ām}} & \\
      依 & \Block{1-2}{\wordending{i}} &  & & & \Block{1-2}{\wordending{su}}&  \\
    \end{NiceTabular}
\end{frame}

\section{词间连声}
\begin{frame}{\insertsection}
  \small
  \tableofcontents[currentsection]
\end{frame}

\subsection{连声复习}
\begin{frame}{词间连声总则}
  \small
  \centering
  \begin{itemize}
    \item 前词尾音和后词首音相遇发生变化。
    \item 一般前音会变,有时后音也变。
    \item 辅音会被后音同化,而且要连写。
    \item 止音 ḥ 变化较特殊,前后原形都要看。
    \item 元音遇元音常融合,遇辅音不变。
  \end{itemize}
\end{frame}

\begin{frame}{词内元音连声}
 % \small
  \begin{itemize}
    \item 简单元音同类融合为长音\\
    \pratyaya{upa\nobreakdash-ā\nobreakdash-}\verbroot{nī} \fullpada{upānayati}  \\
    \item i/u加不同类时变成对应半元音\\
    \pratyaya{vi\nobreakdash-apa\nobreakdash-}\verbroot{nī} \fullpada{vyapanayati}  \\
    \pratyaya{adhi\nobreakdash-ā\nobreakdash-}\verbroot{gam} \fullpada{adhyāgacchati}  \\
    \item a/ā加不同类时升一级\\ 
    \pratyaya{pra\nobreakdash-ud\nobreakdash-}\verbroot{dhṛ} \fullpada{proddharati}  \\
  \end{itemize}
\end{frame}

\subsection{词间元音连声}
\begin{frame}{词间元音连声总则}
  \small
    \begin{itemize}
      \item \textbf{词尾元音 + 词首辅音:} \\不变化也不连写。
      \item \textbf{词尾元音 + 词首元音:} \\发生融合或音变,元音连续会出现。
    \end{itemize}
\end{frame}

\begin{frame}{词间元音连声总表}  
  \centering
  \includegraphics[width=\textwidth]{external_vowel_sandhi.png} %
\end{frame}

\subsection{~~~简单元音}

\begin{frame}{简单元音}
  \small
  \begin{itemize}
    \item a/ā自加变长音,其他后者升一级。\\
    \reconstruction{na adya} $\to$ \fullsentence{nādya} \sktsentence{nādya}  \\
    \reconstruction{na iha} $\to$ \fullsentence{neha} \sktsentence{neha}  %\\
    \reconstruction{na eva} $\to$ \fullsentence{naiva} \sktsentence{naiva} \\
    \reconstruction{sā auṣadhiḥ} $\to$ \fullsentence{sauṣadhiḥ} \sktsentence{sauṣadhiḥ}\\
    \reconstruction{yathā ṛṣiḥ} $\to$ \fullsentence{yathārṣiḥ} \sktsentence{yathārṣiḥ} \\
    \item i、u、ṛ自加变长音,遇他变为半元音。 \\
    \reconstruction{devī iha} $\to$ \fullsentence{devīha} \sktsentence{devīha} \\
    \reconstruction{sādhu uktam} $\to$ \fullsentence{sadhūktam} \sktsentence{sadhūktam} \\
    \reconstruction{iti uktvā} $\to$ \fullsentence{ity uktvā} \sktsentence{ityuktvā} \\
    \reconstruction{astu eva} $\to$ \fullsentence{astv eva} \sktsentence{astveva} \\
  \end{itemize}
\end{frame}

\subsection{~~~复合元音}
\begin{frame}{复合元音}
  \small
    \begin{itemize}
    \item e o除了能吃a,遇到元音就变a。\\
    \reconstruction{te api} $\to$ \fullsentence{te 'pi} \sktsentence{te 'pi}  \\
    \reconstruction{vane āste} $\to$ \fullsentence{vana āste} \sktsentence{vana āste}  %\\
    \item 只要遇到是元音,ai变ā来au变āv。 \\
    \reconstruction{tasmai adāt} $\to$ \fullsentence{tasmā adāt} \sktsentence{tasmā adāt} \\
    \reconstruction{tau ubhau} $\to$ \fullsentence{tāv ubhau} \sktsentence{tāvubhau} \\
  \end{itemize}
\end{frame}

\subsection{~~~不变尾元音}
\begin{frame}{不变尾元音}
  \small
  \begin{itemize}
  \item 双数格位ī、ū、e,感叹、呼唤和amī,\\
  在元音前都不变,传统称为pragṛhya。\\
    \fullsentence{cakṣuṣī ime} \sktsentence{cakṣuṣī ime}  \\
    \fullsentence{kanye āsāte atra} \sktsentence{kanye āsāte atra}  \\
    \fullsentence{he indra} \sktsentence{he indra}  \\
  \end{itemize}
\end{frame}


\section{本节作业}

\begin{frame}{\insertsection }
  \begin{itemize}
    \item
      \textbf{期末作业(12月31日截止):} \\第16章练习 1 (学习通有截图)
    \bigskip
    \item
      \notsoimportant{阅读教材第16课相关内容}
    \bigskip
    \item
      现在请做学习通\nobreakdash-章节\nobreakdash-课后问卷
  \end{itemize}
\end{frame}

\end{document}