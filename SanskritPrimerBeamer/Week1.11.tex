%%% XeLaTeX-article %%%
%# -*- coding: utf-8 -*-
%!TEX encoding = UTF-8 Unicode
%!TEX TS-program = xelatex  
%---------------------虽然加了%还是要保留!

\documentclass[17pt]{beamer}
\mode<presentation>
{
\usetheme[width=40pt]{Hannover}
\usecolortheme[]{dove}
\usefonttheme[]{structurebold}
\setbeameroption{hide notes}
}

\usepackage{fontspec}
\usepackage{polyglossia}
\setmainfont{Arial} %设置主字体
\newfontfamily\sanskritfont[Script=Devanagari,Mapping=romantodevanagari,Scale=1.15]{Sanskrit 2003}             %输出天城体
%\newfontfamily\sanskritfont[Mapping=tex-text]{Times New Roman}              %输出转写
\doublehyphendemerits=-10000
\newcommand{\skt}[1]{{\sanskritfont{#1}}} %输出天城体
\newcommand{\skttrans}[1]{{\skt{#1}~#1}}  %输出天城体和转写
%----------------------------------------------------设置梵文输入方法 danda । ॥

\usepackage[UTF8,fontset=windows]{ctex}
\usepackage{amsmath}
%----------------------------------------------------设置中文环境

\usepackage{graphicx}
\usepackage{flafter} 
\graphicspath{{pic/}}
\usepackage{booktabs} 
\usepackage{nicematrix}
\newenvironment{indentlist}
  {\begin{list}{}{\setlength{\leftmargin}{2em}\setlength{\itemsep}{0.5em}}}
  {\end{list}}
%-----------------------------------------插图表格

\usepackage{hyperref} 
\usepackage[dvipsnames]{xcolor}
\usepackage{colortbl}
\definecolor{light-gray}{gray}{0.9}
%------------------------------颜色

\newcommand{\verbroot}[1]{\textcolor{red}{$\sqrt{}$#1}}
\newcommand{\sktroot}[1]{{\verbroot{\skt{#1}}}}
\newcommand{\skttransroot}[1]{{\sktroot{#1}~\textcolor{red}{#1}}}

\newcommand{\nounstem}[1]{\textcolor{red}{#1\nobreakdash-}}
\newcommand{\sktnounstem}[1]{{\textcolor{red}{\skt{#1\nobreakdash-}}}}
\newcommand{\skttransnounstem}[1]{{\sktnounstem{#1}~\nounstem{#1}}}

\newcommand{\verbstem}[1]{\textcolor{blue}{#1\nobreakdash-}}
\newcommand{\sktverbstem}[1]{{\textcolor{blue}{\skt{#1\nobreakdash-}}}}
\newcommand{\skttransverbstem}[1]{{\sktverbstem{#1}~\verbstem{#1}}}

\newcommand{\wordending}[1]{\textcolor{Orange}{\nobreakdash-#1}}
\newcommand{\sktending}[1]{{\textcolor{Orange}{\skt{-#1}}}}
\newcommand{\skttransending}[1]{{\sktending{#1}~\wordending{#1}}}

\newcommand{\fullpada}[1]{\textcolor{OliveGreen}{#1}}
\newcommand{\sktpada}[1]{{\textcolor{OliveGreen}{\skt{#1}}}}
\newcommand{\skttranspada}[1]{{\sktpada{#1}~\fullpada{#1}}}

\newcommand{\pratyaya}[1]{\textcolor{Plum}{#1}}
\newcommand{\sktpratyaya}[1]{{\textcolor{Plum}{\skt{#1}}}}
\newcommand{\skttranspratyaya}[1]{{\sktpratyaya{#1}~\pratyaya{#1}}}

\newcommand{\reconstruction}[1]{\textcolor{gray}{*#1}}
\newcommand{\fullsentence}[1]{\textcolor{MidnightBlue}{#1}}
\newcommand{\sktsentence}[1]{\textcolor{MidnightBlue}{\skt{#1}}}

\newcommand{\veryimportant}[1]{\textcolor{red}{#1}}
\newcommand{\important}[1]{\textcolor{blue}{#1}}
\newcommand{\notsoimportant}[1]{\textcolor{gray}{#1}}
%-------------------------------------------词根等标颜色

\title{{梵语入门}}
\subtitle{12. 未完成时与祈愿语气}
\author[张雪杉]{文学院~~张雪杉 \\ zhangxueshan@sdnu.edu.cn}
\date{}
%\institute{}

\begin{document}	

\begin{frame}
  \titlepage
\end{frame}

\begin{frame}
  \frametitle{本节内容}
  \small
  \tableofcontents
\end{frame}

\section{上节作业}

\begin{frame}{第11章练习5}
  %\small
  \raggedright
  \begin{verse}
    \skt{1) vṛkṣātpatati} \\
    \skt{2) vṛkṣādavapatati} \\
    \skt{3) kumārameva} \\
    \skt{4) ślokamavagacchāmi} \\
    \skt{5) siṃhānna vidravati} \\
    \skt{6) gṛhāddhi} \\
    \skt{7) narāṃśca kumārāṃśca paśyāmi} \\
    \skt{8) mitramupaveśayati} \\
    \skt{9) kṣatriyāñjayati} \\
  \end{verse}
\end{frame}

\subsection{复习}
\begin{frame}{词间辅音连声总表}  
  \centering
  \includegraphics[width=0.75\textwidth]{consonantsandhi1.png} %
\end{frame}

\section{动词时态两种}
\begin{frame}{\insertsection }
  \small
  \tableofcontents[currentsection]
\end{frame}

\subsection{动词复习}
\begin{frame}{\insertsubsection ~~相关概念}
  \small
  \centering
  \resizebox{\textwidth}{!}{
    \begin{tabular}{@{}lllll@{}} % 7 columns: type, length/type, and 5 vowels
      形态层面 &     \\
      时态(Tense) & 现在时、未完成时、祈愿语气、\\
      & 命令语气、将来时、迂回将来时、\\
      & 条件式、完成时、不定过去时、祈求式 \\
      %\midrule
      意义层面 &  \\
      时(Time)  & 现在、过去、将来 \\
      体(Aspect) & 完成、未完成、一般 \\
      语气(Mood) & 陈述、祈使、虚拟、条件等 \\
      语态(Voice) & 主动、中间、被动 \\
      人称(Person) & 第一、第二、第三 \\
      数(Number) & 单数、双数、复数 \\
    \end{tabular} 
  }
\end{frame}

\begin{frame}{\insertsubsection ~~时态和语气}
  \small
  \centering
  \resizebox{\textwidth}{!}{
    \begin{tabular}{@{}lllll@{}} % 7 columns: type, length/type, and 5 vowels
      形态 &   & 意义 &   \\
      时态名称 & 时  & 体 & 语气   \\
      \midrule
      现在时  & 现在  & 一般/进行 & 陈述 \\
      \textcolor{blue}{未完成时}  & 过去  & 进行 & 陈述 \\
      命令语气  & 现在  & 一般/进行 & 祈使 \\
      祈愿语气  & 现在  & 一般/进行 & 可能 \\
      \textcolor{gray}{虚拟语气}  & 将来  & 一般/进行 & 祈求 \\
      将来时  & 将来  & 一般/进行 & 陈述 \\
      迂回将来时  & 将来  & 一般/进行 & 陈述 \\
      条件式  & 将来/过去  & 一般/进行 & 可能/条件 \\
      \textcolor{blue}{完成时}  & 现在 & 完成 & 陈述 \\
      \textcolor{blue}{不定过去时}  & 过去 & 一般 & 陈述 \\
      祈求式  & 现在 & 一般 & 祈使 \\
    \end{tabular} 
  }
\end{frame}

\begin{frame}{\insertsubsection ~~现在时构成}
  %\small
  \centering
  \begin{tabular}{@{}llll@{}} % 4 columns
    词根 & 词干  & 词尾  & 最终形式 \\
    \verbroot{bhū (1)} & \verbstem{bhava} & \wordending{ti}  & \fullpada{bhavati}  \\
    \verbroot{hṛṣ (4)} & \verbstem{hṛṣya} & \wordending{ti}  & \fullpada{hṛṣyati}  \\
    \verbroot{muc (6)} & \verbstem{muñca} & \wordending{ti}  & \fullpada{muñcati}  \\
    \verbroot{kath (10)} & \verbstem{kathaya} & \wordending{ti}  & \fullpada{kathayati}  \\
  \end{tabular} 
  \bigskip

  \begin{tabular}{@{}llll@{}} % 4 columns
    & 单数  & 双数  & 复数 \\
    第一人称 & \wordending{mi} & \wordending{vaḥ}  & \wordending{maḥ}  \\
    第二人称 & \wordending{si} & \wordending{thaḥ} & \wordending{tha}   \\
    第三人称 & \wordending{ti} & \wordending{taḥ} & \wordending{nti}  \\
  \end{tabular}   
\end{frame}

\subsection[未完成时\\与祈愿语气]{未完成时与祈愿语气}

\subsection{~~~意义}
\begin{frame}{未完成时与祈愿语气~~意义}
  %\small
  \begin{itemize}
      \item 未完成时 (Imperfect) 表示过去
      
      \item 祈愿语气 (Potential / Optative) \\ 表示一种泛指的可能性
      \begin{itemize}
        \item 第三人称可以表示希望、要求、推测、条件等
        \item 第一人称常用作 "让我们..." 
      \end{itemize}
    \end{itemize}
\end{frame}

\subsection{~~~形式}
\begin{frame}{未完成时与祈愿语气~~形式}
  %\small
  \begin{itemize}
    \item \textbf{语干:} 现在时语干加上时态标志
    \begin{itemize} 
      \item 未完成时:语干前加 \pratyaya{a\nobreakdash-} \\
        称为前加元音 (Augment)
      \item 祈愿语气:语尾前加 \pratyaya{\nobreakdash-ī\nobreakdash-}  \\
        在元音前加 \pratyaya{\nobreakdash-īy\nobreakdash-},\\
        和语干的 \pratyaya{\nobreakdash-a\nobreakdash-} 融合成 \pratyaya{\nobreakdash-e\nobreakdash-}(\pratyaya{\nobreakdash-ey\nobreakdash-})。
    \end{itemize}
  \end{itemize}
\end{frame}

\begin{frame}{未完成时与祈愿语气~~形式}
  \centering
  \begin{itemize}
    \item \textbf{语尾:} 均使用派生语尾\\ ~~~~~~~(Secondary Endings)
  \end{itemize} 
  \bigskip

  \begin{tabular}{@{}cccc@{}} % 4 columns
    & 单数  & 双数  & 复数 \\
    第一人称 & \wordending{(a)m} & \wordending{va}  & \wordending{ma}  \\
    第二人称 & \wordending{ḥ} & \wordending{tam} & \wordending{ta}   \\
    第三人称 & \wordending{t} & \wordending{tām} & \wordending{(a)n/\nobreakdash-uḥ}  \\
  \end{tabular}     
\end{frame}

\begin{frame}{例词:\verbroot{bhṛ (1)} 持}
  %\small
  \centering
  \begin{tabular}{@{}llll@{}} % 4 columns
    Impf & 单  & 双  & 复 \\
    1st & \fullpada{abharam} & \fullpada{abharāva}  & \fullpada{abharāma}  \\
    2nd & \fullpada{abharaḥ} & \fullpada{abharatam} & \fullpada{abharata}   \\
    3rd & \fullpada{abharat} & \fullpada{abharatām} & \fullpada{abharan}  \\
    \\
    Pot & 单  & 双  & 复 \\
    1st & \fullpada{bhareyam} & \fullpada{bhareva}  & \fullpada{bharema}  \\
    2nd & \fullpada{bhareḥ} & \fullpada{bharetam} & \fullpada{bhareta}   \\
    3rd & \fullpada{bharet} & \fullpada{bharetām} & \fullpada{bhareyuḥ}  \\
  \end{tabular}
\end{frame}

\subsection{~~~前加元音}
\begin{frame}{前加元音 \pratyaya{a\nobreakdash-} 的两条说明}
  %\small
  \begin{itemize}
    \item 前加元音 \pratyaya{a\nobreakdash-} 的位置紧挨词干,\\因此在前缀之后。\\
    \pratyaya{prati\nobreakdash-} + \pratyaya{a\nobreakdash-} + \verbstem{gaccha} + \wordending{t}\\
     $\to$ \fullpada{pratyagacchat} \\    
    \item 前加元音 \pratyaya{a\nobreakdash-} 与词根开头的元音\\结合变为三合元音。\\
    \verbroot{iṣ} ~现在时 \fullpada{icchati} \\
    ~~~~~~未完成时 \fullpada{aicchat} \\
  \end{itemize}
\end{frame}

\section{课堂练习}

\begin{frame}{辅音连声练习}
  \raggedright
  \begin{verse}
    \skt{1) nagarādgrāmaṃ gacchati ।}   \\
    \skt{2) grāmaṃ nagarānna gacchati ।}  \\
    \skt{3) nagarādiha gacchati ।}   \\
    \skt{4) nagarāddhi gacchati ।}   \\
    \skt{5) nagarādgrāmamāgacchati ।}   \\
    \skt{6) nagarādapagacchati ।}   \\
    \skt{7) buddhvā । bud\-dhvā ।}   \\
    \mbox{\skt{8) etānhatānyudhe nṛpa śocitumarhasi ।}}  \\
  \end{verse}
\end{frame}  

\begin{frame}{找到词根与前缀}
  %\small
  \raggedright
  例: \skt{samāgacchati} = \pratyaya{sam} + \pratyaya{ā} + \verbroot{gam}
  \bigskip  
  \begin{verse}
    \skt{1) paryanunayāmaḥ} \\
    \skt{2) saṃnibhṛtaḥ} \\
    \skt{3) anūttiṣṭhati} \\
    \skt{4) pratyupadravasi} \\
    \skt{5) samutkṣipataḥ} \\
    \skt{6) prodgacchanti । vyapāgacchāma} \\
    \skt{7) apātiṣṭhāvaḥ । apātiṣṭhāva}
  \end{verse}
\end{frame}

\section{本节作业}

\begin{frame}{\insertsection }
  \begin{itemize}
    \item
      第12章练习 6 (注意\veryimportant{勘误})。
    \item
      阅读教材第12课相关内容
    \bigskip
    \item
      现在请做学习通\nobreakdash-章节\nobreakdash-课后问卷
  \end{itemize}
\end{frame}

\end{document}