%%% XeLaTeX-article %%%
%# -*- coding: utf-8 -*-
%!TEX encoding = UTF-8 Unicode
%!TEX TS-program = xelatex  
%---------------------虽然加了%还是要保留!

\documentclass[17pt]{beamer}
\mode<presentation>
{
\usetheme[width=40pt]{Hannover}
\usecolortheme[]{dove}
\usefonttheme[]{structurebold}
\setbeameroption{hide notes}
}

\usepackage{fontspec}
\setmainfont{Arial} %设置主字体
\newfontfamily\sanskritfont[Script=Devanagari,Mapping=romantodevanagari,Scale=1.15]{Sanskrit 2003}             %输出天城体
%\newfontfamily\sanskritfont[Mapping=tex-text]{Times New Roman}              %输出转写
\doublehyphendemerits=-10000
\newcommand{\skt}[1]{{\sanskritfont{#1}}} %输出天城体
\newcommand{\skttrans}[1]{{\skt{#1}~#1}}  %输出天城体和转写
%----------------------------------------------------设置梵文输入方法 danda । ॥

\usepackage[UTF8,fontset=windows]{ctex}
\usepackage{amsmath}
%----------------------------------------------------设置中文环境

\usepackage{graphicx}
\usepackage{flafter} 
\graphicspath{{pic/}}
\usepackage{booktabs} 
\usepackage{nicematrix}
%-----------------------------------------插图表格

\usepackage{hyperref} 
\usepackage[dvipsnames]{xcolor}
\usepackage{colortbl}
\definecolor{light-gray}{gray}{0.9}
%------------------------------颜色

\newcommand{\verbroot}[1]{\textcolor{red}{$\sqrt{}$#1}}
\newcommand{\sktroot}[1]{{\verbroot{\skt{#1}}}}
\newcommand{\skttransroot}[1]{{\sktroot{#1}~\textcolor{red}{#1}}}

\newcommand{\nounstem}[1]{\textcolor{red}{#1\nobreakdash-}}
\newcommand{\sktnounstem}[1]{{\textcolor{red}{\skt{#1\nobreakdash-}}}}
\newcommand{\skttransnounstem}[1]{{\sktnounstem{#1}~\nounstem{#1}}}

\newcommand{\verbstem}[1]{\textcolor{blue}{#1\nobreakdash-}}
\newcommand{\sktverbstem}[1]{{\textcolor{blue}{\skt{#1\nobreakdash-}}}}
\newcommand{\skttransverbstem}[1]{{\sktverbstem{#1}~\verbstem{#1}}}

\newcommand{\wordending}[1]{\textcolor{Orange}{\nobreakdash-#1}}
\newcommand{\sktending}[1]{{\textcolor{Orange}{\skt{-#1}}}}
\newcommand{\skttransending}[1]{{\sktending{#1}~\wordending{#1}}}

\newcommand{\fullpada}[1]{\textcolor{OliveGreen}{#1}}
\newcommand{\sktpada}[1]{{\textcolor{OliveGreen}{\skt{#1}}}}
\newcommand{\skttranspada}[1]{{\sktpada{#1}~\fullpada{#1}}}

\newcommand{\pratyaya}[1]{\textcolor{Plum}{#1}}
\newcommand{\sktpratyaya}[1]{{\textcolor{Plum}{\skt{#1}}}}
\newcommand{\skttranspratyaya}[1]{{\sktpratyaya{#1}~\pratyaya{#1}}}

\newcommand{\veryimportant}[1]{\textcolor{red}{#1}}
\newcommand{\important}[1]{\textcolor{blue}{#1}}
\newcommand{\notsoimportant}[1]{\textcolor{gray}{#1}}
%-------------------------------------------词根等标颜色

\title{{梵语提高}}
\subtitle{20. 代词概述}
\author[张雪杉]{文学院~~张雪杉 \\ zhangxueshan@sdnu.edu.cn}
\date{}
%\institute{}


\begin{document}


\begin{frame}
  \titlepage
\end{frame}

\begin{frame}
  \frametitle{本节内容}
  \tableofcontents
\end{frame}

\section{上节作业}

\begin{frame}{第十九章练习3}
  \small
  \raggedright
  \begin{verse}
    \mbox{\skt{1) svaptumicchāma ityuktvā kanyā gṛhamāyan ।}} \\
    \skt{2) janāḥ pāpaṃ dviṣyuḥ śūraṃ ca stuyuḥ ।} \\
    \skt{3) cāpau gṛhītvā narāstau \important{nṛpāya} dadyuḥ ।} \\
    \skt{4) narā gṛhe syurna tu santīti nārī vakti ।} \\
    \skt{5) api chinnaṃ vṛkṣaṃ paśyasītyapṛccham । } 
    \mbox{\skt{na draṣṭuṃ śaknomīti kanyā punarabravīt ।}} \\
    \skt{6) saṃstutya devāñjuhuyāteti narānvadasi ।} \\
  \end{verse}
\end{frame}


\begin{frame}{第十九章练习3}
  \small
  \raggedright
  \begin{verse}
    \skt{7) śūrāḥ kṣatriyānaghnanniti janāḥ parituṣya nagaramalamakurvan ।} \\
    \skt{8) aśvaṃ narāya dadyā itīśvaro dāsīmavadat । }
    \mbox{\skt{janānāṃ pālaḥ sadāsīditi dāsī punarabravīt ।}} \\
    \mbox{\skt{9) īśvarasya vacanāni nāśṛṇumeti janā avadan ।}} \\
    \mbox{\skt{10) īśvaro varāṃ dāsīṃ stutvā dānānyadadāt ।}} \\
    \skt{11) rathasthaḥ kṣatriyo nidhanaṃ jñātvā na bibheti । kṣatriyaṃ stuyāma ।} \\
    \skt{12) hṛṣṭanetraḥ kumāraḥ kanyāyā rūpaṃ dṛṣṭvā darśanaṃ prabhāṃ ca sukhairvacanairastaut ।} \\
  \end{verse}

\end{frame}

\subsection{复习}
\begin{frame}{不带插入元音的动词}
  \raggedright
  \begin{itemize}
    \item 未完成时:
    \begin{itemize}
      \item 主动语态单数是强语干
      \item 词干前加 \pratyaya{a\nobreakdash-} 
    \end{itemize} 
    \item 祈愿语气
    \begin{itemize}
      \item 全部弱语干,词尾前加 \pratyaya{\nobreakdash-yā-}
      \item 第三人称复数词尾直接是 \wordending{yuḥ}
    \end{itemize}
  \end{itemize}
\end{frame}



\section{代词}
\begin{frame}{\insertsection }
    \tableofcontents[currentsection]
\end{frame}

\subsection{代词概述}
\begin{frame}{\insertsubsection}
  %\small
  \begin{itemize}
    \item 代词的分类(按照形态划分)
    \begin{itemize}
      \item 人称代词:你我\notsoimportant{(第32章)}他
      \item 指示代词:这那
      \item 按代词变化的形容词\notsoimportant{(第22章)}
    \end{itemize}
    \item 代词的特点
    \begin{itemize}
      \item 代词有单独的语尾
      \item 很多代词有多个语干
    \end{itemize}
  \end{itemize}
  
\end{frame}

\subsection{第三人称代词}
\begin{frame}{\insertsubsection ~~\nounstem{saḥ/tad}}
  \small
  \resizebox{\textwidth}{!}{
    \begin{tabular}{@{}llllllll@{}} % 8 columns
       &   \multicolumn{3}{c}{阳性}  &   \multicolumn{3}{c}{阴性}  \\
       & 单数 & 双数 & 复数  & 单数 & 双数 & 复数 \\
      主 & \fullpada{saḥ}  & \fullpada{tau} & \fullpada{te}  & \fullpada{sā}  & \fullpada{te} & \fullpada{tāḥ} \\
      业 & \fullpada{tam} & \fullpada{tau} & \fullpada{tān} & \fullpada{tām} & \fullpada{te} & \fullpada{tāḥ}\\
      具 & \fullpada{tena} & \fullpada{tābhyām} & \fullpada{taiḥ} & \fullpada{tayā} & \fullpada{tābhyām} & \fullpada{tābhiḥ} \\
      为 & \fullpada{tasmai} & \fullpada{tābhyām} & \fullpada{tebhyaḥ} & \fullpada{tasyai} & \fullpada{tābhyām} & \fullpada{tābhyaḥ}\\
      从 & \fullpada{tasmāt} & \fullpada{tābhyām} & \fullpada{tebhyaḥ} & \fullpada{tasyāḥ} & \fullpada{tābhyām} & \fullpada{tābhyaḥ}\\
      属 & \fullpada{tasya} & \fullpada{tayoḥ} & \fullpada{teṣām} & \fullpada{tasyāḥ} & \fullpada{tayoḥ} & \fullpada{tāsām} \\
      依 & \fullpada{tasmin} & \fullpada{tayoḥ} & \fullpada{teṣu} & \fullpada{tasyām} & \fullpada{tayoḥ} & \fullpada{tāsu}\\
       &   \multicolumn{3}{c}{中性}  &   \multicolumn{3}{c}{}  \\
      主业 & \fullpada{tat}  & \fullpada{te} & \fullpada{tāni}  &   \multicolumn{3}{c}{}  \\
    \end{tabular}
  }   
\end{frame}

\begin{frame}{\veryimportant{saḥ} 特殊连声}
  %\small
  \begin{itemize}
    \item \veryimportant{后词是辅音时变为} \fullpada{sa}
    
    \fullpada{sa naraḥ}
    \item 后词是 a 时变为 \fullpada{so}
    
    \skttranspada{so 'gacchat}
    \item 后词是其他元音时变为 \fullpada{sa}
    
    \fullpada{yo yacchraddhaḥ sa eva saḥ.}
  \end{itemize}
\end{frame}

\begin{frame}{用法}
  %\small
  \begin{itemize}
    \item 单独使用时为第三人称代词
    
    \fullpada{so 'gacchat.}
    \item 和其他词同格时可以是定冠词、指示代词或强调
    
    \fullpada{sa naraḥ. so 'ham. sa Rāmaḥ.}
    \item yad...tad...引起关系从句\notsoimportant{(23章)}
    
    \fullpada{yo yacchraddhaḥ sa eva saḥ.}
  \end{itemize}
\end{frame}

\subsection{指示代词}
\begin{frame}{\insertsubsection ~~\nounstem{ayam/idam}}
  \small
  \resizebox{\textwidth}{!}{
    \begin{tabular}{@{}llllllll@{}} % 8 columns
       &   \multicolumn{3}{c}{阳性}  &   \multicolumn{3}{c}{阴性}  \\
       & 单数 & 双数 & 复数  & 单数 & 双数 & 复数 \\
      主 & \fullpada{ayam}  & \fullpada{imau} & \fullpada{ime}  & \fullpada{iyam}  & \fullpada{ime} & \fullpada{imāḥ} \\
      业 & \fullpada{imam} & \fullpada{imau} & \fullpada{imān} & \fullpada{imām} & \fullpada{ime} & \fullpada{imāḥ}\\
      具 & \fullpada{anena} & \fullpada{ābhyām} & \fullpada{ebhiḥ} & \fullpada{anayā} & \fullpada{ābhyām} & \fullpada{ābhiḥ} \\
      为 & \fullpada{asmai} & \fullpada{ābhyām} & \fullpada{ebhyaḥ} & \fullpada{asyai} & \fullpada{ābhyām} & \fullpada{ābhyaḥ}\\
      从 & \fullpada{asmāt} & \fullpada{ābhyām} & \fullpada{ebhyaḥ} & \fullpada{asyāḥ} & \fullpada{ābhyām} & \fullpada{ābhyaḥ}\\
      属 & \fullpada{asya} & \fullpada{anayoḥ} & \fullpada{eṣām} & \fullpada{asyāḥ} & \fullpada{anayoḥ} & \fullpada{āsām} \\
      依 & \fullpada{asmin} & \fullpada{anayoḥ} & \fullpada{eṣu} & \fullpada{asyām} & \fullpada{anayoḥ} & \fullpada{āsu}\\
       &   \multicolumn{3}{c}{中性}  &   \multicolumn{3}{c}{}  \\
      主业 & \fullpada{idam}  & \fullpada{ime} & \fullpada{imāni}  &   \multicolumn{3}{c}{}  \\
    \end{tabular}
  }   
\end{frame}

\begin{frame}%[fragile]
  \frametitle{指示代词的远近程度}
  \small
  \centering
  \begin{NiceTabular}{cccc}
    \CodeBefore
      %\rectanglecolor{light-gray}{3-2}{5-2}
      %\rectanglecolor{light-gray}{7-2}{9-2}
    \Body % 7 columns
    %&   \multicolumn{3}{c}{现在时}  \\
    最近 & 近  & 远 & 最远  \\
    \fullpada{eṣaḥ/etad} & \fullpada{ayam/idam} & \fullpada{saḥ/tad} & \fullpada{asau/adas} \\
  \end{NiceTabular}   
\end{frame}

\section{本节作业}

\begin{frame}{\insertsection }
  \begin{itemize}
    \item
      第二十章练习4
    %\item
    %  勘误:第3题括号后加\skt{nṛpāya}
    \bigskip
    %\item
    %  现在请做学习通\nobreakdash-章节\nobreakdash-课后问卷
  \end{itemize}
\end{frame}  

\end{document}	
