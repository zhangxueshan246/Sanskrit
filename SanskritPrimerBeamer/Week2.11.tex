%%% XeLaTeX-article %%%
%# -*- coding: utf-8 -*-
%!TEX encoding = UTF-8 Unicode
%!TEX TS-program = xelatex  
%---------------------虽然加了%还是要保留!

\documentclass[17pt]{beamer}
\mode<presentation>
{
\usetheme[width=40pt]{Hannover}
\usecolortheme[]{dove}
\usefonttheme[]{structurebold}
\setbeameroption{hide notes}
}

\usepackage{fontspec}
\setmainfont{Arial} %设置主字体
\newfontfamily\sanskritfont[Script=Devanagari,Mapping=romantodevanagari,Scale=1.15]{Sanskrit 2003}             %输出天城体
%\newfontfamily\sanskritfont[Mapping=tex-text]{Times New Roman}              %输出转写
\doublehyphendemerits=-10000
\newcommand{\skt}[1]{{\sanskritfont{#1}}} %输出天城体
\newcommand{\skttrans}[1]{{\skt{#1}~#1}}  %输出天城体和转写
%----------------------------------------------------设置梵文输入方法 danda । ॥

\usepackage[UTF8,fontset=windows]{ctex}
\usepackage{amsmath}
%----------------------------------------------------设置中文环境

\usepackage{graphicx}
\usepackage{flafter} 
\graphicspath{{pic/}}
\usepackage{booktabs} 
\usepackage{nicematrix}
\usepackage{diagbox}
%-----------------------------------------插图表格

\usepackage{hyperref} 
\usepackage[dvipsnames]{xcolor}
\usepackage{colortbl}
\definecolor{light-gray}{gray}{0.85}
%------------------------------颜色

\newcommand{\verbroot}[1]{\textcolor{red}{$\sqrt{}$#1}}
\newcommand{\sktroot}[1]{{\verbroot{\skt{#1}}}}
\newcommand{\skttransroot}[1]{{\sktroot{#1}~\textcolor{red}{#1}}}

\newcommand{\nounstem}[1]{\textcolor{red}{#1\nobreakdash-}}
\newcommand{\sktnounstem}[1]{{\textcolor{red}{\skt{#1\nobreakdash-}}}}
\newcommand{\skttransnounstem}[1]{{\sktnounstem{#1}~\nounstem{#1}}}

\newcommand{\verbstem}[1]{\textcolor{blue}{#1\nobreakdash-}}
\newcommand{\sktverbstem}[1]{{\textcolor{blue}{\skt{#1\nobreakdash-}}}}
\newcommand{\skttransverbstem}[1]{{\sktverbstem{#1}~\verbstem{#1}}}

\newcommand{\wordending}[1]{\textcolor{Orange}{\nobreakdash-#1}}
\newcommand{\sktending}[1]{{\textcolor{Orange}{\skt{-#1}}}}
\newcommand{\skttransending}[1]{{\sktending{#1}~\wordending{#1}}}

\newcommand{\fullpada}[1]{\textcolor{OliveGreen}{#1}}
\newcommand{\sktpada}[1]{{\textcolor{OliveGreen}{\skt{#1}}}}
\newcommand{\skttranspada}[1]{{\sktpada{#1}~\fullpada{#1}}}

\newcommand{\pratyaya}[1]{\textcolor{Plum}{#1}}
\newcommand{\sktpratyaya}[1]{{\textcolor{Plum}{\skt{#1}}}}
\newcommand{\skttranspratyaya}[1]{{\sktpratyaya{#1}~\pratyaya{#1}}}

\newcommand{\reconstruction}[1]{\textcolor{gray}{*#1}}
\newcommand{\fullsentence}[1]{\textcolor{MidnightBlue}{#1}}

\newcommand{\veryimportant}[1]{\textcolor{red}{#1}}
\newcommand{\important}[1]{\textcolor{blue}{#1}}
\newcommand{\notsoimportant}[1]{\textcolor{gray}{#1}}
%-------------------------------------------词根等标颜色

\title{{梵语提高}}
\subtitle{26. i 和 u 的变格}
\author[张雪杉]{文学院~~张雪杉 \\ zhangxueshan@sdnu.edu.cn}
\date{}
%\institute{}

\begin{document}	

\begin{frame}
  \titlepage
\end{frame}

\begin{frame}
  \frametitle{本节内容}
  %\small
  \tableofcontents
\end{frame}

\section{上节作业}

\begin{frame}{第25章练习 4}
  \raggedright
  \small
  \begin{verse}
    \skt{1) api mahātapasaṃ kumāramāśramasya samīpa āsīnamapaśyaḥ । sa} \\
    \mbox{~~~~~\skt{duḥkhaśokavānihāgatya na kiṃ cidbhāṣate ।}}\\
    \mbox{\skt{2) tebhyo narebhyo dānānyābharadbhyo 'pi bibhemi ।}} \\
    \skt{3) mahābalo vīraḥ śocantībhyo nārībhyo 'pagacchanna kadā
    citpunaryuddhaṃ syādityaicchat ।}\\
    \skt{4) pāpaṃ nṛpaṃ jayato mahārathasya mahadyaśaḥ syāt ।}   \\
  \end{verse}
\end{frame}

\begin{frame}{第25章练习 4 }
  \raggedright
  \small
  \begin{verse}
    \skt{5) api bālau naraṃ mahācetasaṃ śṛṇvantau paśyasi । tau dhīmantau guṇavidau ca bhaviṣyataḥ ।}\\
    \skt{6) yadbhadraṃ ca guṇavaccāsti taddhīmadbhirjñāyate ।}\\
    \skt{7) naraḥ śokavāṃśchāyāvati vana āsīno 'pi na smayate ।}\\
    \skt{8) ratnavatīṃ rājñīṃ dṛṣṭavatī kanyā smayantyuttiṣṭhati ।}\\
  \end{verse}
\end{frame}

\begin{frame}{第25章练习 4 }
  \raggedright
  \small
  \begin{verse}    
    \skt{9) vardhamānena vismayena dāsī mahatyā nagaryāḥ samīpe bahujalāṃ nadīmapaśyat ।}   \\
    \skt{10) yāvalloke guṇavanto janāstāvadiha sukhaṃ bhaviṣyati ।}\\
    \skt{11) guṇavatāṃ narāṇāṃ manaḥsu yadasti tatsarveṣāṃ manaḥsu syāt ।} \\
  \end{verse}
\end{frame}

\begin{frame}{佛教书第29页}
    \small
  %\raggedright
  \begin{verse}\normalfont
    6. ārya\nobreakdash-avalokiteśvaraḥ bodhisattvaḥ \mbox{skandhān svabhāva\nobreakdash-śūnyān paśyati |} \\
    7. atha khalu tathāgataḥ piṇḍa\nobreakdash-pātāt pratikrāmati | \\
    8. sattvāḥ aṇḍajāḥ vā jarāyujāḥ vā saṃsvedajāḥ vā upapādukāḥ vā | na tu 
    kaścit sattvaḥ parinirvāṇam \mbox{gacchati |} \\
  \end{verse}
    
\end{frame}

\begin{frame}{佛教书第29页}
    \small
    \begin{verse}\normalfont
      9. buddha\nobreakdash-dharmaḥ buddha\nobreakdash-dharmaḥ iti abuddha\nobreakdash-dharmāḥ ca eva te | \\
    10. sarva\nobreakdash-dharmāḥ iti adharmāḥ | tasmāt tathāgataḥ sarva\nobreakdash-dharmāḥ buddha\nobreakdash-dharmāḥ iti vadati |\\
    11. kuśalāḥ dharmāḥ kuśalāḥ dharmāḥ iti adharmāḥ |
    \end{verse}
\end{frame}

\subsection{复习}
\begin{frame}{多语干名词变格}
  \small
  \nounstem{guṇavant} adj. 有德 
  \bigskip

  \centering
  \resizebox{\textwidth}{!}{
    \begin{NiceTabular}{|c|c|c|c|c|c|c|}[hvlines, rules/width=0.3pt, rules/color=gray]
      阳性 & 单数 & 双数 & 复数  \\
      主 & \cellcolor{light-gray}\fullpada{guṇavān}  & \cellcolor{light-gray}\Block{3-1}{\fullpada{guṇavantau}} & \cellcolor{light-gray}\Block{2-1}{\fullpada{guṇavantaḥ}}  \\
      呼 & \cellcolor{light-gray}\fullpada{guṇavan} & \cellcolor{light-gray} & \cellcolor{light-gray} \\
      业 & \cellcolor{light-gray}\fullpada{guṇavantam} & \cellcolor{light-gray} & \fullpada{guṇavataḥ} \\
      中性 & \Block{2-1}{\fullpada{guṇavat}}  & \Block{2-1}{\fullpada{guṇavatī}} & \Block{2-1}{\fullpada{guṇavanti}}   \\
      主呼业 &  &  &  \\
      具 & \fullpada{guṇavatā} & \Block{3-1}{\fullpada{guṇavadbhyām}} & \fullpada{guṇavadbhiḥ} \\
      为 & \fullpada{guṇavate} &  & \Block{2-1}{\fullpada{guṇavadbhyaḥ}} \\
      从 & \Block{2-1}{\fullpada{guṇavataḥ}} &  &  \\
      属 &  & \Block{2-1}{\fullpada{guṇavatoḥ}} & \fullpada{guṇavatām} \\
      依 & \fullpada{guṇavati} &  & \fullpada{guṇavatsu} \\
    \end{NiceTabular}
  }
\end{frame}

\begin{frame}{各种分词}
  \small
  \centering
    \begin{NiceTabular}{@{}lllll@{}} % 6 columns
       &  & 主动 & 中间 & 被动\\
      过去 &  & \pratyaya{\nobreakdash-tavant\nobreakdash-}  & $\times$ & \pratyaya{\nobreakdash-ta\nobreakdash-}\\
      \Block{2-1}{现在} & a语干 & \pratyaya{\nobreakdash-nt\nobreakdash-} & \pratyaya{\nobreakdash-māna\nobreakdash-} & \Block{2-1}{\pratyaya{\nobreakdash-yamāna\nobreakdash-}} \\
       & 非a语干 & \pratyaya{\nobreakdash-ant} & \pratyaya{\nobreakdash-āna\nobreakdash-} &  \\
      将来 &  & \pratyaya{\nobreakdash-syant\nobreakdash-} & \Block{1-2}{\pratyaya{\nobreakdash-syamāna\nobreakdash-}} & \\
    \end{NiceTabular}
\end{frame}

\section[i u 变格]{i 和 u 结尾的名词}
\begin{frame}{\insertsection }
    \small
    \tableofcontents[currentsection]
\end{frame}

\subsection{名词变格}
\begin{frame}{\insertsubsection }
  \small
  \begin{itemize}
    \item i 和 u 结尾的名词阳性、阴性和中性都有可能。
    
    \item 变格都不一样但都比较眼熟。
    
    \item 中性名词在所有元音格尾前加 \pratyaya{\nobreakdash-n\nobreakdash-}。
    
    \item 阳性名词某些格不加 \pratyaya{\nobreakdash-n\nobreakdash-} 而是二合。
    
    \item 阴性名词单数后四格有两套形式:按阳性或 ī/ū 结尾阴性词变化均可。

  \end{itemize}
\end{frame}

\begin{frame}{i u 变格大表}  
  \centering
  \includegraphics[width=\textwidth]{iudeclension.png} %
\end{frame}

\subsection{形容词变格}

\begin{frame}{\insertsubsection }
  \small
  \begin{itemize}
    \item i 和 u 结尾的形容词一般按对应的名词变化。也有例外:\\
    \item 中性形容词在加 \pratyaya{\nobreakdash-n\nobreakdash-} 的地方也可按阳性变。\\
        
    \nounstem{guru} 重 ~~中性单数为格:\\ ~~~\fullpada{guruṇe} 或 \fullpada{gurave} 都可以。\\
    \item u结尾的形容词阴性也可加阴性词缀 \pratyaya{\nobreakdash-ī\nobreakdash-}。\\
        
    \nounstem{guru} 重 ~~阴性词干:\\ ~~~\nounstem{guru} 或 \nounstem{gurvī} 都可以。
  \end{itemize}
\end{frame}

\subsection{ti词缀}
\begin{frame}{构词法~~\insertsubsection }
  \small
  \begin{itemize}
    \item 大部分 i 结尾的阴性词是抽象名词。\\
    \item 形式是零级词根 + \pratyaya{\nobreakdash-ti\nobreakdash-} 词缀。\\   
    \verbroot{nī} 带领 ~~\nounstem{nīti} 规范、政事\\ 
    \verbroot{man} 想 ~~\nounstem{mati} 智慧 \\ 
    \item 注意 t 前连声。\\
    \verbroot{dṛś} 看 ~~\nounstem{dṛṣṭi} 视力 \\ 
    \verbroot{vṛdh} 增长 ~~\nounstem{vṛddhi} 增长 \\   
    \verbroot{budh} 觉醒 ~~\nounstem{buddhi} 智慧 \\  
  \end{itemize}
\end{frame}

\section{本节作业}

\begin{frame}{\insertsection }
  \begin{itemize}
    \item
      第26章练习4
    \item 先不做佛教了……
  \end{itemize}
\end{frame}  

\end{document}	