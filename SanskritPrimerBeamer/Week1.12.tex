%%% XeLaTeX-article %%%
%# -*- coding: utf-8 -*-
%!TEX encoding = UTF-8 Unicode
%!TEX TS-program = xelatex  
%---------------------虽然加了%还是要保留!

\documentclass[17pt]{beamer}
\mode<presentation>
{
\usetheme[width=40pt]{Hannover}
\usecolortheme[]{dove}
\usefonttheme[]{structurebold}
\setbeameroption{hide notes}
}

\usepackage{fontspec}
\usepackage{polyglossia}
\setmainfont{Arial} %设置主字体
\newfontfamily\sanskritfont[Script=Devanagari,Mapping=romantodevanagari,Scale=1.15]{Sanskrit 2003}             %输出天城体
%\newfontfamily\sanskritfont[Mapping=tex-text]{Times New Roman}              %输出转写
\doublehyphendemerits=-10000
\newcommand{\skt}[1]{{\sanskritfont{#1}}} %输出天城体
\newcommand{\skttrans}[1]{{\skt{#1}~#1}}  %输出天城体和转写
%----------------------------------------------------设置梵文输入方法 danda । ॥

\usepackage[UTF8,fontset=windows]{ctex}
\usepackage{amsmath}
%----------------------------------------------------设置中文环境

\usepackage{graphicx}
\usepackage{flafter} 
\graphicspath{{pic/}}
\usepackage{booktabs} 
\usepackage{nicematrix}
\newenvironment{indentlist}
  {\begin{list}{}{\setlength{\leftmargin}{2em}\setlength{\itemsep}{0.5em}}}
  {\end{list}}
%-----------------------------------------插图表格

\usepackage{hyperref} 
\usepackage[dvipsnames]{xcolor}
\usepackage{colortbl}
\definecolor{light-gray}{gray}{0.9}
%------------------------------颜色

\newcommand{\verbroot}[1]{\textcolor{red}{$\sqrt{}$#1}}
\newcommand{\sktroot}[1]{{\verbroot{\skt{#1}}}}
\newcommand{\skttransroot}[1]{{\sktroot{#1}~\textcolor{red}{#1}}}

\newcommand{\nounstem}[1]{\textcolor{red}{#1\nobreakdash-}}
\newcommand{\sktnounstem}[1]{{\textcolor{red}{\skt{#1\nobreakdash-}}}}
\newcommand{\skttransnounstem}[1]{{\sktnounstem{#1}~\nounstem{#1}}}

\newcommand{\verbstem}[1]{\textcolor{blue}{#1\nobreakdash-}}
\newcommand{\sktverbstem}[1]{{\textcolor{blue}{\skt{#1\nobreakdash-}}}}
\newcommand{\skttransverbstem}[1]{{\sktverbstem{#1}~\verbstem{#1}}}

\newcommand{\wordending}[1]{\textcolor{Orange}{\nobreakdash-#1}}
\newcommand{\sktending}[1]{{\textcolor{Orange}{\skt{-#1}}}}
\newcommand{\skttransending}[1]{{\sktending{#1}~\wordending{#1}}}

\newcommand{\fullpada}[1]{\textcolor{OliveGreen}{#1}}
\newcommand{\sktpada}[1]{{\textcolor{OliveGreen}{\skt{#1}}}}
\newcommand{\skttranspada}[1]{{\sktpada{#1}~\fullpada{#1}}}

\newcommand{\pratyaya}[1]{\textcolor{Plum}{#1}}
\newcommand{\sktpratyaya}[1]{{\textcolor{Plum}{\skt{#1}}}}
\newcommand{\skttranspratyaya}[1]{{\sktpratyaya{#1}~\pratyaya{#1}}}

\newcommand{\reconstruction}[1]{\textcolor{gray}{*#1}}
\newcommand{\fullsentence}[1]{\textcolor{MidnightBlue}{#1}}
\newcommand{\sktsentence}[1]{\textcolor{MidnightBlue}{\skt{#1}}}

\newcommand{\veryimportant}[1]{\textcolor{red}{#1}}
\newcommand{\important}[1]{\textcolor{blue}{#1}}
\newcommand{\notsoimportant}[1]{\textcolor{gray}{#1}}
%-------------------------------------------词根等标颜色

\title{{梵语入门}}
\subtitle{13. ī/ū 变格与止音连声}
\author[张雪杉]{文学院~~张雪杉 \\ zhangxueshan@sdnu.edu.cn}
\date{}
%\institute{}

\begin{document}	

\begin{frame}
  \titlepage
\end{frame}

\begin{frame}
  \frametitle{本节内容}
  \small
  \tableofcontents
\end{frame}

\section{上节作业}

\begin{frame}{第12章练习6}
  \small
  \begin{verse}
    \skt{1) naraḥ· purātpratyāgatya bālāyai vṛkṣamadarśayat ।} \\
    \skt{2) devānāṃ prabhāṃ paśyeḥ· iti naraḥ· mitramavadat ।} \\
    \skt{3) ugraṃ kṣatriyaṃ dṛṣṭvā naraḥ· mitramapi kṣatriyamapaśyaḥ· iti· apṛcchat ।} \\
    \mbox{\skt{4) devānyajñaiḥ· toṣayema· iti śūraḥ· avadat ।}}\\
    \skt{5) yuddhe yathā nagarāṇi \veryimportant{naṣṭāni} tathā janāḥ· \veryimportant{naṣṭāḥ} ।} \\
  \end{verse}
\end{frame}

\begin{frame}{第12章练习6}
  \small
  \begin{verse}
    \skt{6) ugraḥ·  nṛpaḥ·  janebhyaḥ·  dānāni·  āptumaicchat ।}\\
    \mbox{\skt{7) padmaṃ darśayeḥ·  iti bālau kanyāmavadatām ।}}\\
    \skt{8) bhadre kāle punar·  samāgacchema·  iti·  uktvā kumāraḥ kṣaṇādapāgacchat ।}\\
    \skt{9) kṣatriyasya vacanāni śrutvā narāḥ·  atra \mbox{niṣadya·  cintāḥ·  vismareḥ·  iti·  avadan ।}}\\
    \skt{10) yathā janāḥ·  kṣetre sthitasya vṛkṣasya phalaiḥ·  tuṣyanti tathā·  aśvāḥ ।}\\
  \end{verse}
\end{frame}

\subsection{复习}
\begin{frame}{未完成时与祈愿语气}
  %\small
  \centering
  \begin{tabular}{@{}llll@{}} % 4 columns
    Impf & 单  & 双  & 复 \\
    1st & \fullpada{abharam} & \fullpada{abharāva}  & \fullpada{abharāma}  \\
    2nd & \fullpada{abharaḥ} & \fullpada{abharatam} & \fullpada{abharata}   \\
    3rd & \fullpada{abharat} & \fullpada{abharatām} & \fullpada{abharan}  \\
    \\
    Pot & 单  & 双  & 复 \\
    1st & \fullpada{bhareyam} & \fullpada{bhareva}  & \fullpada{bharema}  \\
    2nd & \fullpada{bhareḥ} & \fullpada{bharetam} & \fullpada{bhareta}   \\
    3rd & \fullpada{bharet} & \fullpada{bharetām} & \fullpada{bhareyuḥ}  \\
  \end{tabular}
\end{frame}

\section[ī ū 变格]{ī和ū结尾的名词变格}
\begin{frame}{\insertsection}
  \small
  \tableofcontents[currentsection]
\end{frame}

\begin{frame}{课本附录有变格大表}  
  \centering
  \includegraphics[width=\textwidth]{longiudeclension.png} %
\end{frame}

\begin{frame}{ī和ū结尾的名词概述}
  \small
  \begin{itemize}
    \item 以ī和ū结尾的名词均为\important{阴性},\\
    变格也很像以ā结尾的阴性名词。
    \item 按照音节可分两类:\\
    多音节:\nounstem{nadī} (河), \nounstem{camū} (军队)。\\
    单音节:\nounstem{dhī} (思想), \nounstem{bhū} (大地)。\\
    \important{主要区别:} \\
    单音节某些格可用两套格尾。\\
    遇到元音词尾时词内连声规则不同。
  \end{itemize}
\end{frame}

\subsection[多音节ī名词]{多音节 ī 结尾名词}
\begin{frame}{\insertsubsection}
  \small
  \begin{itemize}
    \item 元音格尾前 ī 变 y。
     \nounstem{nadī} f. 河
  \end{itemize}
  \centering
  \begin{NiceTabular}{|c|c|c|c|}[hvlines, rules/width=0.3pt, rules/color=gray]
    & 单数 & 双数 & 复数 \\
    主 & \fullpada{nadī} & \Block{3-1}{\fullpada{nadyau}} & \Block{2-1}{\fullpada{nadyaḥ}}  \\
    呼 & \fullpada{nadi} &  &  \\
    业 & \fullpada{nadīm} &  & \fullpada{nadīḥ} \\
    具 & \fullpada{nadyā} & \Block{3-1}{\fullpada{nadībhyām}} & \fullpada{nadībhiḥ} \\
    为 & \fullpada{nadyai} &  & \Block{2-1}{\fullpada{nadībhyaḥ}} \\
    从 & \Block{2-1}{\fullpada{nadyāḥ}} &  &  \\
    属 &  & \Block{2-1}{\fullpada{nadyoḥ}} & \fullpada{nadīnām} \\
    依 & \fullpada{nadyām} &  & \fullpada{nadīṣu} \\
  \end{NiceTabular}
    \vspace*{0.3em}

  \footnotesize{注意:和ā不太像的地方和a像。}
\end{frame}


\subsection[单音节ī名词]{单音节 ī 结尾名词}
\begin{frame}{\insertsubsection}
  \small
  \begin{itemize}
    \item 元音格尾前 ī 变 iy。
    \nounstem{dhī} f. 思想\\
  \end{itemize}
  \centering
  \begin{NiceTabular}{|c|c|c|c|}[hvlines, rules/width=0.3pt, rules/color=gray]
    & 单数 & 双数 & 复数 \\
    主呼 & \fullpada{dhīḥ} & \Block{2-1}{\fullpada{dhiyau}} & \Block{2-1}{\fullpada{dhiyaḥ}} \\
    业 & \fullpada{dhiyam} &  & \\
    具 & \fullpada{dhiyā} & \Block{3-1}{\fullpada{dhībhyām}} & \fullpada{dhībhiḥ} \\
    为 & \fullpada{dhiye dhiyai} &  & \Block{2-1}{\fullpada{dhībhyaḥ}} \\
    从 & \Block{2-1}{\fullpada{dhiyaḥ dhiyāḥ}} &  &  \\
    属 &  & \Block{2-1}{\fullpada{dhiyoḥ}} & \fullpada{dhiyām dhīnām} \\
    依 & \fullpada{dhiyi dhiyām} &  & \fullpada{dhīṣu} \\
  \end{NiceTabular}
  \vspace*{0.3em}

  \footnotesize{注意:单数主呼增加止声。\\两种形式的格使用ā或辅音语尾。}
\end{frame}

\subsection{ū 结尾名词}
\begin{frame}{\insertsubsection}
  \small
  \begin{itemize}
    \item ū 结尾名词几乎和 ī 结尾名词完全平行。\\
    \item 多音节词元音格尾前 ū 变 v。\\
    \nounstem{camū} f. 军队\\
    只有\important{单数主格}不同 \fullpada{camūḥ}\\
    单业 \fullpada{camūm} 单具 \fullpada{camvā}
    
    \item 单音节词元音格尾前 ū 变 uv。\\
    \nounstem{bhū} f. 大地\\
    单主 \fullpada{bhūḥ} 
    单业 \fullpada{bhuvam} 单具 \fullpada{bhuvā}
  \end{itemize}
\end{frame}

\subsection{阴性词缀}
\begin{frame}{构词法~~\insertsubsection}
  \small
  \begin{itemize}
    \item 大多数以a结尾的词变阴性加 \pratyaya{\nobreakdash-ā\nobreakdash-}\\
    \nounstem{bāla} m. 男孩 \nounstem{bālā} f. 女孩 \\
    \nounstem{nava} m. 新的 \nounstem{navā} f. 新的 \\    
    \item 也有的不加 \pratyaya{\nobreakdash-ā\nobreakdash-} 而是加 \pratyaya{\nobreakdash-ī\nobreakdash-}\\
    \nounstem{deva} m. 天神 \nounstem{devī} f. 女神 \\
  \end{itemize}
\end{frame}

\begin{frame}{性数格一致原则}
  \small
  \begin{itemize}
    \item 此前见到的形容词和名词格尾都一样。\\
    
    阳性 \fullsentence{priyaḥ bālaḥ}\\
    中性 \fullsentence{priyaṃ mitram} \\
    阴性 \fullsentence{priyā kanyā}
    
    \item 现在有可能长得不一样了但还是一致的。\\
    
    主格 \fullsentence{priyā nārī}\\
    呼格 \fullsentence{priye nāri}\\
    \bigskip
    
    希望大家能认出来。
  \end{itemize}
\end{frame}


\section{止音连声}
\begin{frame}{\insertsection}
  \small
  \tableofcontents[currentsection]
\end{frame}


\begin{frame}{复习~~词间连声总则}
  \small
  \centering
  \begin{itemize}
    \item 前词尾音和后词首音相遇发生变化。
    \item 一般前音会变,有时后音也变。
    \item 辅音会被后音同化,而且要连写。
    \item \important{止音 ḥ 变化较特殊,前后原形都要看。}
    \item 元音遇元音常融合,遇辅音不变。
  \end{itemize}
\end{frame}

\begin{frame}{止音连声表}  
  \centering
  \includegraphics[width=0.4\textwidth]{visargasandhi.png} %
\end{frame}

\subsection{基本情况}
\begin{frame}{止音连声\insertsubsection}
  \small
    \begin{itemize}
      \item 止音 (ḥ) 通常来源于词尾的 s 或 r。
      
      它的变化主要可分三种情况:
      \item 后遇清辅音,变咝或止音 (ḥ)。
      \item 后遇浊音较复杂,前面元音影响大。
      \item 原始结尾 r,浊前不变化。
    \end{itemize}
\end{frame}

\subsection{后遇清辅音}
\begin{frame}{\insertsubsection}
  \small
  \begin{itemize}
    \item s 遇清的腭卷齿,变为相应的咝音。\\    
    \reconstruction{kuṭhāraiḥ ṭaṅkaiḥ ca} \\$\to$ \fullsentence{kuṭhāraiṣ ṭaṅkaiś ca} \sktsentence{kuṭhāraiṣṭaṅkaiśca}\\
    \reconstruction{bhrātaraḥ trayaḥ} \\$\to$ \fullsentence{bhrātaras trayaḥ} \sktsentence{bhrātarastrayaḥ}\\
    \item 遇他清音或落尾,就要变为 visarga (ḥ)。 \\
    \reconstruction{tās kānyāḥ} $\to$ \fullsentence{tāḥ kānyāḥ} \sktsentence{tāḥ kānyāḥ}\\
    \reconstruction{vrīhis pacyate} \\$\to$ \fullsentence{vrīhiḥ pacyate} \sktsentence{vrīhiḥ pacyate}\\    
  \end{itemize}
\end{frame}

\subsection{后遇浊音}
\begin{frame}{\insertsubsection}
  \small
  \begin{itemize}
    \item 浊前非a/ā元音后,s音就要变成r。 \\
    \reconstruction{matiḥ mama} \\$\to$ \fullsentence{matir mama} \sktsentence{matirmama}\\ 
    \reconstruction{paśuḥ iva} \\$\to$ \fullsentence{paśur iva} \sktsentence{paśuriva}\\   
  \end{itemize}
\end{frame}

\begin{frame}{\insertsubsection}
  \small
  \begin{itemize}
    \item as遇浊辅变成o, \\
    \reconstruction{aśvas vahati} $\to$ \fullsentence{aśvo vahati} \sktsentence{aśvo vahati}\\ 
    \item 遇a变o还吃a, \\
    \reconstruction{aśvas api} $\to$ \fullsentence{aśvo 'pi} \sktsentence{aśvo 'pi}\\   
    \item 遇他元音则脱落。 \\
    \reconstruction{aśvas iva} $\to$ \fullsentence{aśva iva} \sktsentence{aśva iva}\\
    \item ās浊前都脱落。 \\
    \reconstruction{aśvās vahanti} $\to$ \fullsentence{aśvā vahanti} \sktsentence{aśvā vahanti}\\
    \reconstruction{aśvās ūhuḥ} $\to$ \fullsentence{aśvā ūhuḥ} \sktsentence{aśvā ūhuḥ}\\
  \end{itemize}
\end{frame}

\subsection{原始结尾r}
\begin{frame}{\insertsubsection}
  \small
  \begin{itemize}
    \item 清前之r变如s。\\
    \reconstruction{punar tatra} $\to$ \fullsentence{punas tatra} \sktsentence{punastatra}\\ 
    \reconstruction{punar punar} $\to$ \fullsentence{punaḥ punaḥ} \sktsentence{punaḥ punaḥ}\\ 
    \item 浊前之r不变化。\\
    \reconstruction{punar asti} $\to$ \fullsentence{punar asti} \sktsentence{punarasti}\\ 
  \end{itemize}
\end{frame}

\subsection{重复的r}
\begin{frame}{\insertsubsection}
  \small
  \begin{itemize}
  \item 不可出现两个r,不管原始s或r。\\
  词尾的r要消失,前音若短要变长。 \\
    \reconstruction{punar rohati} \\$\to$ \fullsentence{punā rohati} \sktsentence{punā rohati}\\ 
    \reconstruction{taruḥ rohati} \\$\to$ \fullsentence{tarū rohati} \sktsentence{tarū rohati}\\
  \end{itemize}
\end{frame}

\section{课堂练习}

\begin{frame}{找到词根与前缀}
  %\small
  \raggedright
  例: \skt{samāgacchati} = \pratyaya{sam} + \pratyaya{ā} + \verbroot{gam}
  \bigskip  
  \begin{verse}
    \skt{1) paryanunayāmaḥ} \\
    \skt{2) saṃnibhṛtaḥ} \\
    \skt{3) anūttiṣṭhati} \\
    \skt{4) pratyupadravasi} \\
    \skt{5) samutkṣipataḥ} \\
    \skt{6) prodgacchanti । vyapāgacchāma} \\
    \skt{7) apātiṣṭhāvaḥ । apātiṣṭhāva}
  \end{verse}
\end{frame}

\begin{frame}{止音连声练习}
  \begin{verse}
    	\skt{1) naraḥ + tatra}\\
    	\skt{2) naraḥ + uktavān}\\
    	\skt{3) bālāḥ + gacchanti}\\
    	\skt{4) kaviḥ + vadati}\\
    	\skt{5) guroḥ + āśramaḥ}\\
    	\skt{6) naraḥ + ayam}\\
    	\skt{7) punaḥ + rāmaḥ}
  \end{verse}
\end{frame}

\section{本节作业}

\begin{frame}{\insertsection }
  \begin{itemize}
    \item
      没有作业\string~
    \item
      阅读教材第13课相关内容
    \bigskip
    \item
      现在请做学习通\nobreakdash-章节\nobreakdash-课后问卷
  \end{itemize}
\end{frame}

\end{document}
