%%% XeLaTeX-article %%%
%# -*- coding: utf-8 -*-
%!TEX encoding = UTF-8 Unicode
%!TEX TS-program = xelatex  
%---------------------虽然加了%还是要保留!

\documentclass[17pt]{beamer}
\mode<presentation>
{
\usetheme[width=40pt]{Hannover}
\usecolortheme[]{dove}
\usefonttheme[]{structurebold}
\setbeameroption{hide notes}
}

\usepackage{fontspec}
\usepackage{polyglossia}
\setmainfont{Arial} %设置主字体
\newfontfamily\sanskritfont[Script=Devanagari,Mapping=romantodevanagari,Scale=1.15]{Sanskrit 2003}             %输出天城体
%\newfontfamily\sanskritfont[Mapping=tex-text]{Times New Roman}              %输出转写
\doublehyphendemerits=-10000
\newcommand{\skt}[1]{{\sanskritfont{#1}}} %输出天城体
\newcommand{\skttrans}[1]{{\skt{#1}~#1}}  %输出天城体和转写
%----------------------------------------------------设置梵文输入方法 danda । ॥

\usepackage[UTF8,fontset=windows]{ctex}
\usepackage{amsmath}
%----------------------------------------------------设置中文环境

\usepackage{graphicx}
\usepackage{flafter} 
\graphicspath{{pic/}}
\usepackage{booktabs} 
\usepackage{nicematrix}
\newenvironment{indentlist}
  {\begin{list}{}{\setlength{\leftmargin}{2em}\setlength{\itemsep}{0.5em}}}
  {\end{list}}
%-----------------------------------------插图表格

\usepackage{hyperref} 
\usepackage[dvipsnames]{xcolor}
\usepackage{colortbl}
\definecolor{light-gray}{gray}{0.9}
%------------------------------颜色

\newcommand{\verbroot}[1]{\textcolor{red}{$\sqrt{}$#1}}
\newcommand{\sktroot}[1]{{\verbroot{\skt{#1}}}}
\newcommand{\skttransroot}[1]{{\sktroot{#1}~\textcolor{red}{#1}}}

\newcommand{\nounstem}[1]{\textcolor{red}{#1\nobreakdash-}}
\newcommand{\sktnounstem}[1]{{\textcolor{red}{\skt{#1\nobreakdash-}}}}
\newcommand{\skttransnounstem}[1]{{\sktnounstem{#1}~\nounstem{#1}}}

\newcommand{\verbstem}[1]{\textcolor{blue}{#1\nobreakdash-}}
\newcommand{\sktverbstem}[1]{{\textcolor{blue}{\skt{#1\nobreakdash-}}}}
\newcommand{\skttransverbstem}[1]{{\sktverbstem{#1}~\verbstem{#1}}}

\newcommand{\wordending}[1]{\textcolor{Orange}{\nobreakdash-#1}}
\newcommand{\sktending}[1]{{\textcolor{Orange}{\skt{-#1}}}}
\newcommand{\skttransending}[1]{{\sktending{#1}~\wordending{#1}}}

\newcommand{\fullpada}[1]{\textcolor{OliveGreen}{#1}}
\newcommand{\sktpada}[1]{{\textcolor{OliveGreen}{\skt{#1}}}}
\newcommand{\skttranspada}[1]{{\sktpada{#1}~\fullpada{#1}}}

\newcommand{\pratyaya}[1]{\textcolor{Plum}{#1}}
\newcommand{\sktpratyaya}[1]{{\textcolor{Plum}{\skt{#1}}}}
\newcommand{\skttranspratyaya}[1]{{\sktpratyaya{#1}~\pratyaya{#1}}}

\newcommand{\reconstruction}[1]{\textcolor{gray}{*#1}}
\newcommand{\fullsentence}[1]{\textcolor{MidnightBlue}{#1}}
\newcommand{\sktsentence}[1]{\textcolor{MidnightBlue}{\skt{#1}}}

\newcommand{\veryimportant}[1]{\textcolor{red}{#1}}
\newcommand{\important}[1]{\textcolor{blue}{#1}}
\newcommand{\notsoimportant}[1]{\textcolor{gray}{#1}}
%-------------------------------------------词根等标颜色

\title{{梵语入门}}
\subtitle{15. 以辅音结尾的名词变格}
\author[张雪杉]{文学院~~张雪杉 \\ zhangxueshan@sdnu.edu.cn}
\date{}
%\institute{}

\begin{document}	

\begin{frame}
  \titlepage
\end{frame}

\begin{frame}
  \frametitle{本节内容}
  \small
  \tableofcontents
\end{frame}

\section{上节作业}

\begin{frame}{第14章练习4}
  %\small
  \begin{verse}
    \skt{1) yuddhārthāḥ kṣatriyā raṇamagacchan ।}\\
    \skt{2) vyāghrā aśvācchīghratarā aśvamaharan ।}\\
    \mbox{\skt{3) kṣatriyabalabhayātkumāro 'pādravat ।}}\\
    \skt{4) abhayāḥ kṣatriyā nidhanaṃ gacchanti·  iti dāsī·  avadat ।}\\
    \skt{5) sūryacandrau kṛtsnaṃ lokaṃ paśyataḥ ।}
  \end{verse}
\end{frame}

\begin{frame}{第14章练习4}
  \small
  \begin{verse}
    \skt{6) bhīmakeśaḥ siṃho bālau·  apādrāvayat ।}\\
    \skt{7) janā hataputraṃ śūraṃ saṃśucya mṛtānputrānpuramabharan ।}\\
    \skt{8) balārtho bālo vṛkṣānrohituṃ nadīkṣetrāṇi tartumiṣṭvā·  apādravat ।}\\
    \skt{9) atiguṇā kanyā varaṃ devebhya āharat ।}\\
    \skt{10) kumāro 'tirūpāṃ kanyāṃ dṛṣṭvā·  acintayatsā kanyā prabhāmukhā sūrya iva·  iti ।}
  \end{verse}
\end{frame}

\subsection{复习}
\begin{frame}{上节\insertsubsection }
  \small
  复合词传统分类:
  \begin{itemize}
    \item 并列复合词 (相违释 / Dvandva)
    \item 限定复合词 (依主释 / Tatpuruṣa)
    \begin{itemize}
      \item 同位复合词\\ (持业释 / Karmadhāraya)
    \end{itemize}
    \item 定语复合词 (多财释 / Bahuvrīhi)
  \end{itemize}
\end{frame}

\section[辅音变格]{以辅音结尾的名词变格}
\begin{frame}{\insertsection}
  \small
  \tableofcontents[currentsection]
\end{frame}

\subsection{标准格尾}
\begin{frame}{\insertsubsection}
  \small
  \centering
    \begin{NiceTabular}{|c|c|c|c|c|c|c|}[hvlines, rules/width=0.3pt, rules/color=gray]
       & \Block{1-2}{单数} & & \Block{1-2}{双数} & & \Block{1-2}{复数} &  \\
       & 阳阴 & 中 & 阳阴 & 中 & 阳阴 & 中  \\
      主 & \wordending{s} & \Block{3-1}{\wordending{}} & \Block{3-1}{\wordending{au}}  & \Block{3-1}{\wordending{ī}}  & \Block{3-1}{\wordending{aḥ}} &  \Block{3-1}{\wordending{ni}}   \\
      呼 & \wordending{} & & & & & \\
      业 & \wordending{am} &  & & & & \\
      具 & \Block{1-2}{\wordending{ā}} &  & \Block{3-2}{\wordending{bhyām}}  & & \Block{1-2}{\wordending{bhiḥ}} & \\
      为 & \Block{1-2}{\wordending{e}} &  & & & \Block{2-2}{\wordending{bhyaḥ}}  & \\
      从 & \Block{2-2}{\wordending{aḥ}} &  & & & & \\
      属 & &  & \Block{2-2}{\wordending{oḥ}} & &  \Block{1-2}{\wordending{ām}} & \\
      依 & \Block{1-2}{\wordending{i}} &  & & & \Block{1-2}{\wordending{su}}&  \\
    \end{NiceTabular}
\end{frame}

\subsection{齿音词干}
\begin{frame}{\insertsubsection}
  \small
  \begin{itemize}
    \item 以 t 或 d 结尾的词干。
    \item 加标准格尾,只需注意连声问题。
    \begin{itemize}
      \item 词尾辅音在元音格尾前保持不变,\\ \fullpada{marutā} ~~\fullpada{āpadā}
      \item 在浊辅音语尾前浊化,\\ \fullpada{marudbhiḥ}~~ \fullpada{āpadbhiḥ}
      \item 在清辅音语尾前清化。\\ \fullpada{marutsu}~~ \fullpada{āpatsu}
    \end{itemize}
  \end{itemize}
\end{frame}

\begin{frame}{例词~~ \nounstem{marut} (m.)}
  \small
  \centering
  \resizebox{\textwidth}{!}{
  \begin{tabular}{|l|l|l|l|}
    \hline
    & \textbf{Sg} & \textbf{Du} & \textbf{Pl} \\
    \hline
    \textbf{Nom} & \important{marut} & marutau & marutaḥ \\
    \textbf{Acc} & marutam & ... & ... \\
    \textbf{Instr} & marutā & \important{marudbhyām} & \important{marudbhiḥ} \\
    \textbf{Dat} & marute & ... & \important{marudbhyaḥ} \\
    \textbf{Abl/Gen} & marutaḥ & ... & ... \\
    \textbf{Abl/Gen} & ... & marutoḥ & marutām \\
    \textbf{Loc} & maruti & ... & \important{marutsu} \\
    \hline
  \end{tabular}
  }
  \bigskip
  
  \nounstem{āpad} (f.) 变化完全相同。
\end{frame}

\begin{frame}{例词~~ \nounstem{jagat} (n.)}
  \small
    \centering
  中性词仅在主业呼格不同。
  \bigskip
  

  \begin{tabular}{|l|l|l|l|}
    \hline
    & \textbf{Sg} & \textbf{Du} & \textbf{Pl} \\
    \hline
    \textbf{Nom/Acc} & jagat & jagatī & jaganti \\
    \hline
  \end{tabular}
  \bigskip
  
  复数插入鼻音 \pratyaya{\nobreakdash-n\nobreakdash-} (类似 \fullpada{phalāni})。
\end{frame}

\subsection{腭音词干}
\begin{frame}{\insertsubsection}
  \small
  \begin{itemize}
    \item 以 c 或 j 结尾的词干。
    \item 加标准格尾,连声比齿音略微复杂。
    \begin{itemize}
      \item 词尾辅音在元音格尾前保持不变,\\ \fullpada{vācā} ~~\fullpada{yajñabhujā}
      \item 在词尾或辅音格尾前腭音变喉音,\\ \fullpada{vāk}~~ \fullpada{yajñabhuk}
      \item 然后清浊根据辅音格尾变化。\\ \fullpada{vāgbhiḥ}~~ \fullpada{yajñabhugbhih}\\\fullpada{vākṣu}~~ \fullpada{yajñabhukṣu}
    \end{itemize}
  \end{itemize}
\end{frame}

\begin{frame}{例词~~ \nounstem{vāc} (f.)}
  \small
  \centering
  %\resizebox{\textwidth}{!}{
  \begin{tabular}{|l|l|l|l|}
    \hline
    & \textbf{Sg} & \textbf{Du} & \textbf{Pl} \\
    \hline
    \textbf{Nom} & \important{vāk} & vācau & vācaḥ \\
    \textbf{Acc} & vācam & ... & ... \\
    \textbf{Instr} & vācā & \important{vāgbhyām} & \important{vāgbhiḥ} \\
    \textbf{Dat} & vāce & ... & \important{vāgbhyaḥ} \\
    \textbf{Abl} & vācaḥ & ... & ... \\
    \textbf{Gen} & ... & vācoḥ & vācām \\
    \textbf{Loc} & vāci & ... & \important{vākṣu} \\
    \hline
  \end{tabular}
  %}
\end{frame}

\begin{frame}{例词~~ \nounstem{yajñabhuj} (m./n.)}
  \small
  \begin{itemize}
    \item 阳性变化和 \nounstem{vāc} 相同,
    \item   中性词仅在主业呼格不同。
  \end{itemize}
    \centering
  \begin{tabular}{|l|l|l|l|}
    \hline
    & \textbf{Sg} & \textbf{Du} & \textbf{Pl} \\
    \hline
    \textbf{Nom/Acc} & yajñabhuk & yajñabhujī & yajñabhuñji  \\
    \hline
  \end{tabular}

\end{frame}

\section{复合词续}
\begin{frame}{\insertsection}
  \small
  \tableofcontents[currentsection]
\end{frame}

\subsection{词根复合词}
\begin{frame}{\insertsubsection}
  \small 
  词根作为依主释的后词,具有分词含义。
  \begin{itemize}
    \item 辅音结尾的词根直接做后词,\\
    \nounstem{yajñabhuj} 享受祭祀~~\nounstem{vedavid} 懂吠陀的 \\
    \item 以非a/ā元音结尾的词根常加t。\\
    \nounstem{yuddhajit} 战胜~~ \nounstem{lokakṛt} 创世 \\

    \item 以ā结尾的词根ā变a,\\
    \nounstem{sarvajña} 全知~~\nounstem{rathastha} 站在车上 \\
    \item 以鼻音结尾的词根鼻音消失。\\
    \nounstem{kulaja} 生于好家族~~\nounstem{khaga} 在空中走,鸟\\
  \end{itemize}
\end{frame}

\subsection{sa- 多财释}
\begin{frame}{sa- 构成的多财释}
  \small 
  \begin{itemize}
    \item \pratyaya{sa\nobreakdash-} 加在名词前形成多财释复合词。
    \item 和动词前缀 \nounstem{sam} 和副词 \fullpada{saha} 相关。\\
    \item 修饰名词时可译为“和”或“有”。\\
    \fullsentence{nṛpaḥ sabhāryaḥ puraṃ gacchati} \\
    国王和王后去城市。\\
    \item 中性单数业格可做副词。\\
    \fullpada{sakrodham} 气愤地\\
  \end{itemize}
\end{frame}
\section{本节作业}

\begin{frame}{\insertsection }
  \begin{itemize}
    \item
      第15章练习 5
    \item
      阅读教材第15课相关内容
    \bigskip
    \item
      现在请做学习通\nobreakdash-章节\nobreakdash-课后问卷
  \end{itemize}
\end{frame}

\end{document}