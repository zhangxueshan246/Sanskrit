%%% XeLaTeX-article %%%
%# -*- coding: utf-8 -*-
%!TEX encoding = UTF-8 Unicode
%!TEX TS-program = xelatex  
%---------------------虽然加了%还是要保留!

\documentclass[17pt]{beamer}
\mode<presentation>
{
\usetheme[width=40pt]{Hannover}
\usecolortheme[]{dove}
\usefonttheme[]{structurebold}
\setbeameroption{hide notes}
}

\usepackage{fontspec}
\setmainfont{Arial} %设置主字体
\newfontfamily\sanskritfont[Script=Devanagari,Mapping=romantodevanagari,Scale=1.15]{Sanskrit 2003}             %输出天城体
%\newfontfamily\sanskritfont[Mapping=tex-text]{Times New Roman}              %输出转写
\doublehyphendemerits=-10000
\newcommand{\skt}[1]{{\sanskritfont{#1}}} %输出天城体
\newcommand{\skttrans}[1]{{\skt{#1}~#1}}  %输出天城体和转写
%----------------------------------------------------设置梵文输入方法 danda । ॥

\usepackage[UTF8,fontset=windows]{ctex}
\usepackage{amsmath}
%----------------------------------------------------设置中文环境

\usepackage{graphicx}
\usepackage{flafter} 
\graphicspath{{pic/}}
\usepackage{booktabs} 
\usepackage{nicematrix}
%-----------------------------------------插图表格

\usepackage{hyperref} 
\usepackage[dvipsnames]{xcolor}
\usepackage{colortbl}
\definecolor{light-gray}{gray}{0.9}
%------------------------------颜色

\newcommand{\verbroot}[1]{\textcolor{red}{$\sqrt{}$#1}}
\newcommand{\sktroot}[1]{{\verbroot{\skt{#1}}}}
\newcommand{\skttransroot}[1]{{\sktroot{#1}~\textcolor{red}{#1}}}

\newcommand{\nounstem}[1]{\textcolor{red}{#1\nobreakdash-}}
\newcommand{\sktnounstem}[1]{{\textcolor{red}{\skt{#1\nobreakdash-}}}}
\newcommand{\skttransnounstem}[1]{{\sktnounstem{#1}~\nounstem{#1}}}

\newcommand{\verbstem}[1]{\textcolor{blue}{#1\nobreakdash-}}
\newcommand{\sktverbstem}[1]{{\textcolor{blue}{\skt{#1\nobreakdash-}}}}
\newcommand{\skttransverbstem}[1]{{\sktverbstem{#1}~\verbstem{#1}}}

\newcommand{\wordending}[1]{\textcolor{Orange}{\nobreakdash-#1}}
\newcommand{\sktending}[1]{{\textcolor{Orange}{\skt{-#1}}}}
\newcommand{\skttransending}[1]{{\sktending{#1}~\wordending{#1}}}

\newcommand{\fullpada}[1]{\textcolor{OliveGreen}{#1}}
\newcommand{\sktpada}[1]{{\textcolor{OliveGreen}{\skt{#1}}}}
\newcommand{\skttranspada}[1]{{\sktpada{#1}~\fullpada{#1}}}

\newcommand{\pratyaya}[1]{\textcolor{Plum}{#1}}
\newcommand{\sktpratyaya}[1]{{\textcolor{Plum}{\skt{#1}}}}
\newcommand{\skttranspratyaya}[1]{{\sktpratyaya{#1}~\pratyaya{#1}}}

\newcommand{\reconstruction}[1]{\textcolor{gray}{*#1}}
\newcommand{\fullsentence}[1]{\textcolor{MidnightBlue}{#1}}

\newcommand{\veryimportant}[1]{\textcolor{red}{#1}}
\newcommand{\important}[1]{\textcolor{blue}{#1}}
\newcommand{\notsoimportant}[1]{\textcolor{gray}{#1}}
%-------------------------------------------词根等标颜色

\title{{梵语提高}}
\subtitle{22. 分词代词续}
\author[张雪杉]{文学院~~张雪杉 \\ zhangxueshan@sdnu.edu.cn}
\date{}
%\institute{}

\begin{document}	

\begin{frame}
  \titlepage
\end{frame}

\begin{frame}
  \frametitle{本节内容}
  \small
  \tableofcontents
\end{frame}

\section{上节作业}

\begin{frame}{第21章练习4}
  \raggedright
  \small
  \begin{verse}
    \skt{1) api sūryaḥ śvaḥ punardyotiṣyata iti kumāro bālayā pṛṣṭaḥ ।}   \\
    \skt{2) īśvarasya vacanāni śrutvā kopaṃ ca dṛṣṭvā janā vepiṣyante ।}   \\
    \skt{3) dānāni gṛhe bhriyanta iti lakṣayitvā bālo 'tuṣyat ।}   \\
    \skt{4) pāpā janānna vepayiṣyantītīśvaro bhāṣitvā yuddhaṃ prati rathenohyate ।}  \\
  \end{verse}
\end{frame}  

\begin{frame}{第21章练习4}
  \raggedright
  \small
  \begin{verse}
    \skt{5) kadā kumārāḥ puraṃ pratyaśvairvakṣyanta iti pṛṣṭveśvaro nyasīdat ।}   \\
    \skt{6) puramāptuṃ mitrāṇi ca moktuṃ śakṣyāma iti śūrā bhāṣante ।}   \\
    \skt{7) kṣetraṃ pratyeṣyāmaḥ pāpāṃśca kṣaṇena lapsyāmaha ityuktvā kṣatriyaḥ kumārānapānayat ।}  \\
    \skt{8) puraṃ lokapālaiḥ pālayiṣyata iti matvā janāḥ sukhā bhavanti ।}   \\
  \end{verse}
\end{frame}  

\begin{frame}{第21章练习4}
  \raggedright
  \small
  \begin{verse}
    \skt{9) na cintayā krodhena vābhibhaviṣya ityuktvā kṣatriyo yuddhameti ।} \\
    \skt{10) naro bhāryāyai ratnaṃ dāsyati ।}   \\
    \skt{11) vṛkṣānroheva । kṛtsnaṃ nagaraṃ draṣṭuṃ śakṣyāva iti bālau bhāṣete ।}   \\
    \skt{12) api vane sthitaṃ naraṃ nekṣadhva iti pṛṣṭvā kumārāvaśvau labdhvāpavahete ।}   \\
  \end{verse}
\end{frame}  

\begin{frame}{第21章练习4}
  \raggedright
  \small
  \begin{verse}
    \skt{13) aśvaṃ yuñjyā nagaraṃ ca prati vaheriti narau bālamavadatām ।}  \\
    \skt{14) yuddhe yoddhuṃ bhadramastīti matvāśvaṃ ca yuktvā kumāraḥ kṣatriyaiḥ saha yuddhamāyāt ।}  \\
    \skt{15) kṣatriya ugreṇa marutā rathādapohyate । avapatya hanyate ।}  \\
    \skt{16) nara āpadā hataḥ suhṛdbhiḥ śucyate ।}   \\
    \skt{17) sukhāḥ syāmeti devā janaiḥ pṛcchyante ।}   \\
  \end{verse}
\end{frame}  

\subsection{复习}
\begin{frame}{\insertsubsection}
  \begin{itemize}
    \item 将来时
    
    词根二合 + 将来时标志 \pratyaya{\nobreakdash-sya\nobreakdash-/\nobreakdash-iṣya\nobreakdash-} + 原始语尾
    \item 中间语态原始语尾
    
    {\small
    \begin{tabular}{@{}cccc@{}} % 6 columns
       & 单 & 双 & 复  \\
      1st &  \wordending{e} & \wordending{vahe}  & \wordending{mahe}  \\
      2nd & \wordending{se} & \wordending{(e/ā)the} & \wordending{dhve}   \\
      3rd & \wordending{te} & \wordending{(e/ā)te} & \wordending{a(n)te}  \\
    \end{tabular}
    }
    \item 现在时被动语态
    
    词根零级 + \pratyaya{\nobreakdash-ya\nobreakdash-} + 中间语态语尾  
  \end{itemize}
\end{frame}

\section{分词续}
\begin{frame}{\insertsection }
    \small
    \tableofcontents[currentsection]
\end{frame}

\subsection{现在时中间语态}
\begin{frame}{\insertsubsection 分词}
  \begin{itemize}
    \item 现在时语干 + \pratyaya{\nobreakdash-māna\nobreakdash-/\nobreakdash-āna\nobreakdash-}  
    \begin{itemize}
      \item 带插入元音语干 + \pratyaya{\nobreakdash-māna\nobreakdash-}  
      
      \verbroot{dyut(1Ā)} 照亮 \verbstem{dyota} \nounstem{dyotamāna}
      \item 非插入元音弱语干 + \pratyaya{\nobreakdash-āna\nobreakdash-}
      
      \verbroot{bhuj(7Ā)} 享用 \verbstem{bhuñj} \nounstem{bhuñjāna}
      \item 特殊
      
      \verbroot{ās(2Ā)} 坐 \verbstem{ās} \nounstem{āsīna}
    \end{itemize}
  \end{itemize}
\end{frame}

\subsection{现在时被动语态}
\begin{frame}{\insertsubsection 分词}
  \begin{itemize}
    \item 现在时被动语干 + \pratyaya{\nobreakdash-māna\nobreakdash-}  

    \verbroot{nī(1)} 带领 \nounstem{nīyamāna}\\
    \verbroot{īkṣ(1Ā)} 观看 \nounstem{īkṣyamāna}\\
    \verbroot{śru(5)} 听 \nounstem{śrūyamāna}\\
    \verbroot{dā(3)} 给予 \nounstem{dīyamāna}
  \end{itemize}
\end{frame}

\subsection{将来时中间语态}
\begin{frame}{\insertsubsection 分词}
  \begin{itemize}
    \item 将来时语干 + \pratyaya{\nobreakdash-māna\nobreakdash-}  

    \verbroot{bhāṣ(1Ā)} 说 \nounstem{bhāṣiṣyamāna}\\
    \verbroot{yudh(4Ā)} 观看 \nounstem{yotsyamāna}\\
    \verbroot{īkṣ(1Ā)} 观看 \nounstem{īkṣiṣyamāna}

    \item 将来时中间语态和被动语态分词\\形式相同
  \end{itemize}
\end{frame}

\begin{frame}{分词例句}
  \begin{itemize}
    \item 现在时中间语态\\
    {\small
    \fullsentence{bālā nāryā bhāṣamānāyāḥ vacanāni śṛṇoti }\\}
    \mbox{\footnotesize
    The girl listens to the words of the speaking woman. 
    }
    \item 现在时被动语态
    
    {\small
    \fullsentence{tataḥ sarvaiḥ pūjyamāna āgacchat }\\}
    {\footnotesize
    Then came the man who was worshipped by all. }

    \item 将来时中间/被动语态\\
    {\small
    \fullsentence{senāṃ pṛtanāyāṃ yotsyamānāṃ vetti }\\}
    {\footnotesize
    He knows/recognises the army that is about to fight in the battle.  }
  \end{itemize}
\end{frame}


\section{代词续}

\subsection{tad式代词}
\begin{frame}{按照 \nounstem{saḥ/tad} 变化的代词}
  \begin{itemize}
    \item \nounstem{eṣaḥ/etad} 指示代词这、那 (ruki)
    \item \nounstem{ya} 关系代词 (复合词用 \nounstem{yad})
    
    \nounstem{yatkula} of which family
    \item \nounstem{ka} 疑问代词 (复合词用 \nounstem{kim})
    
    \nounstem{kimartha} for what purpose
    \item \nounstem{anya} 另一个 (复合词用 \nounstem{anya} 或 \nounstem{anyad})
  \end{itemize}
\end{frame}

\subsection{代词式形容词}
\begin{frame}{按代词变化的形容词}
  \begin{itemize}
    \item 按照 \nounstem{saḥ/tad} 变化的
    
    \nounstem{sarva} \nounstem{viśva} 所有 
    
    \nounstem{sva} 自己的 \nounstem{eka} 一个
    
    但是中性单数主业格是 \fullpada{sarvam} 等
    \item 按照 \nounstem{saḥ/tad} 和形容词正常变化都可以的

    \nounstem{para} 远的 \nounstem{pūrva} 过去的

    \mbox{\nounstem{katara} 两个中哪一个 \nounstem{uttama} 最高的}
    \item \notsoimportant{多为表示关系而非性质的形容词}
  \end{itemize}
\end{frame}


\section{本节作业}

\begin{frame}{\insertsection }
  \begin{itemize}
    \item
      第22章练习4
  \end{itemize}
\end{frame}  

\end{document}	
